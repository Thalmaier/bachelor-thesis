\def\insideAnhang{0}

\newenvironment{anhang}{%
	\ifnum\insideAnhang=1%
	\errhelp={Let other blocks end at the beginning of the next block.}
	\errmessage{Nested Alpha section not allowed}
	\fi%
	\def\insideAnhang{1}
	
	\renewcommand\figurename{%
		\ifnum\insideAnhang=1% 
		Anhang\else%
		Abbildung\fi%
	}%
	
	\ifnum\insideAnhang=1%
	\def\table{\def\figurename{Anhang}\figure}
	\let\endtable\endfigure
	\fi%
	
	\renewcommand\thefigure{%
		\ifnum\insideAnhang=1% 
		\Alph{figure}\else%
		\arabic{figure}\fi%
	}%
	
}{%
	\def\insideAnhang{0}
}%

%%%%%%%%%%%%%%%%%%%%%%%%%%%%%%%%%%%%%%%%%%%%%%%%
%%%%%%%%%%%%%%%%%%%%%%%%%%%%%%%%%%%%%%%%%%%%%%%%
%%%%%%%%%%%%%%%%%%%%%%%%%%%%%%%%%%%%%%%%%%%%%%%%
%%%%%%%%%%%%%%%%%%%%%%%%%%%%%%%%%%%%%%%%%%%%%%%%

\begin{anhang}

\chapter{Anhang}

\section{Sourcecode des Proof-of-Concepts}
\label{label:sourcecode}

Die Binärdateien des Projekts wurden zur Versionierung in ein Git-Repository hinterlegt. Diese umfassen die Definition der Lasttests, API-Beschreibung, und den Quelltext, inklusive der analysierten Aggregationsschnitte. Das Repository kann unter dem Link '\url{https://github.com/Thalmaier/bachelor-thesis-checkout-poc}' aufgerufen werden. Alternativ ist in Anhang \ref{fig:Github} ein QR-Code abgebildet.

\begin{figure}[h]
	\centering
	\vspace{0.8cm}
	\includesvg[inkscapelatex=false, width=100px, height=100px]{svg/github qr code.svg}
	\caption{QR-Code des GitHub Repositories}
	\small URL: \hspace{0.3mm} \url{https://github.com/Thalmaier/bachelor-thesis-checkout-poc}
	\label{fig:Github}
\end{figure}

\newpage

\section{Aktivitätsdiagramme der Anwendungsfälle}

\begin{figure}[h!]
	\centering
	\includesvg[inkscapelatex=false, width=\textwidth]{svg/AD Basketcreation.svg}
	\caption{Aktivitätsdiagramm für die Erstellung eines Baskets}
	\label{fig:SL-Basketcreation}
\end{figure}

\begin{figure}[h!]
	\centering
	\includesvg[inkscapelatex=false, width=\textwidth]{svg/AD Basketstornierung.svg}
	\caption{Aktivitätsdiagramm für die Stornierung eines Baskets}
	\label{fig:SL-Basketstornierung}
\end{figure}

\begin{figure}[h!]
	\centering
	\includesvg[inkscapelatex=false, width=\textwidth]{svg/AD Checkoutdata.svg}
	\caption{Aktivitätsdiagramm für das Setzen der Checkout Daten}
	\label{fig:SL-Checkoutdata}
\end{figure}

\begin{figure}[h!]
	\centering
	\includesvg[inkscapelatex=false, width=\textwidth]{svg/AD PutBezahlmethode.svg}
	\caption{Aktivitätsdiagramm für das Hinzufügen einer Bezahlmethode}
	\label{fig:SL-PutBezahlmethode}
\end{figure}

\phantom{}
\newpage

\section{API-Endpunkte}

Die OpenAPI Definition der API ist in Anhang \ref{fig:Github} hinterlegt. Zudem bietet bietet Anhang \ref{fig:REST-API} eine grafische Übersicht der Endpunkte für Variante A des Aggregate-Designs.

\begin{figure}[h!]
	\centering
	\includesvg[inkscapelatex=false, width=\textwidth, height=0.85\textheight]{svg/REST API.svg}
	\caption{REST-API der Checkout-Software für diesen Proof-of-Concept}
	\label{fig:REST-API}
\end{figure}

\newpage
\section{Vollständiges Datenmodell des Proof-of-Concepts} \label{label:Daten-Modell}

\groupedDomainModell{Basket}{
	\item \textbf{BasketId: } {Eindeutige Identifikation des Baskets zur Referenzierung durch die Touchpoints.}
	\item \textbf{OutletId: } {Eine Referenz zugehörig zu dem Markt oder Onlineshop, durch welchen der Basket angelegt wurde. Unerlässlich für die Bestimmung von unter anderem Lagerbeständen, Lieferzeiten, Fulfillment-Optionen und Versandkosten.}
	\item \textbf{BasketStatus: } {Repräsentiert den aktuellen Zustands des Baskets. Mögliche Werte sind 'OPEN', 'FROZEN', 'FINALIZED' und 'CANCELED'.} 
	\item \textbf{Customer: } {Speichert Kundendaten (IdentifiedCustomer) oder Session-Informationen (SessionCustomer).} 
	\item \textbf{FulfillmentType: } {Lieferart, wie 'PICKUP' oder 'DELIVERY'.} 
	\item \textbf{BillingAddress: } {Adresse für die Rechnungserstellung.} 
	\item \textbf{ShippingAddress: } {Adresse für die Warenlieferung.} 
	\item \textbf{BasketItems: } {Liste aller enthaltenen Produkten und ihren zugehörigen Informationen.}
	\item \textbf{BasketCalculationResult: } {Beinhaltet die berechneten Werte des Basket, wie Nettobetrag, Bruttobetrag und Mehrwertsteuer. Die Speicherung dieser Werte wäre technisch nicht notwendig, spart aber an Rechenzeit ein, da nicht bei jeder Abfrage des Basket dieser Wert neu berechnet werden muss.}
	\item \textbf{PaymentProcess: } {Bindet alle Informationen zur erfolgreichen Abwicklung des Zahlungsprozesses.}
	\item \textbf{Order: } {Speichert eine Referenz auf die Bestellung für einen Basket. Wird erst nach Zahlungsabschluss befüllt.}  
}

\groupedDomainModell{SessionCustomer}{
	\item \textbf{SessionID: } {Eindeutige ID zur Zuweisung einer Session im Onlineshop zum zugehörigen Basket. Notwendig, um eine Einkaufmöglichkeit für anonyme Kunden zu bieten.}
}

\groupedDomainModell{IdentifiedCustomer}{
	\item \textbf{Name: } {Enthält den Vor- und Nachnamen als eigenes Datenkonstrukt.}
	\subDomainModell{
		\item \textit{FirstName: } {Vorname des Kunden.}
		\item \textit{LastName: } {Nachname des Kunden.}
	}
	\item \textbf{E-Mail: } {E-Mail des Kunden.}
	\item \textbf{CustomerTaxId: } {Die Steuernummer des Kunden. Relevant aus Sicht der Rechnungsabwicklung und für den Ausdruck der Rechnung.} 
	\item \textbf{BusinessType: } {Bestimmt ob Kunde als Business-to-Customer (B2C) oder Business-to-Business (B2B) gilt.}
	\item \textbf{CompanyName: } {Firmenname des Kunden. Kann optional angegeben werden oder ist verpflichtend für einen B2B-Kunden.}
	\item \textbf{CompanyTaxId: } {Steuernummer der Firma eines B2B-Kunden.}
}

\groupedDomainModell{BasketItem}{
	\item \textbf{Id: } {Eindeutige Referenz auf den Warenkorbeintrag.}
	\item \textbf{Product: } {Beinhaltet alle Produktinformationen, welche durch die Touchpoints benötigt werden.}
	\item \textbf{Price: } {Aktueller Preis des zugehörigen Products. Kann sich zeitlich ändern, muss daher durch eine Businessfunktion aktualisiert werden. }
	\item \textbf{ShippingCost: } {Betrag der Lieferkosten des Items.}
	\item \textbf{BasketItemCalculationResult: } {Speichert die Bruttokosten des Produktes, die errechneten Nettokosten, Lieferkosten und den Gesamtpreis.}
}

\groupedDomainModell{Product}{
	\item \textbf{Id: } {Eindeutige Referenz des Products im externen System.}
	\item \textbf{Name: } {Textuelle Produktbezeichnung des Products.}
	\item \textbf{Vat: } {Mehrwertsteuerinformationen des Products.}
	\item \textbf{UpdatedAt: } {Zeitstempel notwendig für die Aktualisierungsfunktion der Artikelinformationen.}
}

\groupedDomainModell{Vat}{
	\item \textbf{Sign: } {Identifizierung des Steuertyps, abhängig von jeweiligen Prozentsatz und Land.}
	\item \textbf{Rate: } {Prozentualer Wert der Mehrwertsteuer, wie beispielsweise '19\%'.}
}

\groupedDomainModell{Price}{
	\item \textbf{PriceId: } {Setzt sich zusammen aus der ProductId und der OutletId.}
	\item \textbf{GrossAmount: } {Bruttobetrag mit Währung.}
	\item \textbf{UpdatedAt: } {Zeitstempel notwendig für die Aktualisierungsfunktion des Preises.}
}

\groupedDomainModell{BasketItemCalculationResult}{
	\item \textbf{ItemCost: } {Beinhaltet Netto, Brutto und VAT Informationen in Form eines CalculationResults.}
	\item \textbf{ShippingCost: } {Betrag der Lieferkosten.}
	\item \textbf{TotalCost: } {Zusammengerechnete Werte der einzelnen Preise im Form eines CalculationResults.}
}

\groupedDomainModell{CalculationResult}{
	\item \textbf{GrossAmount: } {Bruttobetrag mit Währung.}
	\item \textbf{NetAmount: } {Nettobetrag mit Währung.}
	\item \textbf{VatAmounts: } {Eine zusammengebautes Set aus VatAmounts der Preise der BasketItems. Benötigt, da Vats mit unterschiedlichen Prozentbeträgen rechtlich nicht kombiniert werden dürfen.}
}

\groupedDomainModell{VatAmount}{
	\item \textbf{Sign: } {Identifizierung des Steuertyps, abhängig von genauen Prozentsatz und zugehörigen Land.}
	\item \textbf{Rate: } {Prozentualer Wert der Mehrwertsteuer.}
	\item \textbf{Amount: } {Berechneter Betrag der Mehrwertsteuer zugehörig zu einem Bruttobetrag.}
}

\groupedDomainModell{BasketCalculationResult}{
	\item \textbf{GrandTotal: } {Betrag der finalen Gesamtkosten des ganzen Baskets.}
	\item \textbf{NetTotal: } {Fasst alle Nettobeträge zusammen in einem einzelnen Betrag.}
	\item \textbf{ShippingTotal: } {Fasst alle Lieferkosten zusammen in einem einzelnen Betrag.}
	\item \textbf{VatAmount: } {Rechnet alle Vats zusammen, welche das gleiche Sign besitzen.}
}

\groupedDomainModell{PaymentProcess}{
	\item \textbf{BasketId: } {Id des zugehörigen Baskets.}
	\item \textbf{ExternalPaymentRef: } {Referenz auf den Bezahlvorgangs im externen System. Anfangs leer bis zur Initiierung des Payments.}
	\item \textbf{AmountToPay: } {Betrag der insgesamt bezahlt werden muss. Entspricht dem GrandTotal des Baskets.}
	\item \textbf{AmountPayed: } {Rechnet alle Payments zusammen und bestimmt in welchem Maße der Basket bereits bezahlt ist.}
	\item \textbf{AmountToReturn: } {Falls der bezahlte Betrag größer ist als gefordert, wird dieser Wert berechnet. Repräsentiert den Betrag, welcher durch das System zurückgegeben werden muss.}
	\item \textbf{PaymentProcessStatus: } {Status wieweit der der AmountToPay bezahlt ist. Kann die Werte 'TO\_PAY', 'PARTIALLY\_PAID' und 'PAID' annehmen.}
	\item \textbf{Payment: } {Liste aller Payments zugehörig zu diesem Prozess.}
}

\groupedDomainModell{Payment}{
	\item \textbf{PaymentId: } {Die Id der Zahlung.}
	\item \textbf{PaymentMethod: } {Bezahlungsart, wie Gutschein oder Barzahlung.}
	\item \textbf{PaymentStatus: } {Aktueller Zustand des Payments. Mögliche Werte entsprechen 'SELECTED', 'INITIALIZED', 'EXECUTED', 'CANCELED'. Ein Payment ist bei Hinzufügung im Status 'SELECTED'.}
	\item \textbf{AmountSelected: } {Betrag, welcher durch dieses Payment bezahlt werden soll. Falls dieser Wert leer ist, wird der gesamte Warenkorb durch dieses Payment bezahlt.}
	\item \textbf{AmountUsed: } {Betrag wie viel insgesamt durch dieses Payment abgedeckt wurde, falls nur ein Bruchteil des AmountSelectes benötigt wird.}
	\item \textbf{AmountOverpaid: } {Berechnet durch Subtraktion von AmountSelected und AmountUsed.}
}

\groupedDomainModell{Order}{
	\item \textbf{OrderRef: } {Referenz auf die Bestellung des Warenkorbs. Wird bei Abschluss des Zahlungsprozesses gesetzt.}
}

\clearpage
\section{Klassendiagramme des Datenmodells}

\vspace{1cm}
\begin{figure}[htbp]
	\centering
	\includesvg[inkscapelatex=false, width=0.95\textwidth]{svg/ValueObjectBasketDiagram.svg}
	\caption{Klassendiagramm eines Baskets}
	\label{fig:VO-Basket}
\end{figure}

\begin{figure}[htbp]
	\centering
	\includesvg[inkscapelatex=false, width=0.95\textwidth]{svg/ValueObjectCustomerDiagram.svg}
	\caption{Klassendiagramm des Customer Value Objects}
	\label{fig:VO-Customer}
\end{figure}

\begin{figure}[htbp]
	\centering
	\includesvg[inkscapelatex=false, width=0.95\textwidth]{svg/ValueObjectPaymentDiagram.svg}
	\caption{Klassendiagramm des Payment Process}
	\label{fig:VO-Payment}
\end{figure}

\section{Ergebnisse des Lasttests} \label{label:Lasttests}

Der Lasttest wurde mithilfe der Software 'JMeter' durchgeführt. Die Vorlage der jeweiligen Testfälle sind im Repository des Proof-of-Concepts unter Anhang \ref{fig:Github} zu finden. Ein Durchlauf bezieht sich auf einen typischen User Story, welche folgende Aspekte beinhaltet: Erstellen eines Baskets, dreimaliges Hinzufügen von Artikeln, Setzen der Checkout-Daten, zweimaliges Abrufen des Warenkorbs, Hinzufügen eines Payments und das Initiieren inklusive Durchführen des Bezahlvorgangs. 

Folgende Abkürzungen wurde zur Übersichtlichkeit genutzt:
\begin{itemize}[noitemsep,nolistsep]
	\item \textbf{A}: 'Variante A' des Aggregationsschnittes
	\item \textbf{D}: 'Variante D' des Aggregationsschnittes
	\item \textbf{M}: Verwendung des Datenbankmanagementsystems MongoDB
	\item \textbf{P}: Verwendung des Datenbankmanagementsystems PostgreSQL
	\item \textbf{F}: Kurz für 'Flag'. Die Kalkulation des Gesamtpreises geschieht erst bei expliziter Abfrage
	\item \textbf{C}: Die Kalkulation des Gesamtpreises findet umgehend bei Anpassungen von relevanten Werten statt.
	\item \textbf{AZ}: Kurz für 'Ablaufzeit'. Angabe in Millisekunden. Die Zeit für einen einzelnen Ablauf des Anwendungsfalles. 
\end{itemize}


\begin{landscape}
\begin{table}[h!]
	\centering
	\small
	\vspace{1cm}
	\begin{tabular}{ |c|c|c|c|c|c|c|c|} 
		\hline
		Name & Anzahl Durchläufe & min. AZ & max. AZ & durchschn. AZ & Median der AZ & Durchläufe pro Sekunde & Testdauer in Millisekunden \\ 
		\hline
		Variante A-M & 100 & 16 & 53 & 28,66 & 29 & 92,08 & 1086,00 \\
		\hline
		Variante A-M & 100 & 17 & 51 & 24,70 & 25 & 88,42 & 1131,00 \\
		\hline
		Variante A-M & 100 & 16 & 53 & 28,19 & 25 & 91,07 & 1098,00 \\
		\hline
		Variante A-M & 1000 & 16 & 83 & 43,56 & 42 & 197,36 & 5067,00 \\
		\hline
		Variante A-M & 1000 & 15 & 92 & 41,56 & 40 & 207,13 & 4828,00 \\
		\hline
		Variante A-M & 1000 & 16 & 86 & 41,53 & 41 & 205,00 & 4878,05 \\
		\hline
		Variante A-M & 10000 & 15 & 78 & 37,67 & 36 & 259,24 & 38574,00 \\
		\hline
		Variante A-M & 10000 & 15 & 76 & 36,86 & 35 & 265,32 & 37690,00 \\
		\hline
		Variante A-M & 10000 & 15 & 78 & 36,78 & 35 & 266,16 & 37571,00 \\
		\hline
		Variante D-FM & 100 & 21 & 55 & 29,10 & 29 & 88,18 & 1134,00 \\
		\hline
		Variante D-FM & 100 & 22 & 58 & 31,03 & 30 & 86,43 & 1157,00 \\
		\hline
		Variante D-FM & 100 & 22 & 60 & 33,25 & 31 & 86,73 & 1153,00 \\
		\hline
		Variante D-FM & 1000 & 21 & 75 & 42,46 & 41 & 199,60 & 5010,02 \\
		\hline
		Variante D-FM & 1000 & 22 & 83 & 44,97 & 44 & 189,47 & 5278,00 \\
		\hline
		Variante D-FM & 1000 & 21 & 80 & 43,41 & 42 & 195,69 & 5110,00 \\
		\hline
		Variante D-FM & 10000 & 21 & 73 & 37,00 & 35 & 264,63 & 37788,00 \\
		\hline
		Variante D-FM & 10000 & 20 & 97 & 36,82 & 35 & 266,36 & 37543,01 \\
		\hline
		Variante D-FM & 10000 & 20 & 72 & 36,83 & 35 & 265,82 & 37619,99 \\
		\hline
		Variante D-CM & 100 & 19 & 54 & 25,22 & 24 & 90,33 & 1107,00 \\
		\hline
		Variante D-CM & 100 & 18 & 60 & 28,40 & 27 & 90,42 & 1106,00 \\
		\hline
		Variante D-CM & 100 & 18 & 63 & 27,24 & 28 & 91,32 & 1095,00 \\
		\hline
		Variante D-CM & 1000 & 19 & 72 & 39,67 & 38 & 211,77 & 4722,00 \\
		\hline
		Variante D-CM & 1000 & 18 & 69 & 40,13 & 39 & 211,46 & 4729,00 \\
		\hline
		Variante D-CM & 1000 & 17 & 79 & 38,92 & 38 & 215,52 & 4640,00 \\
		\hline
		Variante D-CM & 10000 & 17 & 93 & 32,82 & 31 & 297,78 & 33582,00 \\
		\hline
		Variante D-CM & 10000 & 17 & 89 & 33,17 & 31 & 293,90 & 34025,00 \\
		\hline
		Variante D-CM & 10000 & 17 & 91 & 33,42 & 31 & 292,53 & 34185,00 \\
		\hline
	\end{tabular}
	\caption{Analyseergebnis des Lasttests der verschiedenen Variationen in Kombination mit MongoDB}
	\label{fig:performance-mongo}
\end{table}
\end{landscape}

\begin{landscape}
	\begin{table}[h!]
		\centering
		\small
		\vspace{1cm}
		\begin{tabular}{ |c|c|c|c|c|c|c|c|} 
			\hline
			Name & Anzahl Durchläufe & min. AZ & max. AZ & durchschn. AZ & Median der AZ & Durchläufe pro Sekunde & Testdauer in Millisekunden \\ 
			\hline
			Variante A-P & 100 & 37 & 65 & 48,08 & 48 & 76,75 & 1303,00 \\
			\hline
			Variante A-P & 100 & 34 & 77 & 55,08 & 55 & 73,58 & 1359,00 \\
			\hline
			Variante A-P & 100 & 35 & 80 & 55,04 & 54 & 70,95 & 1409,45 \\
			\hline
			Variante A-P & 1000 & 32 & 149 & 60,15 & 57 & 148,39 & 6739,00 \\
			\hline
			Variante A-P & 1000 & 33 & 133 & 60,02 & 57 & 147,21 & 6793,00 \\
			\hline
			Variante A-P & 1000 & 34 & 118 & 59,83 & 59 & 148,35 & 6741,00 \\
			\hline
			Variante A-P & 10000 & 33 & 93 & 57,28 & 57 & 170,35 & 58703,02 \\
			\hline
			Variante A-P & 10000 & 33 & 109 & 55,62 & 55 & 175,18 & 57083,00 \\
			\hline
			Variante A-P & 10000 & 33 & 138 & 55,61 & 55 & 174,23 & 57395,99 \\
			\hline
			Variante D-FP & 100 & 42 & 93 & 63,38 & 63 & 69,93 & 1430,00 \\
			\hline
			Variante D-FP & 100 & 42 & 90 & 63,53 & 63 & 70,13 & 1426,00 \\
			\hline
			Variante D-FP & 100 & 39 & 169 & 69,44 & 64 & 69,49 & 1439,00 \\
			\hline
			Variante D-FP & 1000 & 39 & 170 & 62,82 & 60 & 143,66 & 6961,00 \\
			\hline
			Variante D-FP & 1000 & 37 & 148 & 60,63 & 59 & 148,13 & 6751,00 \\
			\hline
			Variante D-FP & 1000 & 38 & 88 & 61,54 & 60 & 143,55 & 6966,00 \\
			\hline
			Variante D-FP & 10000 & 37 & 135 & 56,06 & 55 & 174,40 & 57338,00 \\
			\hline
			Variante D-FP & 10000 & 41 & 113 & 58,71 & 58 & 166,68 & 59994,98 \\
			\hline
			Variante D-FP & 10000 & 38 & 116 & 56,99 & 56 & 172,22 & 58063,98 \\
			\hline
			Variante D-CP & 100 & 31 & 72 & 47,26 & 44 & 78,93 & 1267,00 \\
			\hline
			Variante D-CP & 100 & 31 & 70 & 45,93 & 45 & 77,82 & 1285,00 \\
			\hline
			Variante D-CP & 100 & 30 & 74 & 46,81 & 46 & 80,13 & 1248,00 \\
			\hline
			Variante D-CP & 1000 & 30 & 78 & 48,06 & 47 & 180,86 & 5529,00 \\
			\hline
			Variante D-CP & 1000 & 30 & 105 & 50,88 & 49 & 170,68 & 5859,00 \\
			\hline
			Variante D-CP & 1000 & 30 & 75 & 50,10 & 50 & 173,97 & 5748,00 \\
			\hline
			Variante D-CP & 10000 & 34 & 94 & 48,87 & 49 & 199,60 & 50099,00 \\
			\hline
			Variante D-CP & 10000 & 34 & 111 & 49,95 & 50 & 195,57 & 51132,01 \\
			\hline
			Variante D-CP & 10000 & 34 & 93 & 48,28 & 48 & 200,92 & 49771,99 \\
			\hline
		\end{tabular}
		\caption{Analyseergebnis des Lasttests der verschiedenen Variationen in Kombination mit Postgres}
		\label{fig:performance-postgres}
	\end{table}
\end{landscape}

\begin{landscape}
	\begin{table}[h!]
		\centering
		\small
		\vspace{1cm}
		\begin{tabular}{ |c|c|c|c|c|c|c|c|} 
			\hline
			Name & Anzahl Durchläufe & min. AZ & max. AZ & durchschn. AZ & Median der AZ & Durchläufe pro Sekunde & Testdauer in Minuten\\ 
			\hline
			Variante A-M & 10000 & 563 & 1526 & 612,04 & 612 & 16,30 & 10,22 \\
			\hline
			Variante A-M & 10000 & 565 & 725 & 617,67 & 617 & 16,06 & 10,38 \\
			\hline
			Variante A-M & 10000 & 576 & 743 & 619,05 & 617 & 16,04 & 10,39 \\
			\hline
			Variante D-FM & 10000 & 1405 & 2904 & 1625,89 & 1612 & 6,13 & 27,19 \\
			\hline
			Variante D-FM & 10000 & 1510 & 1720 & 1608,00 & 1608 & 6,19 & 26,91 \\
			\hline
			Variante D-FM & 10000 & 1512 & 1970 & 1654,91 & 1646 & 6,01 & 27,72 \\
			\hline
			Variante D-CM & 10000 & 1210 & 3307 & 1380,43 & 1364 & 7,23 & 23,06 \\
			\hline
			Variante D-CM & 10000 & 1207 & 1675 & 1384,97 & 1368 & 7,17 & 23,26 \\
			\hline
			Variante D-CM & 10000 & 1219 & 1637 & 1374,96 & 1365 & 7,21 & 23,11 \\
			\hline
			Variante A-P & 10000 & 3354 & 4237 & 3848,23 & 3836 & 2,49 & 66,96 \\
			\hline
			Variante A-P & 10000 & 3402 & 4305 & 3769,70 & 3765 & 2,59 & 64,46 \\
			\hline
			Variante A-P & 10000 & 3364 & 4911 & 3833,21 & 3796 & 2,36 & 70,67 \\
			\hline
			Variante D-FP & 10000 & 5386 & 6919 & 5839,03 & 5804 & 1,66 & 100,49 \\
			\hline
			Variante D-FP & 10000 & 5168 & 6883 & 5778,68 & 5725 & 1,52 & 109,65 \\
			\hline
			Variante D-FP & 10000 & 5235 & 6174 & 5727,97 & 5720 & 1,67 & 99,97 \\
			\hline
			Variante D-CP & 10000 & 4189 & 5799 & 4571,81 & 4549 & 2,11 & 79,14 \\
			\hline
			Variante D-CP & 10000 & 3919 & 5553 & 4553,76 & 4554 & 2,12 & 78,61 \\
			\hline
			Variante D-CP & 10000 & 4000 & 5702 & 4544,34 & 4527 & 2,02 & 82,69 \\
			\hline
		\end{tabular}
		\caption{Lasttest-Ergebnisse mit Datenbanken von einem externen Cloud-Anbieter}
		\label{fig:performance-database}
	\end{table}
\end{landscape}



\begin{landscape}
	\begin{table}[h!]
		\centering
		\small
		\vspace{1cm}
		\begin{tabular}{ |c|c|c|c|c|c|c|c|} 
			\hline
			Name & Anzahl Durchläufe & min. AZ & max. AZ & durchschn. AZ & Median der AZ & Durchläufe pro Sekunde & Testdauer in Minuten\\ 
			\hline
			Variante A-M & 10000 & 488 & 605 & 504,39 & 503 & 19,68 & 8,47 \\
			\hline
			Variante A-M & 10000 & 489 & 585 & 502,43 & 501 & 19,71 & 8,45 \\
			\hline
			Variante A-M & 10000 & 490 & 604 & 502,69 & 501 & 19,73 & 8,45 \\
			\hline
			Variante D-FM & 10000 & 494 & 604 & 511,01 & 511 & 19,39 & 8,59 \\
			\hline
			Variante D-FM & 10000 & 494 & 600 & 509,20 & 508 & 19,38 & 8,60 \\
			\hline
			Variante D-FM & 10000 & 492 & 601 & 508,31 & 507 & 19,50 & 8,55 \\
			\hline
			Variante D-CM & 10000 & 490 & 575 & 503,62 & 502 & 19,69 & 8,46 \\
			\hline
			Variante D-CM & 10000 & 491 & 627 & 503,50 & 502 & 19,68 & 8,47 \\
			\hline
			Variante D-CM & 10000 & 491 & 571 & 505,90 & 504 & 19,61 & 8,50 \\
			\hline
			Variante A-P & 10000 & 507 & 589 & 522,61 & 522 & 18,97 & 8,78 \\
			\hline
			Variante A-P & 10000 & 507 & 630 & 519,33 & 518 & 19,08 & 8,73 \\
			\hline
			Variante A-P & 10000 & 507 & 602 & 521,18 & 520 & 19,03 & 8,76 \\
			\hline
			Variante D-FP & 10000 & 511 & 594 & 526,00 & 525 & 18,86 & 8,84 \\
			\hline
			Variante D-FP & 10000 & 511 & 600 & 526,29 & 525 & 18,82 & 8,85 \\
			\hline
			Variante D-FP & 10000 & 512 & 626 & 526,81 & 525 & 18,80 & 8,87 \\
			\hline
			Variante D-CP & 10000 & 504 & 580 & 517,00 & 516 & 19,14 & 8,71 \\
			\hline
			Variante D-CP & 10000 & 502 & 611 & 518,79 & 518 & 19,11 & 8,72 \\
			\hline
			Variante D-CP & 10000 & 505 & 594 & 518,93 & 518 & 19,09 & 8,73 \\
			\hline
		\end{tabular}
	\caption{Lasttest-Ergebnisse mit Simulation der API-Aufruf durch künstliche Verzögerung}
	\label{fig:performance-delay}
	\end{table}
\end{landscape}

\end{anhang}
