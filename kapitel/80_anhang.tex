
%TODO: Chapter mit Nummer im Inhaltsverzeichnis oder nicht?
\chapter{Anhang}

\section{Ergebnisse des Lasttests}

%TODO: Beschreibung von Werten. Angabe von Bedeutung von AZ und dass alles in ms ist

\begin{landscape}
\begin{table}[h!]
	\centering
	\small
	\begin{tabular}{ |c|c|c|c|c|c|c|c|c|} 
		\hline
		Name & Pre-Daten & Anfragen & min. AZ & max. AZ & durchschn. AZ & Median der AZ & Anfragen/Sekunde & Testdauer \\ 
		\hline
		Variante A-M & 3000 & 100 & 14 & 46 & 18,48 & 16,5 & 94,88 & 1054 \\ 
		\hline
		Variante D-M & 3000 & 100 & 14 & 46 & 18,48 & 16,5 & 94,88 & 1054 \\ 
		\hline
		Variante D-MC & 3000 & 100 & 14 & 46 & 18,48 & 16,5 & 94,88 & 1054 \\ 
		\hline
		\hline
	\end{tabular}
	\caption{Analyseergebnis des Lasttests der verschiedenen Variationen}
	\label{fig:performance-mongo}
\end{table}
\end{landscape}

\begin{landscape}
	\begin{table}[h!]
		\centering
		\small
		\begin{tabular}{ |c|c|c|c|c|c|c|c|c|} 
			\hline
			Name & Pre-Daten & Anfragen & min. AZ & max. AZ & durchschn. AZ & Median der AZ & Anfragen/Sekunde & Testdauer \\ 
			\hline
			Variante A-P & 3000 & 100 & 14 & 46 & 18,48 & 16,5 & 94,88 & 1054 \\ 
			\hline
			Variante D-P & 3000 & 100 & 14 & 46 & 18,48 & 16,5 & 94,88 & 1054 \\ 
			\hline
			Variante D-PO & 3000 & 100 & 14 & 46 & 18,48 & 16,5 & 94,88 & 1054 \\ 
			\hline
		\end{tabular}
		\caption{Analyseergebnis des Lasttests der verschiedenen Variationen}
		\label{fig:performance-postgres}
	\end{table}
\end{landscape}