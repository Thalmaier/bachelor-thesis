
%% EINLEITUNG %% 

\chapter{Einleitung}

Der Schwerpunkt dieses Projekts und somit zugleich dieser Bachelorarbeit ist die Entwicklung eines Proof-of-Concepts (\acrshort{POC}) einer Checkout-Software unter Einsatz von Domain-Driven Design und Hexagonaler Architektur mit speziellen Fokus auf den Aggregationsschnitt des Datenmodells. In der folgenden Einleitung wird diese Problemstellung detaillierter beschrieben, sowie die dahinterliegende Motivation und Ziele erläutert. Hierdurch wird ein grundlegendes Verständnis der Hintergründe dieses Projektes geschaffen.

\section{Problemstellung}

Ein elementarer Bestandteil der Funktionsweisen eines Onlineshops erfüllt der sogenannte Warenkorb. In diesem können, unter anderem, Waren abgelegt werden, um sie zu einem späteren Zeitpunkt zu erwerben oder weitere Service hinzuzufügen. Im Verlauf des Kaufprozesses sollte es zudem möglich sein eine Versandart einzustellen, Kundendaten zu hinterlegen und gewünschte Zahlungsarten auszuwählen. Nach erfolgreicher Überprüfung von Sicherheitsrichtlinien findet die Kaufabwicklung statt, der sogenannte 'Checkout'. Um die hier beschriebenen Anwendungsfälle zu verwirklichen, wird eine eigens dafür designierte Software benötigt. In diesem Projekt wird eine solche Anwendung vereinfacht implementiert und als 'Checkout-Software' bezeichnet.

Eine solche Applikation stellt das Rückgrat des Onlineshops dar. Sie erfährt stetige Änderungen, besitzt eine Vielzahl von Businesslogik und ihre Einbindung in das Frontend beeinflusst weitergehend auch das Kundenerlebnis. Dadurch liegt ein hoher Fokus auf die Erfüllung von Qualitätsmerkmalen, wie Stabilität, Testbarkeit und Wartbarkeit. Der Checkout-Prozess, welcher durch diese Software abgewickelt wird, muss für alle relevanten Länder und ihre individuellen gesetzlichen Voraussetzungen fiskalisch korrekt ausgeführt werden. Jederzeit können neue Businessanforderungen entstehen, wodurch weitere länderspezifische Richtlinien in den Zuständigkeitsbereich der Anwendung fallen oder eine Anpassung des Datenmodells erfordern. Dementsprechend wird ein flexibles Datenkonstrukt benötigt, welches zugleich performant und übersichtlich bleibt. Das \note{Datenmodell}{Wortwiederholung mit Datenkonstrukt. Sehr oft 'model'.} bestimmt weitestgehend wie die externen Systeme mit den internen Softwarekomponenten interagieren. Eine Umfeldanalyse und die klare Definition von Anwendungsfällen besitzen dadurch einen hohen Einfluss auf die resultierende Qualität des Entwicklungsprozesses. Die verwendete Architektur und das Softwaredesign sollte Entwicklern bei der Erfüllung dieser Kriterien unterstützen. 

Zur Erfüllung dieser Kriterien existieren etablierte Vorgehensmodelle und Architekturen. Ein Teilaspekt der Problemstellung besteht in der Auswahl eines geeigneten Ansatzes zur Realisierung der Software. Hierbei wird auf Basis der nachfolgenden Kapiteln die Verwendung von Hexagonaler Architektur inklusive Domain-Driven Design für diese Arbeit argumentativ begründet. Die Projektdurchführung orientiert sich an den empfohlenen Entwicklungsprozess innerhalb eines Domain-Driven Kontextes. Im Zentrum der Bachelorarbeit stehen die möglichen Aggregationsschnitte des Datenmodells. Als Forschungsfrage bildet sich heraus, welche Auswirkungen unterschiedliche Aggregate-Designs auf die Software und ihre Funktionalität besitzen. Zur Veranschaulichung und praktischen Umsetzung der zugehörigen Antwort wird eine konkrete Implementierung in Form eines Proof-of-Concepts umgesetzt.

\comment{Ist die Problemstellung klar und ihre Darstellung so akzeptabel?}
\comment{Hinleitung zum Projektumfeld sinnvoll?}


\section{Das Unternehmen MediaMarktSaturn}

Dieses Projekt wurde in dankbarer Zusammenarbeit mit dem Unternehmen \emph{MediaMarktSaturn Retail Group}, kurz \emph{MediaMarktSaturn}, erarbeitet.

Als größte Elektronik-Fachmarktkette Europas bietet MediaMarktSaturn in über 1023 Märkten und 13 Ländern den Kunden die Erwerbsmöglichkeit einer Vielzahl unterschiedlicher Artikel. Dabei wird ein großer Wert auf ein technologisch neuartiges Kundenerlebnis gelegt, um ein positives Einkaufserlebnis zu garantieren. Die Zugehörigkeiten der Märkte ist in den Marken \emph{Media Markt} und \emph{Saturn} unterteilt.

Über die Jahre gewann der Onlineshop für Media Markt und Saturn zunehmend an Bedeutung, da die prozentuale Verteilung des jährlichen Gewinns in den Märkten zurückgegangen und im Onlineshop gestiegen ist. Als Folge wurden die Unternehmensziele dementsprechend auf die Entwicklung von komplexer Software zur Unterstützung des Onlineshops neu ausgelegt. \emph{MediaMarktSaturn Technology} ist eine Tochtergesellschaft der MediaMarktSaturn Retail Group und zuständig für alle Entwicklungstätigen des Unternehmens. Dank den 705 internen Mitarbeiter am Standort Ingolstadt kann eine einwandfreie Benutzererfahrung der Kunden erzielt werden. 

Die Durchführung und Implementierung des Projektes bzw. Proof-of-Concepts geschah in Kooperation mit dem Team \emph{Chechkout \& Payment}. Die sechs zugehörigen Teammitglieder sind zuständig einen unternehmensweiten universellen Checkout für alle Länder bereitzustellen, sowohl für den Onlineshops als auch im Markt oder per Handyapp. Durch den Einsatz von \Gls{Scrum} wird eine konstante Verbesserung der Applikation und des Arbeitsprozesses erzielt. In kontinuierlichen \Glspl{Sprint} wird zusätzlich die Checkout-Software auf Basis von hinzukommende Anwendungsfälle stetig expandiert. Dieses Projekt soll dem Team als Revision dienen und Aufschlüsse über mögliche architektonische Designansätze darbieten.


\section{Motivation}


Durch den fortlaufend Anstieg der Komplexität von Softwareprojekten haben sich gängige Designprinzipien und Architekturstile für den Entwicklungsprozess etabliert, sodass auch weiterhin die Businessanforderungen in einem zukunftssicheren Ansatz erfüllt und die multiple Prozessabläufen jederzeit angepasst und erweitert werden können. Dadurch ist zur Gewährleistung der Langlebigkeit einer solchen Software eine flexible Grundstruktur entscheidend. Da der Checkout ein wichtiger Bestandteil eines jeden Onlineshops ist, besitzt die Software für MediaMarktSaturn eine zentraler Bedeutung. Folglich ist eine sorgfältige Projektplanung und stetige Revision der bestehenden Software relevant, um auch weiterhin einen reibungslosen Ablauf der Geschäftsprozesse zu ermöglichen. Zur Erreichung dieses Ziels verwendet die zum aktuellen Zeitpunkt bestehende Anwendung des Checkout-Teams eine Hexagonale Architektur und Domain-Driven Design.

Dieses Projekt hilft somit bei der Überprüfung der bestehenden Architektur auf Verbesserungsmöglichkeiten und eventuelle Schwachstellen. Zudem existieren aufgrund des jetzigen zugrundeliegenden Aggregationsschnitts Performance-Einbusen. In diesem Projekt wird analysiert, ob die Performance durch einen andere Aufteilung des Datenmodells und einem damit verbundenen vertretbaren Aufwand gesteigert werden kann. Dies dient ebenfalls als nützliche Untersuchung der bestehenden Anwendung und kann als Reverenz für zukünftige Softwareprojekte verwendet werden, da viele Projekte mit ähnlichen Problemstellungen konfrontiert sind.



\section{Ziele}

Aus den vorhergehenden Motivationen lassen sich folgenden Projektziele ableiten. Grundlegend stellt diese Arbeit einen Anhaltspunkt für neue Softwareprojekte und Mitarbeiter dar. Dies kann zu einem erhöhten Grad an Softwarequalität im Unternehmen beitragen. Zugleich wird durch die Analyse und Durchführung des Proof-of-Concepts das bestehende Softwaredesign überprüft und herausgefordert. Dadurch können konkrete Verbesserungsvorschläge ein mögliches Fazit der Arbeit sein. Womöglich ergeben sich jedoch keine sinnvollen Änderungen der Produktivanwendung. Letzteres stellt dennoch eine wichtige Erkenntnis für das Team und Unternehmen dar, denn zukünftige Projekte können mithilfe der verwendeten Vorgehensmodelle ähnliche Ergebnisse erzielen. Sollten sich durch einen anderen Aggregationsschnitt erhebliche Vorteile bilden, kann ein Resultat dieses Projektes dem Umbau der Software entsprechen.
