
%% EINLEITUNG %% 

\chapter{Einleitung}

Heutzutage werden Applikationen für den langfristigen Gebrauch in einer produktiver Umgebung entwickelt. Im Durchschnitt kann hierbei der erwartete Lebenszyklus dieser Anwendungen auf circa 10 Jahre festgelegt werden \cite{Tamai.1992}, weshalb sich über die vergangenen Jahrzehnte ein starker Fokus auf ein wartbares und flexibles Softwaredesign gebildet hat. Darüber hinaus müssen sie umfangreiche Anwendungsprofile fehlerfrei und performant bewältigen, wodurch ebenfalls ein stabiler Architekturansatz von Nöten ist \cite{Bosch.2001}. Software Engineers stehen daher mittlerweile eine große Bandbreite an Entwurfsmustern und Anti-Pattern zur Verfügung, wie beispielsweise die Hexagonale Architektur, Event-Storming, \emph{\Gls{Anemic Domain Model}} und Microservices. Letzteres gewann aufgrund ihrer Skalierbarkeit und losen Kopplung der Komponenten in den vergangenen Jahren zunehmend an Bedeutung \cite{oreilly.Microservices, Sampaio.2017}. Gleichzeitig erfuhr das im Jahre 2011 erschienene Buch \citetitle{Evans.2011} von \citeauthor{Evans.2011} an Beliebtheit, da viele der hierin bearbeiteten Themenschwerpunkte auf Microservices adaptiert werden können \cite[S. 130ff.]{Vernon.2015}\cite{Microservice.DDD.2017}. Domain-Driven Design bietet zum Lösen gängiger Problemstellungen in der Softwareentwicklung ein Vorgehensmodell für die Realisierung eines businessorientierten Datenmodells, das bei der Umsetzung von Businessanforderungen unterstützt. Die Checkout-Software, welche im Rahmen dieser Bachelorarbeit als Proof-of-Concept entwickelt wird, profitiert von den Vorteilen einer solchen Architektur.

\section{Problemstellung}

Eine elementare Funktion eines Onlineshops ist der Warenkorb. In diesem können Waren zum späteren Erwerb abgelegt oder zusätzliche Dienstleistungen hinzugefügt werden. Im Verlauf des Kaufprozesses ist es zudem möglich, Kundendaten zu hinterlegen, sowie die gewünschte Versandart und Zahlungsarten auszuwählen. Nach erfolgreicher Überprüfung der Validierungsrichtlinien findet die Kaufabwicklung statt – der sogenannte 'Checkout'. Um die zuvor genannten Anwendungsfälle umzusetzen, wird eine Softwarekomponente benötigt. In dieser Bachelorarbeit wird diese Anwendung vereinfacht implementiert und als 'Checkout-Software' bezeichnet. Sie erfährt stetige Änderungen, beinhaltet im Vergleich zu anderen Softwareprojekten umfangreiche Businesslogik und ihre Antwortzeiten haben durch die Einbindung in das Frontend auch direkten Einfluss auf das Kundenerlebnis. Dadurch liegt stets ein Fokus auf die Erfüllung von Qualitätsanforderungen, wie Stabilität, Testbarkeit und Wartbarkeit. Der Checkout-Prozess muss für alle Länder, in denen die Applikation eingesetzt wird, und ihre individuellen gesetzlichen Anforderungen fiskalisch korrekt ausgeführt werden. Jederzeit können neue Businessanforderungen entstehen, wodurch weitere länderspezifische Richtlinien in den Zuständigkeitsbereich der Anwendung fallen und beispielsweise eine Anpassung des Datenmodells erfordern. Zudem benötigt das System zur Abwicklung ihrer Arbeitsprozesse Daten aus verschiedensten Unternehmensbereichen wie Preise, Lieferkosten, Produkt- und Bestandsinformationen. Die Kommunikation mit externen Komponenten erhöht die Komplexität der Anwendung und erfordert die Berücksichtigung systemübergreifender Anforderungen. Eine große Rolle spielt hierbei die Performance. Vor allem bei hoher Auslastung, beispielsweise während Kampagnen, muss das Gesamtsystem weiterhin zuverlässig alle Anfragen in akzeptabler Zeit abarbeiten können. Dementsprechend stellt die Implementierung einer solchen Checkout-Applikation für Software Engineers eine große Herausforderung dar. Sollte die verwendete Architektur im langjährigen Entwicklungsprozess an Übersichtlichkeit verlieren, leidet zugleich auch die Wartbarkeit des Sourcecodes. Als Folge können weitere Qualitätsmerkmale negativ betroffen sein und der Aufwand zur Umsetzung von neuen Businessanforderungen steigt \cite[S. 3f.]{Pigoski.2001}. Daraus ergibt sich für den Checkout der grundsätzliche Bedarf zur Einhaltung bestimmter Industriestandards hinsichtlich der Softwarearchitektur und des Datenmodells. Somit besteht ein Teilaspekt der Problemstellung in der Auswahl einer geeigneten Architektur zur Realisierung der Software. 

Auf Basis der kommenden Kapitel wird die Verwendung einer Hexagonalen Architektur inklusive Domain-Driven Design für den Proof-of-Concept argumentativ begründet. Die Projektdurchführung orientiert sich hierbei am empfohlenen Entwicklungsprozess von Domain-Driven Design. Im Fokus der Bachelorarbeit stehen die möglichen Einteilungen des Datenmodells in Aggregates, in dieser Arbeit als 'Aggregationsschnitt' bezeichnet, welche durch die Untersuchung der Anwendungsfälle erschlossen werden. Das Forschungsthema leitet sich aus der Frage ab, welche funktionalen und nicht-funktionalen Implikationen die unterschiedlichen Aggregationsschnitte auf die Applikationen besitzen. Hierzu werden sie in Form eines Proof-of-Concepts implementiert, analysiert und anhand von Performance-Tests bewertet.

\section{Das Unternehmen MediaMarktSaturn}

Dieses Projekt wurde in Zusammenarbeit mit dem Unternehmen \emph{MediaMarktSaturn Retail Group GmbH}, kurz \emph{MediaMarktSaturn}, erarbeitet. Mit über 1.000 Märkten in 13 Ländern sowie den Onlineshops ist MediaMarktSaturn Europas größter Anbieter von Unterhaltungselektronik sowie zugehöriger Dienstleistungen und Services. Dabei soll eine umfangreiche Produktauswahl in Kombination mit passenden Services und Kundennähe ein einzigartiges Einkaufserlaubnis schaffen - über alle Verkaufskanäle hinweg. Die Zugehörigkeiten der Märkte ist hierbei in die Marken \emph{Media Markt} und \emph{Saturn} unterteilt. \cite{mms.Unternehmen}

Wegen des massiv steigenden Onlineanteils gewann der Onlineshop in den letzten Jahren für Media Markt und Saturn zunehmend an Bedeutung. Der Ausbruch der COVID-19-Pandemie und die damit verbundene europaweite Schließung der Märkte hat die Verlagerung der Umsatzeinnahmen vom Markt- zum Onlinegeschäft nochmals verschärft. Folglich wurden die Unternehmensziele auf die Entwicklung komplexer Software zur Unterstützung des Onlineshops neu ausgelegt. Die Umsetzung obliegt der zentralen IT-Gesellschaft MediaMarktSaturn Technology, die über 700 interne und 1000 externe Engineers in einer skalierten Produktorganisation mit dem Ziel der steten Optimierung der Systemlandschaft beschäftigt \cite{mms.technology}.

Die Durchführung des Projektes bzw. Proof-of-Concepts erfolgte in Kooperation mit dem Bereich \emph{Checkout \& Payment}. Die sechs Teammitglieder sind dafür zuständig einen unternehmensweiten, universellen Checkout für alle Länder bereitzustellen – sowohl für den Onlineshop als auch im Markt und per Handyapp. Durch den Einsatz von \emph{\Gls{Scrum}} wird eine konstante Verbesserung der Applikation und des Arbeitsprozesses erzielt. In iterativen \emph{\Glspl{Sprint}} wird die Checkout-Software auf Basis von neuen Anwendungsfällen kontinuierlich erweitert. Dieses Projekt soll dem Team als Revision dienen und Schlussfolgerungen über mögliche architektonische Designansätze liefern.

\pagebreak

\section{Motivation}

Durch den fortlaufenden Anstieg der Komplexität von Softwareprojekten \cite{Darcy.2010} haben sich gängige Designprinzipien und Architekturstile für den Entwicklungsprozess etabliert, sodass auch weiterhin die Businessanforderungen in einem zukunftssicheren Ansatz erfüllt und die Prozessabläufe jederzeit angepasst und erweitert werden können. Deshalb ist zur Gewährleistung der Langlebigkeit einer solchen Software eine flexible Grundstruktur entscheidend. Folglich ist eine sorgfältige Projektplanung und stetige Revision der Produktivanwendung relevant, um auch weiterhin einen reibungslosen Ablauf der Geschäftsprozesse zu ermöglichen. Zur Erreichung dieses Ziels verwendet die zum aktuellen Zeitpunkt bestehende Anwendung des Checkout-Teams eine Hexagonale Architektur und Domain-Driven Design. Dieses Projekt überprüft die Architektur auf Verbesserungsmöglichkeiten und eventuelle Schwachstellen. Zudem existieren aufgrund des aktuellen Aggregationsschnitts gewisse Nachteile hinsichtlich der Performance und gleichzeitigen Bearbeitungen von Ressourcen. In diesem Projekt wird analysiert, ob die Performance durch eine andere Aufteilung des Datenmodells nachhaltig gesteigert werden kann. Dies dient ebenfalls als Referenz für zukünftige Softwareprojekte, die mit ähnlichen Problemstellungen konfrontiert sind.



\section{Ziele}

Aus der im vorgehenden Kapitel definierten Motivation lassen sich folgenden Projektziele ableiten. Anhand der Analyse und Durchführung des Proof-of-Concepts wird das bestehende Softwaredesign überprüft und herausgefordert. Dabei können konkrete Verbesserungsvorschläge an die produktive Applikation ein mögliches Fazit der Arbeit sein. Sollten sich durch ein anderes Design des Datenmodells erhebliche Vorteile ergeben, kann dies in einem Umbau der Software resultieren. Die Erkenntnisse der Arbeit sind ein wichtiges Ergebnis für das Team und Unternehmen, denn Projekte können auf ihrer Basis eine Architektur und Datenmodell nachhaltig implementieren.
