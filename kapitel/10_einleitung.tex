
%% EINLEITUNG %% 

\chapter{Einleitung}

Heutzutage werden Applikationen für den langfristigen Gebrauch in produktiver Umgebung entwickelt. Im Durchschnitt kann hierbei der erwartete Lebenszyklus dieser Anwendungen auf circa 10 Jahre festgelegt werden \cite{Tamai.1992}, weshalb sich über die vergangenen Jahrzehnte ein starker Fokus auf ein wartbares und flexibles Softwaredesign gebildet hat. Darüber hinaus erfüllen sie umfangreiche Anwendungsprofile und müssen diese fehlerfrei und performant bewältigen, wodurch ebenfalls ein stabiler Architekturansatz von Nöten ist \cite{Bosch.2001}. Dementsprechend stehen Software Engineers mittlerweile eine große Bandbreite an Entwurfsmustern und Anti-Pattern zur Verfügung, wie beispielsweise die Hexagonale Architektur, Event-Storming, Anemic Domain Model und Microservices. Letzteres gewann aufgrund ihrer Skalierbarkeit und losen Kopplung der Komponente in den vergangenen Jahren zunehmend an Bedeutung \cite{oreilly.Microservices, Sampaio.2017}. Hiermit einhergehend, erfuhr das im Jahre 2011 erschiene Buch \citetitle{Evans.2011} von \citeauthor{Evans.2011} an Beliebtheit, da viele der bearbeiteten Themenschwerpunkte auf Microservices adaptiert werden können \cite[S. 130ff.]{Vernon.2015}\cite{Microservice.DDD.2017}. Domain-Driven Design stellt Entwicklern zum Lösen gängiger Problemstellungen der Softwareentwicklung unter anderem ein Vorgehensmodell für die Realisierung eines businessorientierten Datenmodells bereit, das bei der Umsetzung von Businessanforderungen unterstützt. Eine Checkout-Software, welche in dieser Bachelorarbeit als Proof-of-Concept entwickelt wird, profitiert von den Vorteilen einer solchen Architektur. \comment{Letzte zwei Sätze auch Quellen?}

\comment{Verwendung von Aggregate und Aggregationsschnitt}
\comment{Nur Abbildung X und Anhang X oder auch Variationen?}

\section{Problemstellung}

\comment{Publikation von Architekturen in Onlineshops PAPERS}

Ein elementarer Bestandteil der Funktionsweisen eines Onlineshops ist der Warenkorb. In diesem können zum späteren Erwerb Waren abgelegt oder zusätzliche Dienstleistungen hinzugefügt werden. Im Verlauf des Kaufprozesses ist es zudem möglich, Kundendaten zu hinterlegen, sowie die gewünschte Versandart und Zahlungsarten auszuwählen. Nach erfolgreicher Überprüfung mittels Validierungsrichtlinien findet die Kaufabwicklung statt, der sogenannte 'Checkout'. Um die zuvor genannten Anwendungsfälle zu verwirklichen, wird eine eigens dafür geschriebene Software benötigt. In dieser Bachelorarbeit wird diese Anwendung vereinfacht implementiert und als 'Checkout-Software' bezeichnet. Sie erfährt stetige Änderungen, besitzt \note{im Vergleich zu anderen Softwareprojekten eine großen Umfang an Businesslogik}{Vielzahl von} und die Einbindung in das Frontend hat durch ihre Antwortzeit einen Einfluss auf das Kundenerlebnis. Dadurch liegt stets ein Fokus auf die Erfüllung von Qualitätsmerkmalen, wie Stabilität, Testbarkeit und Wartbarkeit. Der Checkout-Prozess, welcher durch diese Software abgewickelt wird, muss für alle Länder, in welche die Applikation operiert, und ihre individuellen gesetzlichen Anforderungen fiskalisch korrekt ausgeführt werden. Jederzeit können neue Businessanforderungen entstehen, wodurch weitere länderspezifische Richtlinien in den Zuständigkeitsbereich der Anwendung fallen und beispielsweise eine Anpassung des Datenmodells erfordern. Zudem erfordert das System zur Abwicklung ihrer Arbeitsprozesse Daten aus verschiedensten Unternehmensbereiche wie Preise, Lieferkosten, Produkt- und Bestandsinformationen. Die Kommunikation mit externen Komponenten erhöht die Komplexität der Anwendung und erfordert die Berücksichtigung systemübergreifender Anforderungen. Eine große Rolle spielt hierbei die Performance. Vor allem bei hoher Auslastung, beispielsweise während Kampagnen, muss das Gesamtsystem weiterhin zuverlässig alle Anfrage in akzeptabler Zeit abarbeiten können. Dementsprechend stellt die Implementierung einer solchen Checkout-Applikation für Software Engineers eine große Herausforderung dar. Sofern die verwendete Architektur im langjährigen Entwicklungsprozess an Übersichtlichkeit verliert, leidet zugleich auch die Wartbarkeit des Sourcecodes darunter. Als Folge können weitere Qualitätsmerkmale negativ betroffen sein und der Aufwand zur Umsetzung von Businessanforderungen steigt an \cite[S. 3f.]{Pigoski.2001}. Aus diesen Gründen hat der Checkout einen Bedarf \note{zur}{nicht "an"} Befolgung eines bestimmten Industriestandards hinsichtlich der Softwarearchitektur und des Datenmodells. Somit besteht ein Teilaspekt der Problemstellung in der Auswahl einer geeigneten Architektur zur Realisierung der Software. 

Auf Basis der kommenden Kapitel wird die Verwendung einer Hexagonaler Architektur inklusive Domain-Driven Design für den Proof-of-Concept argumentativ begründet. Die Projektdurchführung orientiert sich hierbei am empfohlenen Entwicklungsprozess von Domain-Driven Design. Im Fokus der Bachelorarbeit stehen die möglichen Einteilungen des Datenmodells in Aggregates, in dieser Arbeit als 'Aggregationsschnitt' bezeichnet, welche \note{primär}{weglassen?} durch eine Untersuchung der Anwendungsfälle erschlossen werden. Das Forschungsthema bildet sich aus der Frage, welche funktionalen und nicht-funktionalen Implikationen die unterschiedliche Aggregationsschnitte auf die Applikationen besitzen. Hierzu werden sie in Form eines Proof-of-Concepts implementiert, analysiert und anhand von Performance-Tests bewertet.

\section{Das Unternehmen MediaMarktSaturn}

Dieses Projekt wurde in Zusammenarbeit mit dem Unternehmen \emph{MediaMarktSaturn Retail Group GmbH}, kurz \emph{MediaMarktSaturn}, erarbeitet. Mit über 1.000 Märkten sowie den Onlineshops in 13 Ländern ist MediaMarktSaturn Europas größter Anbieter von Unterhaltungselektronik sowie zugehöriger Dienstleistungen und Services. Die umfangreiche Produktauswahl in Kombination mit passenden Services und Kundennähe schaffen ein einzigartiges Einkaufserlaubnis - über alle Verkaufskanäle hinweg. Die Zugehörigkeiten der Märkte ist hierbei in die Marken \emph{Media Markt} und \emph{Saturn} unterteilt. \cite{mms.Unternehmen}

Wegen des massiv steigenden Onlineanteils gewann der Onlineshop über die Jahre für Media Markt und Saturn zunehmend an Bedeutung. Der Ausbruch der Corona-Pandemie und die damit verbundene europaweite Schließung der Märkte hat die Verlagerung der Umsatzeinnahmen vom Markt- zum Onlinegeschäft nochmals verschärft. Als Folge dessen wurden die Unternehmensziele auf die Entwicklung komplexer Software zur Unterstützung des Onlineshops neu ausgelegt. Die Umsetzung obliegt der zentralen IT-Gesellschaft MediaMarktSaturn Technology, wo über 700 interne und 1000 externe Engineers in einer skalierten Produktorganisation an der steten Optimierung der Systemlandschaft arbeiten \cite{mms.technology}.

Die Durchführung des Projektes bzw. Proof-of-Concepts geschah in Kooperation mit dem Bereich \emph{Checkout \& Payment}. Die sechs zugehörigen Teammitglieder sind zuständig einen unternehmensweiten, universellen Checkout für alle Länder bereitzustellen, sowohl für den Onlineshop als auch im Markt und per Handyapp. Durch den Einsatz von \Gls{Scrum} wird eine konstante Verbesserung der Applikation und des Arbeitsprozesses erzielt. In kontinuierlichen \Glspl{Sprint} wird die Checkout-Software auf Basis von neuen Anwendungsfällen stetig erweitert. Dieses Projekt soll dem Team als Revision dienen und Aufschlüsse über mögliche architektonische Designansätze darbieten.


\section{Motivation}

Durch den fortlaufenden Anstieg der Komplexität von Softwareprojekten \cite{Darcy.2010} haben sich gängige Designprinzipien und Architekturstile für den Entwicklungsprozess etabliert, sodass auch weiterhin die Businessanforderungen in einem zukunftssicheren Ansatz erfüllt und die Prozessabläufe jederzeit angepasst und erweitert werden können. Deshalb ist zur Gewährleistung der Langlebigkeit einer solchen Software eine flexible Grundstruktur entscheidend. Folglich ist eine sorgfältige Projektplanung und stetige Revision der Produktivanwendung relevant, um auch weiterhin einen reibungslosen Ablauf der Geschäftsprozesse zu ermöglichen. Zur Erreichung dieses Ziels verwendet die zum aktuellen Zeitpunkt bestehende Anwendung des Checkout-Teams eine Hexagonale Architektur und Domain-Driven Design.

Dieses Projekt hilft somit bei der Überprüfung der Architektur auf Verbesserungsmöglichkeiten und eventuelle Schwachstellen. Zudem existieren aufgrund des jetzigen zugrundeliegenden Aggregate-Designs gewisse Nachteile hinsichtlich der Performance und gleichzeitig stattfindenden Bearbeitung von Ressourcen. In diesem Projekt wird analysiert, ob die Performance durch eine andere Aufteilung des Datenmodells und einem damit verbundenen \note{vertretbaren}{was ist vertretbar? entfernen falls keine Definition} Aufwand gesteigert werden kann. Dies dient ebenfalls als Reverenz für zukünftige Softwareprojekte, denn viele Projekte sind mit ähnlichen Problemstellungen konfrontiert.



\section{Ziele}

Aus den vorhergehenden Motivationen lassen sich folgenden Projektziele ableiten. Anhand der Analyse und Durchführung des Proof-of-Concepts wird das bestehende Softwaredesign überprüft und herausgefordert. Dadurch können konkrete Verbesserungsvorschläge an die produktive Applikation ein mögliches Fazit der Arbeit sein. Sollten sich durch ein anderes Design des Datenmodells erhebliche Vorteile bilden, kann dies in einem Umbau der Software resultieren. Die Erkenntnisse dieser Arbeit sind ein wichtige Ergebnis für das Team und Unternehmen, denn Projekte können auf ihrer Basis eine zukunftssichere Architektur und Datenmodell implementieren.
