
%% EINLEITUNG %% 

\chapter{Einleitung}

Heutzutage werden Applikationen für den langfristigen Gebrauch in produktiver Umgebung entwickelt. Im Durchschnitt kann hierbei der erwartete Lebenszyklus dieser Anwendungen auf circa 10 Jahre festgelegt werden \cite{Tamai.1992}, weshalb sich über die vergangenen Jahrzehnte ein starker Fokus auf ein wartbares und flexibles Softwaredesign gebildet hat. Darüber hinaus erfüllen sie umfangreiche Anforderungsprofile und müssen diese fehlerfrei und performant bewältigen, wodurch ebenfalls ein stabiler Architekturansatz von Nöten ist \cite{Bosch.2001}. Dementsprechend stehen Software Engineers mittlerweile eine große Bandbreite an Entwurfsmustern und Anti-Pattern zur Verfügung, wie beispielsweise Hexagonale Architektur, Event-Storming, Anemic Domain Model und Microservices. Letzteres gewann aufgrund ihrer Skalierbarkeit und losen Kopplung der Komponente in den vergangenen Jahren zunehmender Bedeutung \cite{oreilly.Microservices, Sampaio.2017}. Hiermit einhergehend, erfuhr das im Jahre 2011 erschiene Buch \citetitle{Evans.2011} von \citeauthor{Evans.2011} an Beliebtheit, da viele, der bearbeiteten Themenschwerpunkte auf Microservices adaptiert werden können \cite[S. 130ff.]{Vernon.2015}\cite{Microservice.DDD.2017}. Domain-Driven Design stellt Entwicklern zum Lösen gängiger Problemstellungen der Softwareentwicklung unter anderem ein Vorgehensmodell für die Realisierung eines businessorientierten Datenmodells bereit, woraufhin diese eine Unterstützung bei der Umsetzung von Businessanforderungen erhalten. Eine Checkout-Software, welche in dieser Bachelorarbeit als Proof-of-Concept entwickelt wird, profitiert von den Vorteilen einer solchen Architektur. \comment{Letzte zwei Sätze auch Quellen?}

\comment{Mischung von Aggregate und Aggregation?}
\comment{Zu viele Absätze?}

\section{Problemstellung}

Ein elementarer Bestandteil der Funktionsweisen eines Onlineshops erfüllt der Warenkorb. In diesem können zum späteren Erwerb Waren abgelegt oder zusätzliche Dienstleistungen hinzugefügt werden. Im Verlauf des Kaufprozesses ist es zudem möglich, eine Versandart einzustellen, Kundendaten zu hinterlegen und gewünschte Zahlungsarten auszuwählen. Nach erfolgreicher Überprüfung von Validierungsrichtlinien findet die Kaufabwicklung statt, der sogenannte 'Checkout'. Um die zuvor genannten Anwendungsfälle zu verwirklichen, wird eine eigens dafür geschriebene Software benötigt. In dieser Bachelorarbeit wird diese Anwendung vereinfacht implementiert und als 'Checkout-Software' bezeichnet. Eine solche Applikation stellt das Rückgrat des Onlineshops dar. Sie erfährt stetige Änderungen, besitzt \note{im Vergleich zu anderen Softwareprojekten eine großen Umfang an Businesslogik}{Vielzahl von} und ihre Einbindung in das Frontend beeinflusst durch ihre Antwortzeit weitergehend auch das Kundenerlebnis. Dadurch liegt stets ein hoher Fokus auf die Erfüllung von Qualitätsmerkmalen, wie Stabilität, Testbarkeit und Wartbarkeit. Der Checkout-Prozess, welcher durch diese Software abgewickelt wird, muss für alle relevanten Länder und ihre individuellen gesetzlichen Voraussetzungen fiskalisch korrekt ausgeführt werden. Jederzeit können neue Businessanforderungen entstehen, wodurch weitere länderspezifische Richtlinien in den Zuständigkeitsbereich der Anwendung fallen und beispielsweise eine Anpassung des Datenmodells erfordern. Zudem erfordert das System zur Abwicklung ihrer Arbeitsprozesse Daten aus verschiedensten Unternehmensabteilungen wie Preise, Produkte, Lieferkosten und Bestandsinformationen. Die Kommunikation mit externen Komponenten erhöht weiter die Komplexität der Anwendung und systemübergreifende Anforderungen müssen berücksichtigt werden. Eine große Rolle spielt hierbei die Performance. Vor allem bei hoher Auslastung, beispielsweise während Kampagnen, muss das Gesamtsystem weiterhin zuverlässig alle Anfrage in akzeptabler Zeit abarbeiten können. Dementsprechend stellt die Implementierung einer solchen Checkout-Applikation für die Software Engineers eine Herausforderung dar. Sofern die verwendete Architektur im langjährigen Entwicklungsprozess an Übersichtlichkeit verliert, leidet zugleich auch die Wartbarkeit des Sourcecodes darunter. Als Folge können weitere Qualitätsmerkmale negativ betroffen sein und der Aufwand zur Umsetzung von Businessanforderungen steigt an \cite[S. 3f.]{Pigoski.2001}. Aus diesen Gründen hat der Checkout einen hohen Bedarf an Befolgung eines bestimmten Industriestandards hinsichtlich der Softwarearchitektur und des Datenmodells. Somit besteht ein Teilaspekt der Problemstellung in der Auswahl eines geeigneten Ansatzes zur Realisierung der Software. Auf Basis der nachfolgenden Kapitel wird die Verwendung von Hexagonaler Architektur inklusive Domain-Driven Design für den Proof-of-Concept argumentativ begründet. Die Projektdurchführung orientiert sich hierbei an den empfohlenen Entwicklungsprozess von Domain-Driven Design. Im Fokus der Bachelorarbeit stehen die möglichen Schnitte der Aggregates innerhalb des Datenmodells. Als Forschungsfrage bildet sich heraus, welche funktionalen und nicht-funktionalen Implikationen diese unterschiedliche Modelle auf die Software und ihre Qualitätseigenschaften besitzen. Zur Veranschaulichung und praktischen Umsetzung der zugehörigen Antwort wird eine konkrete Implementierung in Form eines Proof-of-Concepts umgesetzt.

\section{Das Unternehmen MediaMarktSaturn}

Dieses Projekt wurde in dankbarer Zusammenarbeit mit dem Unternehmen \emph{MediaMarktSaturn Retail Group}, kurz \emph{MediaMarktSaturn}, erarbeitet. Als größte Elektronik-Fachmarktkette Europas bietet MediaMarktSaturn in über 1023 Märkten und 13 Ländern den Kunden die Erwerbsmöglichkeit einer Vielzahl unterschiedlicher Artikel. Dabei wird ein großer Wert auf ein technologisch neuartiges Kundenerlebnis gelegt, um ein positives Einkaufserlebnis zu ermöglichen. Hierbei ist die Zugehörigkeiten der Märkte in die Marken \emph{Media Markt} und \emph{Saturn} unterteilt, welche anhand ihrer unterschiedlichen Verkaufsstrategien die Wünsche diverser Kundengruppen abdecken. Wegen des steigenden Anteileis vom Unternehmensumsatzes durch den Onlineshop, gewann dieser über die Jahre für Media Markt und Saturn zunehmend an Bedeutung. Als Folge dessen wurden die Unternehmensziele dementsprechend auf die Entwicklung von komplexer Software zur Unterstützung des Onlineshops neu ausgelegt. Die Erfüllung dieser und damit verbundenen Tätigkeiten übernimmt die Tochtergesellschaft \emph{MediaMarktSaturn Technology}. Dank den 705 internen Mitarbeitern am Standort Ingolstadt kann somit eine einwandfreie Benutzererfahrung der Kunden erzielt werden.

Die Durchführung und Implementierung des Projektes bzw. Proof-of-Concepts geschah in Kooperation mit dem Team \emph{Checkout \& Payment}. Die sechs zugehörigen Teammitglieder sind zuständig einen unternehmensweiten, universellen Checkout für alle Länder bereitzustellen, sowohl für den Onlineshop als auch im Markt und per Handyapp. Durch den Einsatz von \Gls{Scrum} wird eine konstante Verbesserung der Applikation und des Arbeitsprozesses erzielt. In kontinuierlichen \Glspl{Sprint} wird zusätzlich die Checkout-Software auf Basis von hinzukommenden Anwendungsfällen stetig erweitert. Dieses Projekt soll dem Team als Revision dienen und Aufschlüsse über mögliche architektonische Designansätze darbieten.

\comment{Quellen?}


\section{Motivation}

Durch den fortlaufenden Anstieg der Komplexität von Softwareprojekten \cite{Darcy.2010} haben sich gängige Designprinzipien und Architekturstile für den Entwicklungsprozess etabliert, sodass auch weiterhin die Businessanforderungen in einem zukunftssicheren Ansatz erfüllt und die Prozessabläufe jederzeit angepasst und erweitert werden können. Deshalb ist zur Gewährleistung der Langlebigkeit einer solchen Software eine flexible Grundstruktur entscheidend. Folglich ist eine sorgfältige Projektplanung und stetige Revision der Produktivanwendung relevant, um auch weiterhin einen reibungslosen Ablauf der Geschäftsprozesse zu ermöglichen. Zur Erreichung dieses Ziels verwendet die zum aktuellen Zeitpunkt bestehende Anwendung des Checkout-Teams eine Hexagonale Architektur und Domain-Driven Design.

Dieses Projekt hilft somit bei der Überprüfung der Architektur auf Verbesserungsmöglichkeiten und eventuelle Schwachstellen. Zudem existieren aufgrund des jetzigen zugrundeliegenden Aggregate-Designs gewisse Nachteile hinsichtlich der Performance und gleichzeitig stattfindenden Bearbeitung von Ressourcen. In diesem Projekt wird analysiert, ob die Performance durch eine andere Aufteilung des Datenmodells und einem damit verbundenen vertretbaren Aufwand gesteigert werden kann. Dies dient ebenfalls als Reverenz für zukünftige Softwareprojekte, denn viele Projekte sind mit ähnlichen Problemstellungen konfrontiert.



\section{Ziele}

Aus den vorhergehenden Motivationen lassen sich folgenden Projektziele ableiten. Grundlegend stellt diese Arbeit einen Anhaltspunkt für neue Softwareprojekte und Mitarbeiter dar. Dies kann zu einem erhöhten Grad an Softwarequalität im Unternehmen beitragen. Zugleich wird durch die Analyse und Durchführung des Proof-of-Concepts das bestehende Softwaredesign überprüft und herausgefordert. Dadurch können konkrete Verbesserungsvorschläge ein mögliches Fazit der Arbeit sein. Womöglich ergeben sich jedoch keine sinnvollen Änderungen der Produktivanwendung. Letzteres stellt dennoch eine wichtige Erkenntnis für das Team und Unternehmen dar, denn zukünftige Entwicklungstätigkeiten können mithilfe der verwendeten Vorgehensmodelle ähnliche Ergebnisse erzielen. Sollten sich durch einen anderen Schnitt des Datenmodells erhebliche Vorteile bilden, kann ein Resultat dieses Projektes dem Umbau der Software entsprechen.
