
\newacronym{POC}{POC}{Proof-of-Concept}
\newacronym{DDD}{DDD}{Domain-Driven Design}
\newacronym{SRP}{SRP}{Single-Responsibility-Prinzip}
\newacronym{OCP}{OCP}{Open-Closed-Prinzip}
\newacronym{LSP}{LSP}{Liskovsches Substitutionsprinzip}
\newacronym{ISP}{ISP}{Interface-Segregation-Prinzip}
\newacronym{DIP}{DIP}{Dependency-Inversion-Prinzip}
\newacronym{DTO}{DTO}{Data-Transfer-Object}
\newacronym{HTTP}{HTTP}{Hypertext Transfer Protocol}
\newacronym{REST}{REST}{Representational State Transfer}
\newacronym{CRUD}{CRUD}{Create Update Delete}
\newacronym{CQRS}{CQRS}{Command and Query Responsibility Segregation}


\newglossaryentry{Boilerplate}{name=Boilerplate,
	description={Ein Teil einer Software, welcher viele Zeilen an Code einnimmt, obwohl dadurch nur wenig bis gar keine Funktion bereitgestellt wird}
}
\newglossaryentry{Stakeholder}{name=Stakeholder,
	description={Eine Gruppe von Personen mit relevanten Interesse und Einfluss auf eine Sache bzw. Projekt}
}
\newglossaryentry{Lost Update}{name=Lost Update,
	description={Phänomen, welches bei zeitgleichen Operationen auf den gleichen Datensätzen auftreten kann. Die angepassten Datensätze einer Transaktion gehen verloren, da sie direkt von einer zweiten Transaktion überschrieben werden, welche jedoch als Ausgangspunkt noch auf den alten Stand durchgeführt worden ist}
}
\newglossaryentry{Information-Expert-Prinzip}{name=Information-Expert-Prinzip,
	description={Die Verantwortung eines Anliegens liegt bei der Komponente, welche die notwendigen Informationen zur Erfüllen besitzt}
}
\newglossaryentry{Invariante}{name=Invariante,
	description={Bedinung, welche auch nach Datenanpassungen jederzeit erfüllt sein muss}
}
\newglossaryentry{Kohasion}{name=Kohäsion,
	description={Grad der logischen inneren Zusammengehörigkeit einer Komponente. Komponente, welche nur eng beinaheliegende Aufgaben erfüllen, haben einen hohen Grad an Kohäsion}
}
\newglossaryentry{DI}{name={Dependency Injection},
	description={Eine erweiterte Form des Dependency-Inversion-Prinzip, welches Abhängigkeiten erst zur Laufzeit des Programmes hinzufügt}
}
\newglossaryentry{immutable}{name=immutable,
	description={Die Unveränderlichkeit von Werten bzw. Variablen innerhalb einer Klasse}
}
\newglossaryentry{Scrum}{name=Scrum,
	description={Eine agile Entwicklungsmethode, welches hohen Fokus auf kontinuierliche Verbesserung in einem geregelten Zyklus legt}
}
\newglossaryentry{Sprint}{name=Sprint,
	description={Ein wiederkehrender festgelegter Zeitraum, indem eine vorher definierter Umfang an Arbeitspakten abgearbeitet wird}
}

\newglossaryentry{Product Owner}{name=Product Owner,
	description={Eine Scrum-Rolle, welche die wirtschaftliche Ziele und Prioritäten der Aufgabenpakete bestimmt}
}

%\newglossaryentry{Enumeration}{name=Enumeration,
%	description={Eine Auflistung von konstanten, unveränderlichen Werten}
%}