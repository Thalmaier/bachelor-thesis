\documentclass[
  11pt,					% Schriftgröße
  DIV=13,				% Seitenlayout (Satzspiegel)
  parskip=half,			% Abstand zwischen Absätzen
  twoside,				% Doppelseitig
%  headsepline,
  headings,	
%  draft,				% Korrek\add{text}turfassung
  ]{scrreprt}			% scrartcl	

% Kotlin highlighting
\usepackage[dvipsnames]{xcolor}
\usepackage{listings}
\lstdefinelanguage{Kotlin}{
	comment=[l]{//},
	commentstyle={\color{darkgray}\ttfamily},
	emph={filter, first, firstOrNull, forEach, lazy, map, mapNotNull, println},
	emphstyle={\color{OrangeRed}},
	identifierstyle=\color{black},
	keywords={!in, !is, abstract, actual, annotation, as, as?, break, by, catch, class, companion, const, constructor, continue, crossinline, data, delegate, do, dynamic, else, enum, expect, external, false, field, file, final, finally, for, fun, get, if, import, in, infix, init, inline, inner, interface, internal, is, lateinit, noinline, null, object, open, operator, out, override, package, param, private, property, protected, public, receiveris, reified, return, return@, sealed, set, setparam, super, suspend, tailrec, this, throw, true, try, typealias, typeof, val, var, vararg, when, where, while, returns, variable, function},
	keywordstyle={\color{NavyBlue}\bfseries},
	morecomment=[s]{/*}{*/},
	morestring=[b]",
	morestring=[s]{"""*}{*"""},
	ndkeywords={@Deprecated, @JvmField, @JvmName, @JvmOverloads, @JvmStatic, @JvmSynthetic, Array, Byte, Double, Float, Int, Integer, Iterable, Long, Runnable, Short, String, Any, Unit, Nothing},
	ndkeywordstyle={\color{BurntOrange}\bfseries},
	sensitive=true,
	stringstyle={\color{ForestGreen}\ttfamily}
}

\lstset{
	basicstyle=\scriptsize\sffamily\color{black},
	frame=single,
	numbers=left,
	numbersep=5pt,
	numberstyle=\tiny\color{gray},
	showspaces=false,
	showstringspaces=false,
	tabsize=4,
	linewidth=\textwidth,
	captionpos=b,
	extendedchars=true
	inputencoding=utf8,
	literate={ö}{{\"o}}1
	{ä}{{\"a}}1
	{ü}{{\"u}}1
	{Ö}{{\"O}}1
	{Ä}{{\"A}}1
	{Ü}{{\"U}}1
}

\usepackage[utf8]{inputenc}
\usepackage[ngerman,english]{babel}
\selectlanguage{ngerman}

\usepackage{graphicx}
\graphicspath{{bilder/}}
\usepackage{svg}
\usepackage{float} % Force placement of figures with [H]

\usepackage{enumitem}

% Blindtext
\usepackage{blindtext} %TODO: Remove later
\let\oldblindtext\blindtext
\renewcommand{\blindtext}{{\color{red}\oldblindtext}}
 
% Schrifteinstellungen
\usepackage{lmodern} 		% Vektorschrift
\renewcommand{\familydefault}{\sfdefault}
\usepackage{sansmath}
\sansmath 
\usepackage{microtype}

% Literatur einbinden
\usepackage{csquotes}	% Steuerung der Anführungszeichen
\usepackage[
  backend=bibtex,		% Sortier-Compiler
  style=numeric-comp,	% Zitationsstil
  ]{biblatex}

% Mathemodus
\usepackage{amsmath,amssymb}

% Trennung
\hyphenation{Crash-zo-ne}

\addbibresource{ref/ref.bib} %Citavi Reference Datei

% Kopf- und Fußzeile
\usepackage[
headsepline,		% Kopfzeilen-Sepparationslinie
automark,		% Lebende Kolumnentitel
]
{scrlayer-scrpage}
\pagestyle{scrheadings}		
\ofoot*{\pagemark}			
\ohead*{\headmark}

\usepackage{abstract}
\usepackage{adjustbox}


\usepackage[acronym]{glossaries}

\makeglossaries

%%%%%%%%%%%%%%%%%%%%%%%%%%%%%%%%%%%%%%%%%%%%%%%%%
%%%%%%%%%%%%%%%%%%%%%%%%%%%%%%%%%%%%%%%%%%%%%%%%%
%%%%%%%%%%%%%%%%%%%%%%%%%%%%%%%%%%%%%%%%%%%%%%%%%
%%%%%%%%%%%%%%%%%%%%%%%%%%%%%%%%%%%%%%%%%%%%%%%%%
%%%%%%%%%%%%%%%%%%%%%%%%%%%%%%%%%%%%%%%%%%%%%%%%%
%%%%%%%%%%%%%%%%%%%%%%%%%%%%%%%%%%%%%%%%%%%%%%%%%
%%%%%%%%%%%%%%%%%%%%%%%%%%%%%%%%%%%%%%%%%%%%%%%%%
%%%%%%%%%%%%%%%%%%%%%%%%%%%%%%%%%%%%%%%%%%%%%%%%%



% Titelseite
\titlehead{
  \begin{center}
  	\includegraphics[width=0.5\textwidth]{thi_logo_cropped}
  \end{center}
}

\title{Aggregationsschnitt einer Checkout-Software auf Basis einer Hexagonalen Architektur mit Domain-Driven Design }

\subtitle{ \vspace{2ex} \LARGE Bachelor-Arbeit}

\author{Simon Thalmaier}

\date{}

\publishers{
  \begin{tabular}{rl}
   \textbf{Erstprüfer} 		& - 				\\
   \textbf{Zweitprüfer} 	& - 				\\
   \textbf{Betreuer} 		& Sebastian Apel	\\
   \textbf{Ausgabedatum} 	& - 				\\
   \textbf{Abgabedatum} 	& -					\\
  \end{tabular}
  }
  
% Rückseite der Titelseite
\uppertitleback{Angaben zum Autor oder Vergleichbares.}
\lowertitleback{Dokumenteninformation, Veröffentlichung, Rahmen, bibliographisches Angaben}



\newcommand\frontmatter{%
	\cleardoublepage
	%\@mainmatterfalse
	\pagenumbering{Roman}}

\newcommand\mainmatter{%
	\cleardoublepage
	% \@mainmattertrue
	\pagenumbering{arabic}}

\newcommand\backmatter{%
	\cleardoublepage
	%\@mainmatterfalse
	\pagenumbering{roman}}

\newcommand{\comment}[1]{{\color{Green}//Kommentar: #1}\\} %TODO: Delete later
\newcommand{\note}[2]{{\color{red}#1}}
\newcommand{\domainModell}[2]{
	{\large \textbf{#1:}}
	\begin{itemize}[noitemsep,nolistsep,topsep=-5pt]#2\end{itemize}
	\vspace{12pt}
}
\newcommand{\domainModellWithoutTitle}[1]{
	\begin{itemize}[noitemsep,nolistsep,topsep=-5pt]#1\end{itemize}
	\vspace{12pt}
}
\newcommand{\subDomainModell}[1]{
	\begin{itemize}[noitemsep,topsep=-5pt]#1\end{itemize}
	\vspace{8pt}
}

\usepackage{lscape}
\newcommand{\centertitle}[1]{\vspace{0.8mm}#1\vspace{0.5mm}}
\newcommand{\centertable}[1]{\vspace{0.1cm}#1\vspace{0.1cm}}

\usepackage{array}

\def\table{\def\figurename{Tabelle}\figure}
\let\endtable\endfigure 

\newcommand{\ul}[1]{\emph{#1}}

\begin{document}
  
  
\newacronym{POC}{POC}{Proof-of-Concept}
\newacronym{DDD}{DDD}{Domain-Driven Design}
\newacronym{SRP}{SRP}{Single-Responsibility-Prinzip}
\newacronym{OCP}{OCP}{Open-Closed-Prinzip}
\newacronym{LSP}{LSP}{Liskovsches Substitutionsprinzip}
\newacronym{ISP}{ISP}{Interface-Segregation-Prinzip}
\newacronym{DIP}{DIP}{Dependency-Inversion-Prinzip}


\newacronym[description={Data-Transfer-Object. Ein Objekt zum Transport von Daten innerhalb der Applikation ohne jegliche implementierte Logik. \cite[S. 401ff.]{Fowler.2011}}]{DTO}{DTO}{Data-Transfer-Object}
\newacronym{HTTP}{HTTP}{Hypertext Transfer Protocol}
\newacronym[description={Representational State Transfer. Die Transition von Zuständen der Clients wird durch Abrufen einer Ressource des Servers erreicht. \cite{Fielding.2000} }]{REST}{REST}{Representational State Transfer}
\newacronym{CRUD}{CRUD}{Create Read Update Delete}
\newacronym[description={Command and Query Responsibility Segregation. Trennung des Datenmodells in Befehle für Schreiboperationen und Abfragen für Leseoperationen zur Erreichung einer Aufteilung der Zuständigkeiten und erhöhter Performance. \cite[S. 223ff.]{CQRS_2013}}]{CQRS}{CQRS}{Command and Query Responsibility Segregation}
\newacronym{KPI}{KPI}{Key-Performance-Indicator}

\newglossaryentry{Serialisierung}{name={Serialisierung},
	description={Konvertierung von Datenobjekte in ein sequenzielles Format}
}
\newglossaryentry{Deserialisierung}{name={Deserialisierung}, 
	description={Die Umkehrung einer Serialisierung. Wandelt ein sequenzielles Datenformat in Objekte einer Klasse um}
}
\newglossaryentry{User Story}{name={User Story},
	description={Ein möglicher Ablauf des kompletten Businessprozesses aus Sicht des Kunden}
}
\newglossaryentry{Race Condition}{name={Race Condition},
	description={Zwei gleichzeitg bzw. nahezu gleichzeitig stattfindende Prozesse bedingen sich gegenseitig und führen zu nicht vorgesehenen Verhalten \cite{racecondition}}}
\newglossaryentry{Boilerplate-Code}{name=Boilerplate-Code,
	description={Ein Codeabschnitt, welcher viele Zeilen im Quelltext einnimmt bzw. wiederholt in diesem vorkommt, obwohl hierdurch nur wenige bis gar keine Funktionen bereitgestellt werden \cite{Lammel.2003, Zaveri.2018}}
}
\newglossaryentry{Stakeholder}{name=Stakeholder,
	description={Ein Zusammenschluss von Personen außerhalb des Teams mit relevanten Interesse und/oder Einfluss auf das Projekt \cite{scrum.glossary}}
}
\newglossaryentry{Lost Update}{name=Lost Update,
	description={Phänomen, welches bei zeitgleichen Operationen auf den gleichen Datensätzen innerhalb einer Datenbank auftreten kann. Die angepassten Datensätze einer Transaktion gehen verloren, da sie direkt von einer zweiten Transaktion überschrieben werden. Die zweite Transaktion wurde jedoch noch auf dem alten Datenstand durchgeführt \cite{lostupdate}}
}
\newglossaryentry{Information-Expert-Prinzip}{name=Information-Expert-Prinzip,
	description={Die Verantwortung einer Funktionalität soll bei der Komponente liegen, welche die notwendigen Informationen zur erfolgreichen Abwicklung besitzt \cite[S. 218]{Larman.2009}}
}
\newglossaryentry{Invariante}{name=Invariante,
	description={Businessbedingung, welche jederzeit erfüllt sein muss \cite[S. 353]{Vernon.2015}}
}
\newglossaryentry{Kohasion}{name=Kohäsion,
	description={Grad der logischen inneren Zusammengehörigkeit einer Komponente. Komponente, welche nur eng beinaheliegende Aufgaben erfüllen, haben einen hohen Grad an Kohäsion \cite[S. 95]{Yourdon.1979}}
}
\newglossaryentry{DI}{name={Dependency Injection},
	description={Ein spezifischeres Inversion-of-Control-Prinzip, welches Implementierungen zu passenden Abstraktionen erst zur Laufzeit des Programmes lädt \cite{Fowler.2004}}
}
\newglossaryentry{immutable}{name=Immutable, text=immutable,
	description={Die Unveränderlichkeit von Werten bzw. Variablen}
}
\newglossaryentry{Scrum}{name=Scrum,
	description={Ein agiles Vorgehensmodell, welches hohen Fokus auf kontinuierliche Verbesserung in einem geregelten Zyklus legt}
}
\newglossaryentry{Sprint}{name=Sprint,
	description={Ein wiederkehrender festgelegter Zeitraum in Scrum, indem ein vorher definierter Umfang an Arbeitspakten abgearbeitet wird \cite{scrum.sprint}}
}

\newglossaryentry{Product Owner}{name=Product Owner,
	description={Eine Scrum-Rolle, welche die Verantwortung über die Arbeitsergebnisse des Teams besitzt und hierbei ihre Produktivität maximiert \cite{po.scrum}}
}
\newglossaryentry{Lazy Loading}{name={Lazy Loading},
	description={Das Laden von Daten aus einem Datenspeicher oder sonstigen Quellen wird erst durchgeführt, sobald auf diese zugegriffen werden, wodurch unnötiges Zuvorladen minimiert wird \cite[S. 200]{Fowler.2011}}}
\newglossaryentry{Collection}{name={Collection},
	description={MonogDB persistiert Datensätze in Collections, welche gleichbedeutend sind mit Tabellen einer relationalen Datenbank \cite{mongodb_collections}}}
\newglossaryentry{Connection-Pool}{name={Connection-Pool},
	description={Eine Gruppe von Verbindungen zu Datenbanken oder APIs zur Performance-Optimierung und Isolierung, indem zuvor erzeugte Verbindungen wiederverwendet werden \cite{Sohel.2017}}
}

\newglossaryentry{Anemic Domain Model}{name={Anemic Domain Model},
	description={Ein Anti-Pattern in welchem die Domain-Objekte keine bzw. kaum Businesslogik implementieren \cite{Fowler.AnemicDomainModel}}
}
\newglossaryentry{technische Schulden}{name={technische Schulden},
	description={Bewusst akzeptierte Vernachlässigung von qualitätsschadenden Eigenschaften einer Software \cite{technical.dept}}
}

  
  \maketitle
  
  \frontmatter
  
  
%% ABSTRACT %%


{
	
	\raggedbottom
	\centering
	
	\selectlanguage{ngerman}
	
	\large
	{
		\thispagestyle{plain}
		\vspace*{\fill}
		\adjustbox{minipage=0.88\textwidth}{
			\begin{center}
				\LARGE
				\textbf{Erklärung zur Bachelorarbeit}
			\end{center}
			\vspace{1em}  
			
			Ich erkläre hiermit, dass ich diese Bachelorarbeit selbständig verfasst,
			noch nicht anderweitig für Prüfungszwecke vorgelegt, keine anderen als
			die angegebenen Quellen und Hilfsmittel benutzt sowie wörtliche und
			sinngemäße Zitate als solche gekennzeichnet habe. 
			
			\vspace{4em}
			
			%\begin{tikzpicture}[remember picture,overlay]
			%	\node[xshift=95mm,yshift=-144mm, anchor=north west] at (current page.north west){%
			%		\includegraphics[width=0.5\textwidth]{Unterschrift.png}};
			%\end{tikzpicture}
			
			Ingolstadt, \today
			\hfill
			\rule{220pt}{1pt} \\
			
			\hfill
			Simon Thalmaier 
			
			\par}
		\vfill
	}
	
	
	
	\pagebreak
	
	
	
	\selectlanguage{english}
	\large
	
	{
		\thispagestyle{plain}
		\vspace*{\fill}
		\adjustbox{minipage=0.88\textwidth}{
			\begin{center}
				\LARGE
				\textbf{Abstract}
			\end{center}
			\vspace{1em}  
			
			In the e-commerce sector software projects fulfill complex business requirements and therefore demand a stable architecture as their foundation. This thesis examines an online shop checkout domain and how the bounded context can be implemented utilizing hexagonal architecture and domain-driven design. The focus is placed on the design of the aggregates. Various data models are analyzed and evaluated based on their complexity, performance, parallelism and client-friendliness. Generally, big aggregates suffer from lower parallelism, however can be implemented more easily since all required information is loaded at once from the database. Splitting the data model into different aggregates entails the usage of eventual consistency or force transactions to modify more than one aggregate at the same time. Eventual consistency increases the complexity of the checkout software, on the other hand cross-aggregate transactions make the operation of multiple database hosts within one application more difficult. The performance of individual aggregate designs are measured with a load test. On average, applications with one cohesive aggregate process more requests per second than a model using separated aggregates. Conclusively, the checkout software profits from a higher performance and reduced complexity by implementing only one aggregate. If a common use case is the mutation of one resource by distinct processes simultaneously then the affected objects need to be placed in different aggregates.
			
			\par}
		\vfill
	}
	
	
 	\pagebreak

	\selectlanguage{ngerman}
	\large
	
	{
		\thispagestyle{plain}
		\vspace*{\fill}
		\adjustbox{minipage=0.88\textwidth}{
			\begin{center}
				\LARGE
				\textbf{Zusammenfassung}
			\end{center}
			
			\vspace{1em}  
			
			Softwareprojekte im E-Commerce-Bereich erfüllen komplexe Businessanforderungen und benötigen aufgrund dessen eine stabile Architektur. Dieses Projekt untersucht die Domain eines Onlineshop-Checkouts und wie der Bounded-Context mithilfe einer Hexagonalen Architektur und Domain-Driven Design implementiert werden kann. Besonders liegt der Aggregationsschnitt im Fokus, wobei unterschiedliche Datenmodelle analysiert und anhand von Komplexität, Performance, Parallelität und Client-Freundlichkeit bewertet werden. Große Aggregates leiden generell unter verringerter Parallelität, jedoch bietet ein zusammengehöriges Datenmodell eine vereinfachte Umsetzung von Businessanforderungen, da stets alle Informationen aus der Datenbank geladen werden. Die Aufteilung in mehrere Aggregates erzwingt die Anwendung von Eventueller Konsistenz oder einer Transaktion über mehrere Aggregates. Eventuelle Konsistenz erhöht die Komplexität der Checkout-Software, wohingegen eine aggregate-übergreifende Transaktion die Verwendung von unterschiedlichen Datenbank-Hosts erschwert. Anhand eines Lasttests wird die Performance der Designansätze betrachtet. Im Durchschnitt verarbeitet die Applikation mit einem zusammengehörigen Datenmodell mehr Anfragen pro Sekunde als unter Verwendung getrennter Aggregates. Als Fazit dieser Arbeit wird argumentativ begründet, dass innerhalb einer Checkout-Software die Vorteile eines Designs mit einem einzigen Aggregate dank der erhöhten Performance und reduzierten Komplexität überwiegen. Falls eine zeitgleiche Bearbeitung einer Ressource durch unterschiedliche Prozesse ein gängiger Anwendungsfall ist, müssen die betroffenen Objekte in separate Aggregates verlagert werden.
			
			\par}
		\vfill
	}
	
	
	
	\pagebreak

	\selectlanguage{ngerman}
	\large
	
	{
		\thispagestyle{plain}
		\vspace*{\fill}
		\adjustbox{minipage=0.88\textwidth}{
			\begin{center}
				\LARGE
				\textbf{Danksagung}
			\end{center}
			\vspace{1em} 
			
			
			Mein Dank gilt der Firma MediaMarktSaturn Technology, welche mir ermöglicht hat als Werkstudent über die vergangenen drei Jahre zu arbeiten und das Thema für diese Bachelorarbeit bereitgestellt hat. Besonders bedanke ich mich bei Sebastian Jurjanz für die Unterstützung in meiner Ausbildung und während dieses Projektes. Zusätzlich ist die Bearbeitung des Forschungsthemas in diesem Umfang durch Stefano Lucka dank seiner Betreuung und seinem Feedback ermöglicht worden. \\
			
			Aufseiten der Technischen Hochschule Ingolstadt bedanke ich mich bei \text{Professor Dr. Sebastian Apel} für das konstruktive Feedback und die umfangreiche Beratung. 
			
			
			\par}
		\vfill
	}
	
		
	\pagebreak
	
\par}
  
  \selectlanguage{ngerman}
  
  \tableofcontents
  
  %Abbildungsverzeichnis
  \newpage
  \addcontentsline{toc}{chapter}{Darstellungsverzeichnis}
  \renewcommand{\listfigurename}{Darstellungsverzeichnis}
  \listoffigures
  
  %Abbildungsverzeichnis
  \newpage
  \renewcommand{\lstlistingname}{Codebeispiel}
  \renewcommand*{\lstlistlistingname}{Codebeispiel-Verzeichnis}
  \addcontentsline{toc}{chapter}{Codebeispiel-Verzeichnis}
  \lstlistoflistings
  
  %Acronyms
  \newpage
  \setlength{\glsdescwidth}{0.8\textwidth}
  \addcontentsline{toc}{chapter}{Akronyme}
  \printglossary[type=\acronymtype, style=super, nonumberlist]
  
  %Glossary
  \newpage
  \setlength{\glsdescwidth}{0.75\textwidth}
  \addcontentsline{toc}{chapter}{Glossar}
  \printglossary[nonumberlist, style=super]
  
  \mainmatter
  
  
%% EINLEITUNG %% 

\chapter{Einleitung}

Anfangs wird die vorliegende Problemstellung erläutert, sowie das Projektumfeld und die dahinterliegende Motivation und ihre Ziele. Hierdurch soll ein grundlegendes Verständnis der Hintergründe dieser Arbeit geschaffen werden.

\section{Problemstellung}
\begin{itemize}[noitemsep,nolistsep]
	\item Alot of Business Rules -> future proof architecture
	\item Modelling the Domain defines how the software and the external systems interact with the software
\end{itemize}

% - Erklärung eines Checkouts / Warenkorbs
% - Use-Case erläutern
% - Projektumfeld hinleitung

Ein elementarer Bestandteil eines Onlineshops ist der sogenannte Warenkorb. In diesem können unter anderem Waren, welche eventuell zu einem späteren Zeitpunkt gekauft werden wollen, abgelegt werden. Die Kernfunktionen eines Warenkorbs umfasst das Hinzufügen bzw. Löschen von Waren und das Abändern ihrer Stückzahl. Weiterhin soll es möglich sein eine Versandart einzustellen, Kundendaten zu hinterlegen und eine Zahlungsart auszuwählen. Nach erfolgreicher Überprüfung von Sicherheitskriterien soll abschließend die Kaufabwicklung, der sogenannte 'Checkout', möglich sein. Um die hier beschriebenen Anwendungsfälle zu verwirklichen wird eine dafür designierte Software benötigt. In diesem Projekt wird diese als 'Checkout-Software' bezeichnet.
 
Zusätzliche Anforderungen an der Software können die Modellierung der Daten beeinflussen, daher muss vor der eigentlichen Implementierung eine sorgfältige Use-Case-Analyse durchgeführt werden. Diese wird in einem späteren Kapitel erläutert.

Ein weiterer Teil der Problemdomäne sind die bereits existierenden Systeme, welche vor- bzw. nach dem Checkout-Prozess liegen. Um eine nahtlose Einbindung der Software zu gewährleisten muss eine Kommunikation mit den zuständigen Entwicklerteams, sowie eine Umfeldanalyse stattfinden. Die erarbeiteten Ergebnisse werden im folgenden Kapitel dargestellt.



\section{Projektumfeld}
\begin{itemize}[noitemsep,nolistsep]
	\item MediaMarktSaturn
	\item Currently Checkout-Software exist
	\item Vor bzw Nachgelagerte Systeme
\end{itemize}

\subsection{Das Unternehmen MediaMarktSaturn}

% Komplett überarbeiten
% Eigenmarken?

Diese Bachelorarbeit wird in Zusammenarbeit mit dem Unternehmen der \emph{MediaMarktSaturn Retail Group}, kurz \emph{MediaMarktSaturn}, erarbeitet. %Wurde?
Als größte Elektronik-Fachmarktkette Europas bietet MediaMarktSaturn Kunden in über 1023 Märkten eine Einkaufmöglichkeit einer Vielzahl von Waren. Die Marktzugehörigkeit ist hierbei unterteilt in den Marken \emph{Media Markt} und \emph{Saturn}. % Ist die Software nur für MM?
Über die Jahre gewann der Onlineshop für Media Markt und Saturn an zunehmender Bedeutung, da die prozentuale Verteilung des jährlichen Gewinns in den Märkten zurückgegangen und in den Onlineshops gestiegen ist. Dadurch wurden die Unternehmensziele dementsprechend auf die Entwicklung von Software zur Unterstützung des Onlineshops neu ausgelegt. Die, der MediaMarktSaturn Retail Group unterteilte, Firma \emph{MediaMarktSaturn Technology} ist hierbei verantwortlich für alle Entwicklungstätigkeiten. Dieses Projekt wurde im Team \emph{Chechkout \& Payment} erarbeitet, welches zuständig ist für die Checkout-Software.

\subsection{Benachbarte Systeme der Checkout-Software}

\section{Motivation} 
\begin{itemize}[noitemsep,nolistsep]
	\item MediaMarktSaturn. Warum braucht MMS eine Checkout-Solution bzw dieses Projekt?
	\item Performance
	\item Interaction with the system (?????????)
	\item Reverenz für zukünftige Projekte
\end{itemize}

% Aktuell existiert die Software schon. Ist die Frage welcher Architekturstil verwendet werden soll, relevant für die Arbeit oder bereits vorgegeben??

Durch den stetigen Anstieg an Komplexität von Softwareprojekten haben sich gängige Software Designprinzipien und Architekturstile etabliert, um die erhöhte Anzahl an Businessanforderungen in einem zukunftssicheren Ansatz zu realisieren. Eine Checkout-Software beinhaltet multiple Prozessregeln, welche jederzeit angepasst und erweitert werden können. Dadurch ist eine flexible Grundstruktur entscheidend im die Langlebigkeit der Software zu gewährleisten. Eine Checkout-Software ist ein wichtiger Bestandteil eines Onlineshops und dadurch für MediaMarktSaturn von zentraler Bedeutung. Folglich ist eine sorgfältige Projektplanung und stetige Revision der bestehenden Software relevant, um auch weiterhin einen reibungslosen Ablauf der Geschäftsprozesse zu ermöglichen. Die zum aktuellen Zeitpunkt bestehende Anwendung verwendet eine Hexagonale Architektur und Domain-Driven Design, um dieses Ziel zu erreichen. Der erste Abschnitt dieser Arbeit beschäftigt sich mit der Entscheidung, ob auch weiterhin ein solcher Aufbau verfolgt werden sollte, um die aktuelle Lösung nach Verbesserungsmöglichkeiten zu überprüfen. 

Zudem existieren aufgrund der zugrundeliegenden Architektur Performance-Einbusen. In diesem Projekt wird analysiert, ob die Performance durch einen anderen Aggregationsschnitt und einem vertretbaren Aufwand gesteigert werden kann. Dies dient ebenfalls als nützliche Untersuchung der bestehenden Anwendung und kann als Reverenz für zukünftige Softwareprojekte verwendet werden, da viele weitere Projekte mit ähnlichen Problemstellungen konfrontiert sind.



\section{Ziele}
\begin{itemize}[noitemsep,nolistsep]
	\item Eventueller Umbau
	\item Überprüfen der aktuellen Architektur
	\item Bewertung für zukünftigere Softwareprojekte
\end{itemize}

% Gleiche wie Motivation

  
  
%% GRUNDLAGEN %% 

\chapter{Grundlagen}

Für das Verständnis des Bachelorthemas werden Kernkompetenzen der Softwareentwicklung vorausgesetzt. Dies betrifft vornehmlich Softwaredesign und Architekturstile. Um nachzuvollziehen, wie eine Architektur die Programmierer bei der Entwicklungsphase unterstützt, muss zunächst festgelegt werden, welche Eigenschaften der Quellcode erfüllen soll, damit dieser positive Qualitätsmerkmale widerspiegelt. Hierzu wurden etablierte Designprinzipien über die Jahre festgelegt. Unter anderem die sogenannten 'SOLID'-Prinzipien, die dazu beitragen Architekturansätze miteinander zu vergleichen und zu bewerten.

\section{SOLID-Prinzipien}

Die SOLID-Prinzipen sollen sicherstellen, dass Software auch mit zunehmendem Funktionsumfang weiterhin testbar, anpassbar und fehlerfrei bleibt \cite{Martin.2000, Martin.2018}. Das weitverbreitete Akronym 'SOLID' steht hierbei für die fünf Designprinzipien:

\textbf{\acrfull{SRP}: } {Jede Softwarekomponente darf laut SRP maximal eine zugehörige Aufgabe erfüllen. Eine Änderung in den Anforderungen erfordert somit die Anpassung in genau einer einzelnen Komponente. Dies erhöht stark die \emph{\Gls{Kohasion}} der Komponente und senkt die Wahrscheinlichkeit von unerwünschten Nebeneffekten bei Codeanpassungen. \cite{Martin.SRP, Martin.2018}}

\textbf{\acrfull{OCP}: } Um sicherzustellen, dass eine Änderung in einer Komponente keine Auswirkungen auf eine andere hat, werden diese als 'geschlossen' gegenüber Veränderungen aber weiterhin 'offen' für Erweiterungen definiert. Der erste Teil des Prinzips kann durch eine Schnittstelle, ein sogenanntes Interface, realisiert werden. Es gilt als geschlossen, da die Implementierungen keine Signaturänderungen der im Interface definierten Methoden vernehmen können. Ansonsten müsste der darauf basierende Code ebenfalls bearbeitet werden. Dennoch können weiterhin Modifikationen durch das Vererben von Klassen oder die Einbindung von neuen Interfaces stattfinden. Dies wird als 'offen' im Sinne des OCPs betrachtet. \cite{Martin.2018, Meyer.2009}

\textbf{\acrfull{LSP}: } {Eine wünschenswerte Eigenschaft der Vererbung ist, dass eine Unterklasse S einer Oberklasse T die Korrektheit einer Anwendung nicht beeinflusst, wenn ein Objekt vom Typ T durch ein Objekt vom Typ S ersetzt wird. Dadurch wird die Fehleranfälligkeit bei einer Substitution im Code erheblich reduziert und der Client kann sichergehen, dass die Funktionalität auch weiterhin den erwarteten Effekt hat. Da sich das LSP mit der Komposition von Klassen beschäftigt, ist es für die nachfolgende Architekturanalyse vernachlässigbar. \cite{Martin.2018, Liskov.1994}}

\textbf{\acrfull{ISP}: } {Der Schnitt von Interfaces sollte so spezifisch und klein wie möglich gehalten werden, damit Clients nur Abhängigkeiten zu Funktionalitäten besitzen, welche sie wirklich benötigen. Dadurch wird die Wiederverwendbarkeit und Austauschbarkeit der Komponenten gewährleistet. \cite{Martin.2018}\cite[S. 135ff.]{Martin.2003}}

\textbf{\acrfull{DIP}: } {Module sollten so unabhängig wie möglich voneinander genutzt werden können. Dadurch wird eine erhöhte Testbarkeit und Wiederverwendbarkeit ermöglicht. Das zweiteilige DIP ist von zentraler Bedeutung für eine stabile und flexible Software. Hierbei sollen konzeptionell höhere Komponenten nicht direkt auf darunterliegende Ebenen angewiesen sein, sondern die Kommunikation zwischen ihnen über Interfaces geschehen. Dies erlaubt die Abstraktion von Funktionsweisen und löst die direkten Abhängigkeiten zwischen Modulen auf. Weiterhin wird festgelegt, dass Interfaces nicht an ihre Implementierungen gekoppelt werden sollten, sondern auf deren Abstraktionen beruhen \cite{Martin.1996, Martin.2018}. Dadurch sind die Abhängigkeiten invertiert, was beispielhaft die Anwendung von \emph{\Gls{DI}} ermöglicht \cite{Fowler.2004}.}


\section{Architekturmuster}

Eine Softwarearchitektur beschreibt die grundlegende Struktur der Module, ihre Relationen zueinander und die Art der Kommunikation zwischen den Modulen. Die Wahl der verwendeten Architektur beeinflusst somit die komplette Applikation und ihre Qualitätsmerkmale. Das zu bevorzugende Design einer Anwendung ist gekoppelt an die Anwendungsfälle und ihre Anforderungen. 

In diesem Projekt soll ein Backend-Service erstellt werden, welcher mit den vorgelagerten Systemen über \emph{\acrshort{HTTP}} und \emph{\acrshort{REST}} kommuniziert, wodurch die Auswahl der Architekturen beschränkt wird. Ansätze wie Peer-to-Peer, welche eine Kommunikation zwischen zwei gleichberechtigten Knoten bereitstellen, sind somit in diesem Anwendungsgebiet nur bedingt vertreten. Etablierte Architekturen für Backend-Software, welche die Businessprozesse als Kern der Applikation halten, werden hingegen genauer untersucht. Die Schichtenarchitektur und Hexagonale Architektur werden als Grundlage für das Projekt herangezogen. Im folgenden Abschnitt werden beide Stile untersucht und anhand ihrer Tauglichkeit für eine Checkout-Software bewertet. Dabei wird hinterfragt, in wie fern Entwickler bei der Umsetzung der SOLID-Prinzipien unterstützt werden.

\subsection{Schichtenarchitektur}

In einer Schichtenarchitektur werden die Softwarekomponenten in einzelne Schichten eingeteilt. Die Anzahl der Schichten kann je nach Anwendungsfall variieren, liegt jedoch meist zwischen drei und vier Ebenen. Eine verbreitete Variante beinhaltet die Präsentations-, Business- und Datenzugriffsschicht. Dadurch wird eine Trennung der Verantwortlichkeiten erzwungen \cite[S. 185]{Buschmann.2011}. Der Kontrollfluss der Applikation fließt hierbei stets von einer höheren Schicht in eine tiefer gelegene oder innerhalb einer Ebene zwischen einzelnen Komponenten. Ohne eine konkrete Umkehrung der Abhängigkeiten ist der Abhängigkeitsgraph gleichgerichtet zum Kontrollflussgraph. \cite[S. 17ff.]{Fowler.2011} Der beschriebene Aufbau einer solchen Architektur ist in Abbildung \ref{fig:Schichtenarchitektur} beispielhaft dargestellt. 

\begin{figure}[H]
	\centering
	\large
	\includesvg[width=0.47\textwidth]{svg/Schichtenarchitektur.svg}
	\caption{Beispielhafte Darstellung einer Drei-Schichtenarchitektur}
	\label{fig:Schichtenarchitektur}
\end{figure}

\pagebreak

Wesentliche Ziele einer Schichtenarchitektur sind die Entkopplung der einzelnen Schichten voneinander und das Erreichen von geringen Abhängigkeiten zwischen den Komponenten \cite[S. 17]{Fowler.2011}. Dadurch sollen Qualitätseigenschaften wie Testbarkeit, Erweiterbarkeit und Flexibilität erhöht werden. Dank des simplen Aufbaus gewann dieser Architekturstil an großer Beliebtheit. Weitere bewertende Aspekte einer solchen Softwarestruktur ergeben sich aus der Analyse der SOLID-Prinzipien:

\textbf{\acrlong{SRP}:} Durch die Schichteneinteilung wird die natürliche Einhaltung des \acrshort{SRP}s unterstützt, da eine Komponente zum Beispiel keine Businesslogik und zugleich Funktionen der Datenzugriffsschicht implementieren kann. Nichtsdestotrotz ist eine vertikale Trennung innerhalb einer Schicht nicht gegeben, daher können weiterhin Klassen mehrere, konzeptionell verschiedene Aufgaben entgegen des SRPs erfüllen. 

\textbf{\acrlong{OCP} \& \acrlong{ISP}:} Um die einzelnen Schichten zu entkoppeln, kann die Kommunikation zwischen den Ebenen durch Schnittstellen geschehen. Dadurch wird eine grundlegende Befolgung des \acrshort{ISP} erreicht. Das \acrlong{OCP} soll hierbei helfen, dass Änderungen an den Schnittstellen und ihren Implementierungen die Funktionsweise nicht beeinflussen, auf denen tieferliegende Schichten basieren. Die logische Zuteilung dieser Interfaces ist entscheidend, um eine korrekte Anwendung des \acrlong{DIP}s zu gewährleisten. 

\textbf{\acrlong{DIP}:} Meist wird bei webbasierten CRUD-Applikationen eine Schichtenarchitektur verwendet. \acrshort{CRUD} steht im Softwarekontext für '\textbf{C}reate \textbf{R}ead \textbf{U}pdate \textbf{D}elete' und meint Anwendungen, die Daten mit geringer bis keiner Geschäftslogik erzeugen, bearbeiten und löschen \cite[S. 381]{Martin.1980}. Im Kern einer solchen Software liegen die Daten selbst. Dabei werden Module und die umliegende Architektur angepasst, um die Datenverarbeitung zu vereinfachen. Die Abhängigkeiten in einer Schichtenarchitektur richten sich daher oft von der Businessschicht zur Datenzugriffsschicht \cite{Layered.SOLID}. Bei einer Applikation, die als Hauptbestandteil Businesslogik enthält, sollte hingegen die Abhängigkeiten zur Businessschicht fließen. Daher muss während des Entwicklungsprozesses stets die konkrete Einhaltung des DIPs beachtet werden, da entgegen der intuitiven Denkweise einer Schichtenarchitektur gearbeitet wird. 

Folglich bietet dieser Architekturansatz zwar einerseits einen hohen Grad an Simplizität, jedoch andererseits sind die SOLID-Prinzipien nur gering im Grundaufbau wiederzuerkennen. 

\pagebreak

\subsection{Hexagonale Architektur}

Durch architektonische Vorgaben können Entwickler zu besserem Softwaredesign gezwungen werden, ohne dabei die Implementierungsmöglichkeiten einzuengen. Dieser Denkansatz wird in der von Alistair Cockburn geprägten Hexagonalen Architektur angewandt, indem eine klare Struktur der Softwarekomposition vorgegeben wird. Der Aufbau wird in Abbildung \ref{fig:HexagonaleArchitektur} veranschaulicht. \\

\begin{figure}[H]
	\centering
	\includesvg[width=0.60\textwidth]{svg/HexagonaleArchitektur.svg}
	\caption{Grundstruktur einer Hexagonalen Architektur \cite[angelehnt an][]{hgraca.2017}}
	\label{fig:HexagonaleArchitektur}
\end{figure}

\vspace{0.3cm}

Hierbei existieren drei Bereiche in denen die Komponenten angesiedelt werden können: \cite{Cockburn.Hexagonal, Griffin.2021b}  

\textbf{Ports:} Die gesamte Kommunikation zwischen den Adaptern und dem Applikationskern findet über sogenannte \emph{Ports} statt. Diese dienen als Abstraktionsschicht, sorgen für Stabilität und schützen den Kern vor Codeänderungen anhand des \acrlong{OCP}s. Realisiert werden Ports meist durch Interfaces, welche hierarchisch dem Zentrum zugeteilt und deren Design durch diesen maßgeblich bestimmt werden. Somit erfolgt die Einhaltung des \emph{\acrlong{DIP}s}, wodurch die Applikationslogik von externen Systemen und deren konkreten Implementierungen abgekoppelt wird. Dies verringert die Abhängigkeiten zwischen Komponenten und erhöht zugleich die Testbarkeit der Anwendung. \cite{philipbrown.2014}

\pagebreak

\textbf{Adapter:} Die Komponenten zwischen externen Systemen und der Geschäftslogik heißen Adapter. Ein \emph{primärer Adapter} wird durch das externe System angestoßen, welcher daraufhin den Steuerfluss durch einen wohldefinierten Port in den Applikationskern trägt. Zu diesen externen Systemen zählen unter anderem Benutzerinterfaces, Kommandokonsolen sowie Testfälle. Andererseits bilden alle Komponenten, bei denen der Steuerfluss vom Applikationskern zu den externen Systemen gerichtet ist, die Gruppe der \emph{sekundären Adapter}. So entsteht der Impuls im Vergleich zu den primären Adaptern nicht außerhalb der Applikation, sondern innerhalb. Die von den sekundären Adaptern angesprochenen Systeme können beispielsweise Datenbanken, Message-Broker und weitere Nachbarapplikation sein.  \cite{hgraca.2017}

\textbf{Applikationskern:} Letztendlich werden alle übrigen Module im Applikationskern erschlossen. Diese beinhalten Businesslogik und sind mithilfe der von ihnen zur Verfügung gestellten Ports von konkreten Implementierungen entkoppelt. \\

Zum Verdeutlichen der Funktionsweise einer hexagonalen Applikation wird ein simpler Anwendungsfall durchgespielt. Konkret sollen von Clients übertragene Daten in einer Datenbank gespeichert werden. Ein Webclient spricht eine Schnittstelle des Systems mit den Nutzdaten an, wodurch er den Steuerfluss der Applikation initiiert. Die Schnittstelle ist den primären Adaptern zugeteilt und erledigt Aufgaben wie Authentifizierung, Datenumwandlung und erste Fehlerbehandlungen. Über einen entsprechenden Port wird der Kern mit den übergebenen Daten angestoßen. Innerhalb des Applikationszentrums werden alle business-relevanten Aufgaben erfüllt. Darunter fallen das logische Überprüfen der Werte anhand von Businessrichtlinien, Erstellen neuer Daten und die Steuerung des Entscheidungsflusses. In diesem Anwendungsfall sollen die Nutzdaten in einer Datenbank abgespeichert werden. Dementsprechend wird aus dem Anwendungskern über einen weiteren Port ein sekundärer Adapter aufgerufen, welcher die dauerhafte Speicherung in der Datenbank übernimmt. Anhand des Aufbaus einer Hexagonalen Architektur kann hinsichtlich der SOLID-Prinzipien im Vergleich zur Schichtenarchitektur folgendes Fazit formuliert werden:

\textbf{\acrlong{SRP}: } {Durch die Struktur wird eine strengere konzeptionelle Trennung der Verantwortlichkeiten ermöglicht. Dies wirkt sich positiv auf die Einhaltung des \acrlong{SRP}s aus.}

\textbf{\acrlong{OCP} \& \acrlong{ISP}: } {Als Folge der Nutzung von Ports zwischen dem Applikationskern und dem außenstehenden Komponenten ist die Anwendung der beiden Prinzipien erleichtert und teilweise automatisch gegeben. Die Applikation profitiert von erhöhter Stabilität und Kohäsion. }

\textbf{\acrlong{DIP}: } {In einer Hexagonalen Architektur ist das \acrlong{DIP} fest durch die vorgeschriebene Komposition verankert. Dadurch wird das Austauschen von Komponenten ermöglicht, ohne dabei den Businesskern verändern zu müssen. Dies entkoppelt nicht nur den wichtigsten Bestandteil der Software, sondern fördert schlussfolgernd auch die Testbarkeit. Durch eine native Invertierung der Abhängigkeiten gewinnt somit die Applikation viele positive Qualitätsmerkmale. \cite{Alliaume.2018, Martinez.2021}}

So ergibt sich eine natürliche Einhaltung der SOLID-Prinzipien, wobei der Applikationskern in den Vordergrund gerückt wird. Anzumerken ist, dass erfahrene Entwickler ebenfalls mit einer Schichtenarchitektur ein gleiches Maß an Softwarequalität erzielen können, sofern die Designprinzipien diszipliniert eingehalten werden, da bei genauer Betrachtung eine Hexagonale Architektur äquivalent mit einer dreiteiligen Schichtenarchitektur mit erzwungenem \acrlong{DIP} ist \cite{Seemann.2013} \cite[S. 125ff.]{Vernon.2015}. 


\section{Domain-Driven Design}

In der Entwicklungsphase von komplexer Software besteht stets die Gefahr zu einem sogenannten 'Big Ball of Mud' zu verschmelzen, weil die steigende Anzahl von Anforderungen und Codeänderungen die Übersichtlichkeit des Sourcecodes beeinträchtigt. Die bestehende Architektur wird unübersichtlich, Entstehungschancen für Bugs erhöhen sich und die Businessanforderungen sind überall in der Anwendung verteilt wiederzufinden. Somit kann die Wartbarkeit der Software nicht mehr gewährleistet werden und ihre Langlebigkeit ist stark eingeschränkt. \cite{bbom.1999} Die oben analysierten Architekturstile können bei strikter Umsetzung diese Risiken minimieren, jedoch bestimmen sie nur begrenzt, wie das zugrundeliegende Datenmodell und die damit verbundenen Komponenten gestaltet werden sollen. In dem Buch \citetitle{Evans.2011} entwickelte Eric Evans im Jahr 2003 zu diesem Zweck \acrlong{DDD}, kurz \acrshort{DDD}. Der Buchtitel beschreibt bereits den Hauptgedanken hinter Domain-Driven Design. Liegen die Businessanforderungen im Herzen der Software, sollte dementsprechend auch ihre Implementierung zentral verankert sein. Der Applikationskern stellt somit den 'lebenden' Teil der Anwendung dar. Die verbleibenden Komponenten dienen zur Unterstützung der Businesslogik, indem sie benötigte Dienste dem Kern bereitstellen. Die Businessanforderungen werden somit in DDD strukturell aus dem Quelltext hervorgehoben. Das Datenmodell spiegelt zudem die Sprache der Geschäftsprozesse wider, wodurch die Realisierung der Applikationslogik erleichtert wird. Vor allem Anwendungen mit komplexen Entscheidungssträngen und vielen jederzeit gültigen Konditionen können dadurch übersichtlich implementiert werden. Zu diesem Zweck definiert Domain-Driven Design einige Vorgehensweisen, Richtlinien und Entwurfsmuster, welche in diesem Kapitel erläutert werden. \cite{Evans.2011, Vernon.2015}

\subsection{Unterteilung der Problemebene in Domains und Subdomains}

Der Problemraum eines Projektes spannt in Domain-Driven Design eine \emph{Domain} auf \cite[S. 56]{Vernon.2015}. Dieser Bereich umfasst logisch zusammengehörige Verantwortlichkeiten und Businessprozesse. Anfangs sollte die Domain anhand einer ausführlichen Umfeldanalyse definiert werden, damit alle Aspekte des Problemraums und seine Abhängigkeiten beleuchtet werden. Innerhalb einer Domain liegen die dazugehörigen \emph{Subdomains}. Eine Subdomain repräsentiert einen kleineren, spezifischeren Teil der Domain, wodurch der Problemraum in mehrere Bereiche unterteilt wird. Sie helfen im nachfolgenden Schritt bei der Formulierung des Lösungsraumes. Zur Bestimmung der Subdomains werden die Verantwortlichkeiten stets aus Businesssicht betrachtet und technische Aspekte vernachlässigt. Der Domainumfang ist dabei entscheidend. Sollte dieser zu groß geschnitten sein, sind die Subdomains ebenfalls zu weitreichend. Das gefährdet die Kohäsion der Lösungsebene und somit der Software. Über den Verlauf der Entwicklungsphase könnten aufgrund dessen architektonische Konflikte auftreten. Enthält eine Subdomain mehrere logisch unabhängige Aufgaben, kann sie in kleinere Subdomains weiter unterteilt werden. Für einen Domain-Driven Ansatz ist es entscheidend die Definitionsphase gewissenhaft durchzuführen, damit eine stabile Grundlage für die Umsetzung des Projekts geschaffen werden kann.

\pagebreak

\subsection{Bounded-Contexts und ihre Ubiquitous Language}

Als Ausgangspunkt für die Bestimmung der Lösungsebene dienen die sogenannten Bounded-Contexts \cite[S. 57]{Vernon.2015}, welche eine oder mehrere Subdomains umfassen und ihre zugehörigen Verantwortlichkeiten bündeln. Wie es in der Praxis häufig der Fall ist, können Subdomains und Bounded-Contexts durchaus identisch sein \cite[S. 57]{Vernon.2015}. In jedem Bounded-Context sollte maximal ein Team tätig sein, um Kommunikationsprobleme zu vermeiden und eine klare Zuteilung der Kompetenzen zu gewährleisten \cite{Brandolini.2021}. Jeder Bounded-Context besitzt zudem eine zugehörige \emph{Ubiquitous Language} \cite[S. 62]{Vernon.2015}. Sie wird als wichtiger Bestandteil während der Projektplanung festgelegt und definiert Begriffe, welche durch die \emph{\Gls{Stakeholder}} und das Business verwendet werden. Dadurch können Missverständnisse in der Kommunikation zwischen dem Business und den Entwicklern vorgebeugt und eventuelle Inkonsistenzen aufgedeckt werden \cite[S. 336f.]{Evans.2011}. Der größte Vorteil ergibt sich allerdings, sobald auch das Datenmodell diese Sprache wiedergibt. Entities können als Nomen dargestellt, Funktionen können als Verben definiert und Aktionen können als Events abgebildet werden, wodurch die Businessprozesse auch im Quelltext wiederzufinden sind. Folglich wird die Verständlichkeit und Wartbarkeit der Software \cite[S. 24ff.]{Evans.2011} gesteigert. Zudem werden Entwickler bei der Umsetzung von Test- und Anwendungsfällen unterstützt, da ihre textuellen Definitionen auf das Datenmodell übertragbar sind. Zu beachten ist, dass die \emph{Ubiquitous Language} nur innerhalb eines Bounded-Context Gültigkeit hat \cite[S. 62]{Vernon.2015}. Beispielhaft kann der Begriff 'Kunde' in einem Onlineshop einen zivilen Endkunden, jedoch im Wareneingang eine Lieferfirma beschreiben. Daher ist bei der Kommunikation zwischen Teams in unterschiedlichen Subdomains zu berücksichtigen, dass Begriffe eventuell verschiedene Bedeutungen besitzen.

Die Domains, Subdomains, Bounded-Contexts und ihre Kommunikation zueinander wird durch eine Context-Map dargestellt. Diese ist ein wichtiges Artefakt der Definitionsphase und kann als Werkzeug zur Bestimmung von Verantwortlichkeiten und der Einteilung neuer Anforderungen genutzt werden. Sollte eine eindeutige Zuteilung von Funktionalitäten nicht möglich sein, spricht dies für die Entstehung eines neuen Bounded-Contexts und eventuell einer Subdomain. Wie eine Software Anpassungen erlebt, entwickelt sich gleichermaßen die Context-Map stetig weiter. \cite[S. 87ff]{Vernon.2015} Zur Veranschaulichung wurde in Abbildung \ref{fig:Context-Map-Example} das Personalwesen eines Unternehmens als Domain ausgewählt und in Subdomains bzw. Bounded-Contexts aufgeteilt. Abhängig von der Unternehmensgröße und -strategie können die Bounded-Contexts auch umfassender oder feingranularer ausfallen.

\begin{figure}[H]
	\vspace{0.2cm}
	\centering
	\footnotesize
	\includesvg[width=0.85\textwidth, height=0.85\textwidth]{svg/ExampleDomainV2.svg}
	\caption{Beispiel einer Context-Map anhand des Personalwesens einer Firma}
	\label{fig:Context-Map-Example}
\end{figure}

\subsection{Kombination von Domain-Driven Design und Hexagonaler Architektur}

Innerhalb eines Bounded-Contexts wird die grundlegende Architektur durch das zugehörige Team bestimmt. Diese kann sich je nach Sachverhalt des jeweiligen Anwendungsgebietes stark zwischen den Bounded-Contexts unterscheiden. Beliebte Modellierungs- und Designstile in Verbindung mit DDD sind unter anderem Microservices, \emph{\acrfull{CQRS}}, Event-Driven Design, Schichtenarchitektur und Hexagonale Architektur \cite[S. 113ff.]{Vernon.2015}. In den vorhergehenden Unterkapiteln wurden bereits die Vorzüge und Nachteile der zwei zuletzt genannten Architekturen erläutert. Auf Basis der Analyse wird generell für komplexere Software eine Hexagonale Architektur bevorzugt. Zudem steht im Zentrum von Domain-Driven Design und Hexagonaler Architektur das Domain-Modell, wodurch die Software an Kohäsion und Stabilität gewinnt. Die Kombination beider Ansätze ermöglicht es, bei häufigen technischen Neuerungen und komplexen Businessanforderungen weiterhin eine anpassbare, testbare und übersichtliche Software zu implementieren. Auf ein solches solides Grundgerüst wird mithilfe der Kenntnisse über den Bounded-Context das Domain-Modell gesetzt. Es umfasst sowohl die Datenhaltung als auch das zugehörige Verhalten, wie zum Beispiel die Überprüfung von Richtlinien, Modifikation von Attributen oder ihre dauerhafte Speicherung. Für diesen Zweck existieren in Domain-Driven Design mehrere Arten von Komponenten, welche anhand ihrer Verantwortlichkeiten unterschieden werden. Die korrekte Zuordnung der Klassen zu ihren Rollen ist entscheidend für einen skalierbaren Aufbau. Daher wird in den folgenden Unterkapiteln ein zentraler Überblick der einzelnen Bestandteile aufgeführt.

\subsection{Value Object}

Die Value Objects bilden eine Möglichkeit zusammengehörige Daten zu gruppieren. Entscheidend ist hierbei die Frage, durch welche Eigenschaft der Zusammenschluss identifiziert wird. Die Identität eines Value Object wird alleinig durch die Gesamtheit ihrer Attribute bestimmt. Somit sind zwei Value Objects mit gleichen Werten auch identisch und miteinander austauschbar ohne die Funktionalität der Software zu beeinflussen \cite[S. 227]{Vernon.2015}. Aus diesem Grund gelten Value Objects als \emph{\gls{immutable}}, da sie selbst keinen Werteverlauf besitzen \cite[S. 99]{Evans.2011}. Eine Neuzuweisung der Attribute ist deshalb nicht möglich. Stattdessen wird die Referenz auf eine andere, angepasste Instanz der Klasse umgesetzt \cite[S. 226]{Vernon.2015}. Dies gilt als ein erstrebenswertes Designmuster, da unveränderbare Objekte eine erhöhte Wiederverwendbarkeit ermöglichen und unerwünschte Seiteneffekte vermieden werden \cite[S. 228f.]{Vernon.2015}. Folglich sind sie aufgrund des fehlenden Lebenszyklus lediglich eine Momentaufnahme des Applikationszustandes.

\textbf{Beispiel:} In den meisten E-Commerce Bounded-Contexts sind alleinig die konkreten Werte eines \emph{Preises}, wie Bruttobetrag, Nettobetrag und Mehrwertsteuer relevant, weshalb dieser meist als Value Object angesehen wird. Sollten Preise die gleichen Wertebelegungen besitzen, gelten sie dementsprechend als identisch. Bei einer Aktualisierung eines Preises, kann das vorherige Objekt gelöscht und durch einen Preis mit den neuen Werten ersetzt werden. Ist es notwendig, den Werteverlauf des Preises über eine Zeitspanne zu verfolgen, wird oftmals eine ID innerhalb der Datenstruktur hinterlegt. Die Identität ist dadurch nur noch von der ID abhängig, nicht mehr von den Werten. Die Definition eines Value Objects trifft auf die Klasse nicht mehr zu und ein Design als Entity ist zu bevorzugen.  

\pagebreak

\subsection{Entity}

Im Gegenzug zu einem Value Object wird eine Entity nicht durch den Zusammenschluss ihrer Werte identifiziert, sondern enthält ein vordefiniertes Set an \gls{immutable} Attributen, welche ihre Eindeutigkeit bestimmen \cite[S. 94]{Evans.2011}. Auch nach dem Aktualisieren ihrer Informationen bleibt die ursprüngliche Identität bestehen. Demzufolge gelten die Attribute einer Entity als veränderlich und besitzen ihren eigenen Lebenszyklus, auch wenn dieser nicht explizit abgespeichert werden muss \cite[S. 172]{Vernon.2015}. In einer Entity werden Businessanforderungen, die sich auf enthaltenen Daten beziehen, direkt implementiert und ihre \emph{\Gls{Invariante}n} sichergestellt \cite[S. 208f.]{Vernon.2015}. Dadurch wird eine hohe Kohäsion erzeugt und entsprechend des \emph{\Gls{Information-Expert-Prinzip}s} korrekt verankert.

\textbf{Beispiel:} Ein \emph{Kunde} innerhalb eines Domainmodells ist meist eine Entity. In vielen Bounded-Contexts wird ein Kunde durch eine eindeutige ID ausgewiesen. Somit sind zwei Kunden mit identischen Namen dennoch nicht die gleichen Personen. Sollte der Name einer Person angepasst werden, ist ihre Identität weiterhin äquivalent zur vorhergehenden. Invarianten, wie die korrekte Formatierung der hinterlegten E-Mail, können beispielsweise direkt bei der Aktualisierung überprüft werden.


\subsection{Aggregate}

Innerhalb des Bounded-Contexts ist ein Aggregate der Verbund aus Entities und Value Objects, welcher von außen als eine Einheit wahrgenommen wird. Hierbei findet die Gruppierung anhand ihrer logischen Zusammengehörigkeit und Verantwortungen statt. Externe Komponenten dürfen beim Aufruf eines Aggregates nur auf das sogenannte Aggregate Root zugreifen und enthaltene Objekte nicht direkt referenzieren. Das Aggregate Root ist demzufolge eine Schnittstelle zwischen dem Aggregate und der Außenwelt. \cite[S. 126f.]{Evans.2011}

\textbf{Beispiel:} Ein mögliches Aggregate im Bereich des Personalmanagements ist ein \emph{Mitarbeiter}, welches Value Objects, wie \emph{Gehalt} und \emph{Abteilung}, beinhaltet. Die Klasse \emph{Mitarbeiter} ist auch zugleich ihr eigenes Aggregate Root. Bei Gehaltsanpassungen wird eine Funktion der Mitarbeiter-Klasse aufgerufen, welche das neue Gehalt durch Austausch des Value Objects einträgt. In diesem Schritt können Invarianten überprüft werden, sodass beispielsweise ein neues Gehalt nicht negativ oder niedriger als das vorgehende ausfallen darf. Abhängig vom jeweiligen Bounded-Context ist der Werteverlauf des Gehaltes eventuell relevant, weshalb die Klasse stattdessen als eine Entity realisiert werden sollte. 

Um die Effektivität von Aggregates zu gewährleisten, wurden in Domain-Driven Design einige Einschränkungen und Richtlinien beschlossen. Businessanforderungen bzw. Invarianten von enthaltenen Objekten müssen stets vor und nach einer Transaktion erfüllt sein. Dadurch sind die Grenzen der Aggregates durch den minimalen Umfang der transaktionalen Konsistenz ihrer Komponenten gesetzt \cite[S. 354]{Vernon.2015}. Als Folge dessen, wird immer das komplette Aggregate aus der Datenbank geladen und zurückgeschrieben. Große Aggregates leiden aus diesem Grund an reduzierter Skalierbarkeit und Performance, da die Datenmenge und notwendigen Operationen auf Seiten der Datenbank Last verursacht \cite[S. 355]{Vernon.2015}. Weiterhin sollte pro Transaktion jeweils nur ein Aggregate bearbeitet werden \cite[S. 354]{Vernon.2015}. Dies schränkt umfangreiche Aggregates durch fehlende Parallelität weiter ein. Unter Beachtung der letzten Regel wäre es nicht möglich das Gehalt und die Abteilung der Mitarbeiter-Klasse durch zwei unterschiedliche Personalmitarbeiter zeitgleich anzupassen. Eine der beiden Transaktionen würde sonst auf einen veralteten Stand operieren und müsste zur Vermeidung eines \Gls{Lost Update}s zurückgerollt werden. Sollte ein Anwendungsfall die Anpassung zweier Aggregates benötigt, kann das Konzept der Eventuellen Konsistenz angewandt werden. Dabei entsteht kurzzeitig ein inkonsistenter Stand der Daten, da zwei Transaktionen zeitversetzt gestartet werden. In vielen Fällen ist ein Verzug der Konsistenz aus Sicht der Businessanforderungen akzeptabel und damit eine mögliche Alternative für die Zusammenführung der beiden Aggregates. \cite[S. 364]{Vernon.2015}

\pagebreak

\subsection{Applicationservice}

Aufgaben, welche kein Domainwissen erfordern, werden in den Applicationservices realisiert. Ihre Aufgabe ist die Bereitstellung von notwendigen Dienstleistungen zur Abwicklung der Businesslogik \cite{Gorodinski.2012}. Dazu gehört das Management von Transaktionen, simple Ablaufsteuerung und Aufrufe anderer Services oder Aggregate Roots. Somit dürfen sie keine Businessanforderungen enthalten oder Invarianten überprüfen, da dies in den Zuständigkeitsbereich der Domainservices fällt \cite[S. 267]{Vernon.2015}. Die Namensgebungen der Klassen und ihrer Funktionen stammen meist aus Begriffen der Ubiquitous Language \cite[S. 105]{Evans.2011}. Um Nebeneffekte ausschließen zu können und Parallelität zu ermöglichen, müssen die Applicationservices zustandslos designt werden \cite[S. 105]{Evans.2011}.

\subsection{Domainservice}

Soweit anwendbar, werden Businessanforderungen direkt in den zuständigen Entities oder Value Objects umgesetzt. Allerdings existieren Fälle, in denen keine klare Zuteilung der Aufgaben möglich ist. Dies kann beispielsweise auftreten, wenn sich der Prozess über zwei oder mehr Aggregates spannt. In diesem Fall wird die Funktionalität in einem Domainservice ausgelagert. Sollte die auszuführende Logik Abhängigkeiten zu anderen Services besitzen oder die Kohäsion der Entity bzw. des Value Object verringert werden, ist dies ein weiterer Grund für die Nutzung eines Domainservices. Analog zu den Applicationservices werden sie zustandslos implementiert und stammen aus der Ubiquitous Language. Sie unterscheiden sich lediglich darin, dass es für Domainservices erlaubt ist, Businesslogik umzusetzen und Invarianten zu beinhalten. \cite[S. 268]{Vernon.2015}

\subsection{Factory}

Die wiederholte Erstellung von komplexen Objekten kann unnötigen Platz im Quelltext einnehmen und die Übersichtlichkeit einschränken, vor allem wenn zusätzliche Services zu diesem Zweck benötigt werden. Der Effekt wird verstärkt, wenn das Codefragment an verschiedenen Stellen auftritt. Zur Auslagerung der Objekterzeugung sind sogenannte Factories vorgesehen. Sie nehmen alle nötigen Daten entgegen und geben das gefragte Objekt zurück. Dadurch wird auch die Kohäsion der Applikation gestärkt. \cite[S. 137f.]{Evans.2011}

\subsection{Repository}

Eine Grundfunktion von Applikationen ist das Speichern und Laden von Daten. Repositories ermöglichen und orchestrieren hierbei den Datenbankzugriff \cite[S. 151]{Evans.2011}. In Domain-Driven Design benötigt jedes Aggregate ihr eigenes Repository, da sie unabhängig voneinander geladen werden müssen \cite[S. 401]{Vernon.2015}. Durch diese Zuordnung der Zuständigkeiten werden die konzeptionellen Abhängigkeiten der Domain von den Datenbanken getrennt. Die Kommunikation mit einem Repository sollte über ein fest definiertes Interface geschehen, damit bei Änderungen der darunterliegenden Datenbanktechnologie der Domainkern unbetroffen bleibt \cite[S. 152]{Evans.2011}. \\

Die erarbeiteten Grundgedanken von Domain-Driven Design, Hexagonaler Architektur und der SOLID-Prinzipien bildenden Rahmen für die Durchführung des Projekts. Im Folgenden wird diese Basis in der Planungsphase erweitert und die Checkout-Domain erschlossen.

  
  
%% PLANUNG %% 

\chapter{Planungs- und Analysephase}

Einleitend werden Struktur, Motivation und die abgeleiteten Forschungsfragen diskutiert.

\section{Umfeldanalyse}
\blindtext

\section{Anwendungsfälle}

\begin{itemize}[noitemsep,nolistsep]
	\item Basket erstellen und abrufen
	\item Basket finializing/cancel
	\item Produkte hinzufügen, löschen und Quantity ändern
	\item Lieferdaten abändern bzw hinzufügen ?
	\item Set checkout data
	\item Basket calculation
\end{itemize}

\begin{itemize}[noitemsep,nolistsep]
	\item Caching Funktion für Product und Price
\end{itemize}

\blindtext


\section{Aktuelles Design der Produktivanwendung}
\blindtext

  
  
%% PROOF OF CONCEPT %% 

\chapter{Festlegung des Datenmodells durch Domain-Driven Design}

Durch die Schaffung eines grundlegenden Verständnisses für Designprinzipien, Hexagonaler Architektur und Domain-Driven Design kann auf zusätzlicher Basis der vorherigen Analysen ein Proof-of-Concept der Checkout-Software entwickelt werden. Hierzu wird weiterhin das typische Vorgehen eines Domain-Driven Designs verfolgt und zunächst der Domainumfang und die Ubiquitous Language definiert, gefolgt vom Erstellen des zentralen Domain-Modells.

\section{Abgrenzung der Domain und Bounded Contexts mithilfe der Planungsphase}

Aufgrund der ausführlichen Vorbereitung wurde die Domain bereits passiv festgelegt und analysiert. Beispielsweise beschreibt das Context-Diagramm \ref{fig:ContextDiagramm} hierbei unsere Domaingrenzen. Eine Domain und die dazugehörigen Subdomains spannen den Problemraum über alle definierten Anwendungsfälle und Businessanforderungen auf \cite[S. 56]{Vernon.2015}. Die Größe der Domain ist entscheidend für die Bestimmung der Subdomains und des Bounded-Contexts. Wird der Checkout als eine Domain angesehen, ergibt sich insgesamt nur ein Bounded-Context, da grundlegend pro Bounded-Context nur maximal ein Team zuständig sein sollte \cite{Brandolini.2021}. Der Checkout müsste somit weitere Unterteilungen erfahren oder alternativ die Domain vergrößert werden. Folglich wird als nächstmögliche Eingrenzung der Checkout und alle abhängigen Systeme angesehen. Zu beachten ist hierbei, sich nicht auf die konkreten Systeme zu fixieren, da sie eher der Lösungsebene zuweisbar sind, sondern logisch naheliegende Aufgaben in einer Gruppe zusammenzufassen. In den Zuständigkeitsbereich der Domain fallen unter anderem Anforderungen an der Abwicklung des Zahlungs- und Bestellprozesses, sowie Bereitstellung von Preis- bzw. Artikelinformationen. Hierfür muss ebenfalls eine Verwaltungsmöglichkeit für diese Daten bereitgestellt werden. Die Abgrenzungen der Bounded-Contexts ist durch die jetzigen Überlegungen und die bereits bestehende Architektur vorgegeben, wodurch das Context-Diagramm \ref{fig:ContextDiagramm} zugleich als Context-Map fungieren kann.

\section{Festlegen einer Ubiquitous Language}

In der Kommunikation zwischen dem Business und Entwicklern kann es oft zu Missverständnissen kommen. Womöglich weil Informationen, Einschränkungen oder Prozesse ausgelassen bzw. für selbstverständlich erachtet werden. Durch die klare Definition von gemeinsam verwendeten Begriffen und ihren Bedeutungen wird implizit notwendiges Wissen über die Domain und ihre Eigenschaften geschaffen. Viele dieser Fachbegriffe können für Anwendungsfälle verwendet werden und machen die Personen, welche letztendlich die Businessanforderungen umsetzten sollen, mit der Domain vertraut. Da ein Team mit geringem Domainwissen auch die Korrektheit der Software gefährdet, ist die Ubiquitous Language ein wichtiger Meilenstein im Domain-Driven Design. \cite[S. 335ff.]{Evans.2011}

In Zusammenarbeit mit dem Lead-Developer und \emph{\Gls{Product Owner}} des Teams wird im folgenden Abschnitt die Ubiquitous Language definiert, um ein tieferes Verständnis über den Bounded-Context zu gewährleisten. Hierbei wurde sich auf die, für dieses Projekt, relevanten Terme beschränkt und ist lediglich eine mögliche Umsetzung einer Ubiquitous Language. Dank der Planungsphase sind zahlreiche Begriffe bereits definiert und helfen bei der Erstellung einer solchen Dokumentation. Eingeklammerte Wörter beschreiben Synonyme zu dem vorangestellten Ausdruck.

{\large \textbf{Ubiquitous Language des Domain-Modells:}}
\begin{itemize}[topsep=-3px]
	\item \textbf{Basket: } {Repräsentiert die Funktionalität eines Warenkorbs mit allen Artikeln, Kundeninformationen, Preisen etc.}
	\item \textbf{Basket Status: } {Stellt den aktuellen Zustand des Baskets dar, welcher sich an den Prozessen orientiert. Kann die Werte 'Open', 'Frozen', 'Finalized' und 'Canceled' annehmen. Der Startzustand ist hierbei 'Open'.}
	\item \textbf{Customer (Kunde): } {Ein Endkunde des Onlineshops oder im Markt. Kann eine zivile Person oder Firma sein. Einem nicht-eingeloggten Kunden wird der zugehörige Basket im Onlineshop durch seine eindeutige Session-ID zugewiesen.} 
	\item \textbf{Product (Artikel, Ware): } {Ein Artikel aus dem Warenbestand, welcher zu Verkauf steht. Kann ebenfalls für eine Gruppierung von mehreren Artikeln stehen.}
	\item \textbf{Outlet: } {Repräsentiert einen Markt oder den länderbezogenen Onlineshop, welcher durch eine einzigartige Outlet-Nummer referenziert wird.}
	\item \textbf{BasketItemID: } {Eine, innerhalb eines Baskets, eindeutige Referenz auf einen enthaltenen Artikel. Wird aus technischen Gründen benötigt, um Einträge zu bearbeiten oder entfernen.}
	\item \textbf{Net Amount (Nettobetrag): } {Ein Nettobetrag mit Währung.}
	\item \textbf{VAT (Steuersatz): } {Der Steuersatz eines zugehörigen Nettobetrags.}
	\item \textbf{Gross Amount (Bruttobetrag): } {Der Bruttobetrag eines Preises errechnet aus dem Steuersatz und Nettobetrag. Die Währung gleicht der des Nettobetrags.}
	\item \textbf{Fulfillment: } {Zustellungsart der Waren seitens der Firma.}
	\begin{itemize}[noitemsep,nolistsep, topsep=-5px]
		\item \textit{Pickup: } {Warenabholung in einem ausgewählten Markt durch den Kunden. Nur möglich sofern Artikel im Markt auf Lager ist.}
		\item \textit{Delivery: } {Zustellung der Ware an den Kunden durch einen Vertragspartner.}
		\item \textit{Packstation: } {Lieferung der Ware an eine ausgewählte Packstation durch einen Vertragspartner.}
	\end{itemize}
	\item \textbf{Payment Process (Zahlungsvorgang): } {Beinhaltet alle relevanten Informationen für das Verwalten eines Zahlungsvorgangs, wie Beträge und getätigte Zahlungen.}
	\item \textbf{Payment: } {Eine einzelne Zahlung des Kunden inklusive Betrag und Zahlungsmethode, wie Barzahlung oder PayPal.}
	\item \textbf{Order: } {Bestellung eines Baskets nach Abschluss des Zahlungsvorgangs. Wird durch nachfolgende Systeme angelegt und verwaltet.}
\end{itemize}

\pagebreak

{\large \textbf{Ubiquitous Language der Businessprozesse:}}
\begin{itemize}[topsep=-3px]
	\item \textbf{Touchpoint: } { Eine Komponente, welche mit der Checkout-Software interagiert, wie Kassensysteme im Markt, die Onlineshop-Seite oder Handy-App.}
	\item \textbf{Basket Cancellation: } {Stornieren des zugehörigen Baskets mithilfe eines Zustandswechsels auf 'Canceled'. Nach der Cancellation dürfen keine weiteren Änderungen an dem Basket durchgeführt werden. Der Zustand muss zuvor 'Open' sein. }
	\item \textbf{Basket Creation: } {Explizite oder Implizite Erstellung eines neuen Baskets. Geschieht automatisch sofern noch kein Basket für den Customer existiert oder nach einer Basket Finalization.}
	\item \textbf{Basket Calculation: } {Die Kalkulation von Bruttobeträgen aller Artikel sowie der Summe von beinhalteten Preisen des Baskets. Beträge aus unterschiedlichen Mehrwertsteuersätzen müssen weiterhin aus rechtlichen Gründen einzeln verwiesen werden können.}
	\item \textbf{High Volume Ordering: } {Die Bestellung von Artikeln in hoher Stückzahl. Aufgrund von Businessanforderungen soll es nur begrenzt möglich sein, dass ein Kunde innerhalb eines Baskets oder in mehreren Bestellungen das gleiche Produkt mehrfach kauft.}
	\item \textbf{Basket Validation: } {Durchführung einer Validierung des Baskets auf Inkonsistenzen oder fehlenden, jedoch notwendigen, Werten.}
	\item \textbf{Payment Initialization: } {Start des Zahlungsvorgangs, nachdem das Datenmodell auf invalide Zustände überprüft worden ist. Nur möglich bei einem offenen Basket, welcher Produkte und Payments enthält. Resultiert in den Zustand 'Frozen', wodurch keine weiteren Inhaltsänderungen an dem Basket vorgenommen werden können.}
	\item \textbf{Payment Execution: } {Durchführung des Zahlungsvorgangs eines gefrorenen Baskets. Anschließend findet die Basket Finalization statt.}
	\item \textbf{Basket Finalization: } {Erfolgt automatisch nach erfolgreichem Zahlungsvorgang und setzt den Basket in den Zustand 'Finalized'. Danach folgt die Reservierung der Produkte und das Anlegen einer neuen Bestellung.}
\end{itemize}
\vspace{2em}

Im Verlaufe der Definitionsphase der Ubiquitous Language wurden die Prozesse näher beleuchtet, Benamungen von Datenobjekten aufgedeckt und Businessanforderungen vorgegeben. Ein gutes Modell spiegelt die Sprache des Bounded-Contexts wider, weshalb auf Basis dieses Unterkapitels die Klassen designt werden. 

\pagebreak

\section{Definition der Value Objects}

Aufgrund der positiven Eigenschaften von Value Objects sollte anfangs jede Datenstruktur des Domain-Modells als ein solches implementiert und erst nach gründlicher Überlegung, falls die Notwendigkeit besteht, zu einer Entity umgeschrieben werden \cite[S. 219f.]{Vernon.2015}. Der Basket ist hierbei der Ausgangspunkt des Modells. Es wird auf ein schlankes Design im Vergleich zur Produktivanwendung geachtet, ohne dabei mögliche Aggregationsschnitte zu beeinflussen. Die Ubiquitous Language unterstützt bei der richtigen Klassen-Benamung. \\

\groupedDomainModell{Basket}{
	\item \textbf{BasketId: } {Eindeutige Identifikation des Baskets zur Referenzierung durch die Touchpoints.}
	\item \textbf{OutletId: } {Eine Referenz zugehörig zu einem Markt oder Onlineshop, durch welchen der Basket angelegt wurde. Unerlässlich für die Bestimmung von unter anderem Lagerbeständen, Lieferzeiten, Fulfillment-Optionen und Versandkosten.}
	\item \textbf{BasketStatus: } {Repräsentiert den aktuellen Zustand des Baskets. Mögliche Werte sind 'OPEN', 'FROZEN', 'FINALIZED' und 'CANCELED'.} 
	\item \textbf{Customer: } {Speichert Kundendaten (IdentifiedCustomer) oder Session-Informationen (SessionCustomer).} 
	\item \textbf{FulfillmentType: } {Lieferart, wie 'PICKUP' oder 'DELIVERY'.} 
	\item \textbf{BillingAddress: } {Adresse für die Rechnungserstellung.} 
	\item \textbf{ShippingAddress: } {Adresse für die Warenlieferung.} 
	\item \textbf{BasketItems: } {Liste aller enthaltenen Produkte und ihre zugehörigen Informationen.}
	\item \textbf{BasketCalculationResult: } {Beinhaltet die berechneten Werte des Basket, wie Nettobetrag, Bruttobetrag und Mehrwertsteuer. Die Speicherung dieser Werte wäre technisch nicht notwendig, spart aber Rechenzeit, da nicht bei jeder Abfrage des Basket dieser Wert neu berechnet werden muss.}
	\item \textbf{PaymentProcess: } {Enthält alle Informationen zur erfolgreichen Abwicklung des Zahlungsprozesses.}
	\item \textbf{Order: } {Speichert eine Referenz auf die Bestellung eines Baskets. Wird erst nach Zahlungsabschluss befüllt.}  
}

\vspace{1em}

Durch diese Datenstruktur ist es möglich, alle geforderten Anwendungsfälle korrekt abzuarbeiten. Die untergegliederten Klassen sind ebenfalls mit der gleichen Vorgehensweise in Anhang \ref{label:Daten-Modell} definiert worden, sofern sie nicht durch einen einfachen Text oder Aufzählungen realisierbar sind. Um eine klare Gesamtübersicht zu bieten, wurde ein Klassendiagramm der Datenstruktur dem Anhang \ref{fig:VO-Basket} hinzugefügt. Die Klassen \ul{Customer} und \ul{PaymentProcess} wurden aus Platzgründen in separate Klassendiagramme in Anhang \ref{fig:VO-Customer} und \ref{fig:VO-Payment} verlagert. Anzumerken ist, dass bei fehlender Multiplizität eine Eins-zu-Eins Beziehung vorliegt. Speziell, ist der Kunde in diesem Kontext genau einem Basket zugewiesen, da alleinig die Daten abgespeichert werden, nicht aber seine Kundennummer, wodurch keine Zuweisung zwischen einem Kunden und mehreren Warenkörben existiert.

\section{Bestimmung der Entities anhand ihrer Identität und Lebenszyklen}

Auf Basis der vorangehenden Sektion ist das Datenmodell nun vollständig definiert. Jedoch besteht weiterhin die Frage, ob die jeweiligen Klassen eine eigene Identität besitzen und somit als Entity designt werden müssen. Es existiert in Domain-Driven Design kein objektives Verfahren zur Bestimmung der Entities, da Datengruppierungen je nach Bounded-Context unterschiedliche Eigenschaften besitzen. Als Hilfestellung für diese Entscheidung können grundlegende Richtlinien aus Tabelle \ref{fig:entityvsvalueobject} verwendet werden.

\begin{table}[h!]
	\begin{tabular}{ | >{\centering\arraybackslash}m{0.16\textwidth} | m{0.33\textwidth} | m{0.43\textwidth} | } 
		\hline
		& \vspace{0.8mm}\textbf{Value Object}\vspace{0.5mm} & \vspace{0.8mm}\textbf{Entity}\vspace{0.5mm} \\ 
		\hline
		{\centering Identität} & 
		\centertable{Summe aller Attribute des Objekts. Objekte mit gleichen Werten besitzen gleiche Identität. \cite[S. 99]{Evans.2011}} & 
		\centertable{Bestimmt anhand eines Identifikators, zum Beispiel einer Datenbank-ID. Objekte gelten als ungleich, außer ihre Identifikatoren sind identisch. \cite[S. 92]{Evans.2011}} \\ 
		\hline
		Lebenszyklus & 
		\centertable{Stellt nur eine Momentaufnahme des Applikationszustands dar, da sie bei Änderungen ersetzt werden. \cite[S. 226]{Vernon.2015}} &
		\centertable{Werden zu einem bestimmten Zeitpunkt erstellt, bearbeitet, gespeichert oder gelöscht. Besitzen somit einen impliziten Verlauf ihrer Wertänderungen. \cite[S. 91]{Evans.2011}}  \\ 
		\hline
		Veränderbarkeit & 
		\centertable{Durch einen fehlenden Lebenszyklus gelten Value Objects als \gls{immutable}. \cite[S. 99]{Evans.2011}} &
		\centertable{Aufgrund ihrer Eigenschaften sind Entities veränderbar. \cite[S. 91]{Evans.2011}}  \\ 
		\hline
		Abhängigkeit & 
		\centertable{Nur als Unterobjekt von Entities persistierbar, da sie kein Aggregate Root sein können.} &
		\centertable{Damit ein eigener Lebenszyklus ermöglicht wird, können sie unabhängig von anderen Objekten existieren.}  \\ 
		\hline
		Zugriffsmethode & 
		\centertable{Auf Daten und Funktionen wird mithilfe einer Entität zugegriffen.} &
		\centertable{Können als Aggregate Root, oder durch dieses, direkten Zugriff erfahren. \cite[S. 129]{Evans.2011}} \\ 
		\hline
	\end{tabular}
	\caption{Vergleich zwischen Value Object und Entity}
	\label{fig:entityvsvalueobject}
\end{table}


Anhand dieser Eigenschaften können die Value Objects untersucht und daraufhin alle Entities bestimmt werden:

\textbf{Basket: } {Als zentrales Datenobjekt besitzt ein Basket zur eindeutigen Identifikation durch den Touchpoint eine Referenznummer. Diese Eigenschaft spricht stark für eine Entity. Zusätzlich bestimmen nicht die enthaltenen Attribute wie Products oder der zugehörige Kunde die Identität des Baskets, sondern alleinig dessen ID. Aufgrund der geforderten Anwendungsfälle entsteht zugleich ein Lebenszyklus für die Instanzen eines Baskets und er durchgeht verschiedene Zustandsänderungen. Folglich ist ein Basket eine \emph{Entity}.}

\textbf{IdentifiedCustomer: } {Werden innerhalb eines Bounded-Contexts die Kundendaten verarbeitet, stellen diese meinst eine Entity dar. In der Checkout-Domain finden keine Operationen auf den Informationen statt. Die vorangestellten Systeme senden bei Änderungen die aktualisierten Kundendaten an die Checkout-Software. Folglich besitzen sie keinen eigenen Lebenszyklus und können als \emph{Value Object} designt werden.}

\textbf{SessionCustomer: } {Die Identifikation dieses Objekts geschieht über die Session-ID. Dadurch ist ein SessionCustomer in der Gruppe der \emph{Entities} aufzuhängen. }

\pagebreak

\textbf{Basket-Item: } {Auf den ersten Blick ist ein Basket-Item als Entity zu designen. Es besitzt eine eigene ID und wird durch das Aktualisieren der Preise und Produktdaten bearbeitet. Sie haben somit einen Lebenszyklus. Jedoch lassen sich auch Argumente finden, warum ein Item durchaus ein Value Object sein kann. Die Identifikation erfolgt zwar durch eine ID, allerdings kann dies durch folgenden Anwendungsfall hinterfragt werden. Wenn das gleiche Produkt mehrmals sich im Basket befindet, existieren im Datenmodell auch mehrere zugehörige Basket-Items. Bei der Reduzierung der Stückzahl eines Artikels beispielsweise von vier auf eins, werden alle Items gesucht, welche das gleiche Produkt repräsentieren, und davon drei gelöscht. Dies würde bedeuten, dass ein Basket-Item zusätzlich anhand seines Produktes identifiziert wird. Sollte die ProductID die Identität des Baskets ausmachen, dann wären alle Basket-Items mit dem gleichen Produkt auch identisch. Dies stimmt allerdings nur bedingt, da sie sich theoretisch durch unterschiedliche Preise und Serviceangebote (in der Produktivumgebung) differenzieren können. Als Folgerung kann geschlossen werden, dass ein Basket-Item lediglich eine Momentaufnahme darstellt, wodurch das Design als Value Object berechtigt wäre. Letztendlich kann das Basket-Item in diesem Bounded-Context als Entity oder Value Object definiert werden. Für den Proof-of-Concept wurde das Basket-Item als Entity festgelegt. Die Begründung hierfür ist die schiere Anzahl von Datenanpassungen und Operationen auf einem Basket-Item, welche als \emph{Entity} anhand ihrer Veränderbarkeit natürlicher bewältigt werden können.}

\textbf{Product und Price: } {Die vorgehende Analyse des Basket-Items kann auch auf das Product und den Price angewendet werden. Beide besitzen eine ID und werden stetig aktualisiert. Dennoch sind Prices bzw. Products mit ungleichen Werten aber gleicher ID in dem Checkout-Kontext unterschiedliche Objekte. Beide Klassen sind als \emph{Value Object} umgesetzt worden, da der Zusammenschluss aller ihrer Attribute als Identifikationsmerkmal verwendet wird. }

\textbf{Calculation-Result: } {Als Datenstruktur, welche bei jeder Neuberechnung aktualisiert wird, könnte die Eigenschaft eines Lebenszyklus erfüllt sein. Allerdings ist die Klasse einzig ein Zwischenspeicher der Ergebnisse zur Performance-Verbesserung. Ohne den Kontext eines darüberlegenden, zugehörigen Objektes besitzen diese Daten keine Aussagekraft. Die gleichen Berechnungsergebnisse unterschiedlicher Baskets sind im Sinne der Identität äquivalent. Dadurch überwiegen die Argumente eines \emph{Value Objects}.}

\textbf{Payment-Process: } {Der Payment-Process besitzt zur Ablaufsteuerung einen eigenen Status, weshalb ein Lebenszyklus entsteht. Die Identität eines Payment-Processes ist gleich mit der BasketID, da eine Eins-zu-Eins Relation zwischen ihnen existiert. Die Lebensdauer des Objektes ist somit auch an die des Baskets gebunden. Weiterhin verwaltet ein Payment-Process alle darunterliegenden Payments. Zusammenfassend sprechen die Eigenschaften für ein Design als \emph{Entity}.}

\textbf{Payment: } {Ein Payment hat eine eindeutige ID, welche für den Ablauf des Bezahlprozesses und alle folgenden rechtlichen Prozesse eine hohe Relevanz hat. Dadurch ist weder der konkrete Betrag, noch die Bezahlmethode bei der Identifikation wichtig. Ähnlich zum Payment-Process ist auch hier ein Lebenszyklus in Form eines Statusfeldes vorhanden. Eine Umsetzung als \emph{Entity} ist zu empfehlen.}




  
  \chapter{Design der Aggregates anhand der Anwendungsfälle}
\begin{itemize}[noitemsep,nolistsep]
	\item Eine Einheit
	\item Behält Integrität
	\item Um abzuspeichern in Datenbank
	\item Eine Transaktion darf nur ein Aggregat bearbeiten
	\item Changes within can be inconsistent for a while until operation is done.
\end{itemize}

\section{Initiales Design als ein großes Basket-Aggregate}
\blindtext

\section{Trennung der Zahlungsinformationen von dem Basket}
\blindtext

\section{Verkleinerung der Aggregates durch Anpassung existierender Businessanforderungen }
\blindtext

\section{Maximale Unterteilung des Models mithilfe asynchroner Verarbeitung }
\blindtext

  
  \chapter{Implementierung des Proof-of-Concepts}

Im Verlaufe des Projekts wurde das Datenmodell mitsamt der Hexagonalen Architektur und allen relevanten Anwendungsfällen in einem Proof-of-Concept implementiert. Das Ziel dieser Software ist die Unterstreichung einer möglichen praktischen Umsetzung der Businessanforderungen unter Anwendung der gewonnenen Erkenntnisse. 

Zu Beginn wurde das Projekt mit benötigten Frameworks und Abhängigkeiten aufgesetzt, um \gls{Boilerplate-Code} weitestgehend zu vermeiden. Die Software wurde mithilfe von Kotlin entwickelt. Die Komponenten sind analog zu einer Hexagonalen Architektur aufgeteilt in primäre Adapter, Applikationskern und sekundäre Adapter. Innerhalb des Applikationskerns befindet sich jegliche Businesslogik, sowie die notwendigen Applicationservices, Domainservices und das Domainmodell entsprechend des Domain-Driven Designs. Die umgesetzten Aggregationsschnitte belaufen sich auf die Variante A und D, sowie ihren Abwandlungen. Dieses Kapitel geht auf die Entwicklung des ersteren Ansatzes mit einem einzelnen, großen Basket-Aggregate tiefer ein.

\section{Design der primären Adapter}

Primäre Adapter sind die grundlegenden Kommunikationsschnittstellen zwischen Clients und der Software, welche den Datenfluss anhand eines externen Signals initiieren. Im Proof-of-Concept fallen in diesen Bereich hauptsächlich die sogenannten Controller, welche für den Empfang von REST-API-Anfragen, die Deserialisierung übergebener Daten, sowie die Serialisierung des Antwortinhaltes zuständig sind. Zu Beginn jedes Anwendungsfalles wird ein Controller durch den Touchpoint angesprochen. Der jeweils zuständige Adapter wird aus dem Zusammenschluss der aufgerufenen URL und HTTP-Methode bestimmt. Ein Controller beinhaltet lediglich Logik für den Empfang von Daten und der Formulierung zugehöriger Antworten. Alle externen Informationen müssen vor der Weitergabe an den Applikationskern in ein Objekt des Domainmodells umgewandelt werden. Ist dies nicht der Fall, erhält der zentrale Teil der Software eine Abhängigkeit nach außen und das \acrlong{DIP} wird verletzt. Ein konkreter Controller ist im Codebeispiel \ref{lst:controller} abgebildet. Zur Implementierung dieser Funktionalitäten wurde das Framework 'Ktor' eingesetzt.

\vspace{0.5cm}
\begin{minipage}{\linewidth} % No pagebreak inside a minipage
\begin{lstlisting}[caption={Beispiel eines Controllers zum Aktualisieren von Kundendaten}, label={lst:controller}, language=Kotlin]
put("/basket/{id}/customer") {
	variable basketID = parseParameterFromUrl("id")
	variable customer = request.parseBody<Customer>() 
	
	variable basket = basketApiPort.setCustomer(basketID, customer)
	
	request.respond(HttpStatusCode.OK, basket)
}
\end{lstlisting}

\begin{itemize}
	\setlength\itemsep{-1pt}
	\item Zeile 1: Definiert die HTTP-Methode als 'PUT' und das Format der URL für diesen Endpunkt
	\item Zeile 2: Auslesen der BasketID aus der URL als Pfadparameter
	\item Zeile 3: Deserialisierung der übertragenen Daten zu einem Customer-Objekt
	\item Zeile 5: Weitergabe der Parameter an den zuständigen Applicationservice mithilfe eines Ports
	\item Zeile 7: Antwort an die Anfrage mit HTTP-Status '200' und den geänderten Basket
\end{itemize}
\end{minipage}

\pagebreak

Für jeden definierten Anwendungsfall ist ein entsprechender Port und Controller zuständig. Die tatsächliche Implementierung der Schnittstelle wird mittels \Gls{DI} durch das Framework 'Koin' geladen. Dadurch bleiben Abhängigkeiten jederzeit austauschbar und unabhängig testbar. Beispielsweise kann die korrekte Funktionsweise eines Controllers überprüft werden, indem der Applicationservice durch ein Test-Objekt ausgetauscht und Aufrufe des Objektes ausschließlich simuliert werden. Somit erfahren die einzelnen Komponenten in Testfällen keine Beeinflussung durch eventuell inkorrekt implementierten Code anderer Klassen und Test-Fehlschläge können eindeutig einem bestimmten Abschnitt der Software zugeschrieben werden. \cite{DI_2007, Lindooren.2007}

\section{Realisierung des Applikationskerns}

Der Applikationskern stellt das Herz der Anwendung dar. Das Ziel einer Hexagonalen Architektur ist es, das Zentrum komplett von äußeren Modulen zu entkoppeln \cite{Cockburn.Hexagonal}. In Domain-Driven Design liegen die Applicationservices, Domainservices und das Datenmodell im Inneren des Applikationskerns \cite[S. 125ff.]{Vernon.2015}. 

\subsection{Applicationservices}

Ein simples Beispiel für einen implementierten Applicationservice bietet uns der Basket-Item-Applicationservice, welcher bei Änderungen an den Items angesprochen wird. Der in Figur \ref{lst:applicationservice} dargestellte Code behandelt die Entfernung eines Basket-Items. Hierbei wird keine Businesslogik im Service verankert, lediglich das Transaktionsmanagement und die Ablaufsteuerung der Funktionsaufrufe von Aggregates bzw. Domainservices.

\vspace{0.5cm}
\begin{minipage}{\linewidth} % No pagebreak inside a minipage
	\begin{lstlisting}[caption={Funktion zum Entfernen von Basket-Items in einem Applicationservice}, label={lst:applicationservice}, language=Kotlin]
function removeBasketItem(BasketID basketID, BasketItemID basketItemID) {
	transaction {
		variable basket = basketRepository.findByID(basketID)
		basket.removeBasketItem(basketItemID)
		basketStorageService.store(basket)
	}
}
	\end{lstlisting}

	\begin{itemize}
		\setlength\itemsep{-1pt}
		\item Zeile 2: Starten einer Transaktion über die Zeilen 3 bis 5.
		\item Zeile 3: Laden eines Baskets anhand seiner ID durch ein Repository. 
		\item Zeile 4: Aufruf einer Funktion des Aggregate Roots zum Entfernen des übergebenen Items. Innerhalb dieser Funktion werden zusätzlich Aufgaben erledigt, wie das Neuberechnen des Gesamtpreises. Sollte die Kalkulation zu komplex ausfallen, kann ein Berechnungsservice als Parameter übergeben werden, sodass der Basket weiter für seine eigene Konsistenz verantwortlich ist. 
		\item Zeile 5: Speichern des Baskets mit den abgeänderten Daten.
	\end{itemize}
\end{minipage}

\pagebreak

\subsection{Basket-Aggregate}

Damit der Basket für seine eigene Konsistenz zuständig sein kann, müssen jegliche Änderungen durch eine Methode im Aggregate selbst geschehen. Anwendungsfälle, die es erfordern tiefer gelegene Objekte anzupassen, werden durch eine Kette von Funktionsaufrufen umgesetzt. Zur Trennung der Datenhaltung von den Funktionalitäten implementiert der Basket ein Interface. Dadurch kann das darunterliegende Datenmodell ausgetauscht oder in Tests simuliert werden. Der Codeauszug \ref{lst:basket} veranschaulicht das Abändern der Fulfillment Methode innerhalb des Baskets.

\vspace{0.5cm}
\begin{minipage}{\linewidth} % No pagebreak inside a minipage
	\begin{lstlisting}[caption={Setzen der Fulfillment Methode im Basket Aggregate}, label={lst:basket}, language=Kotlin]
function setFulfillment(Fulfillment fulfillment, FulfillmentPort fulfillmentPort) {
	validateIfModificationIsAllowed()
	
	variable availableFulfillment = fulfillmentPort.getAvailableFulfillment(outletID)
	
	throwIf(availableFulfillment.doesNotContain(fulfillment)) {
		IllegalModificationError("$fulfillment is not avaiable")
	}

	this.fulfillment = fulfillment
}
	\end{lstlisting}


	\begin{itemize}
		\setlength\itemsep{-1pt}
		\item Zeile 2: Überprüfung der Invariante, ob der Basket anhand seines Status aktuell Änderungen zulässt.
		\item Zeile 4: Laden aller verfügbarer Fulfillment Methoden für diesen Basket durch einen sekundären Adapter. Die Kommunikation mit dem Adapter erfolgt über einen Port.
		\item Zeile 6-8: Falls der neue Wert nicht unter den verfügbaren Fulfillments ist, wird der Aufruf zurückgewiesen und eine entsprechende Fehlermeldung an den Client durch den Controller geliefert.
		\item Zeile 10: Überschreiben des alten Wertes. Dieser Punkt wird nicht erreicht, wenn zuvor eine Businessanforderung gescheitert ist.
	\end{itemize}
\end{minipage}

\pagebreak

\subsection{Domainservices}

Aufgaben, welche nicht direkt einem Objekt zugewiesen werden können oder mehrere Aggregates betreffen sind in Domainservices zu implementieren \cite[S. 267]{Vernon.2015}. Beispielsweise wurde im Proof-of-Concept aus Gründen der Übersichtlichkeit und Kohäsion die Abwicklung des Bezahlverfahren aus dem Basket herausgetrennt und in einem Domainservice implementiert. In Code-Ausschnitt \ref{lst:domainservice} ist die Ausführung des Bezahlvorgangs abgebildet. 

\vspace{0.5cm}
\begin{minipage}{\linewidth} % No pagebreak inside a minipage
	\begin{lstlisting}[caption={Ausführung des Bezahlvorgangs in einem Domainservice}, label={lst:domainservice}, language=Kotlin]
function executePaymentProcessAndFinalizeBasket(BasketID basketID) {
	variable basket = basketRepository.findByID(basketID)
	
	throwIf(basket.isNotFrozen() or basket.paymentIsNotInitialized()) {
		IllegalModificationError("cannot cancel payment process")
	}

	variable externalPaymentRef = basket.getExternalPaymentRef()
	paymentPort.executePayment(externalPaymentRef)
	basket.executePayments() and basket.finalize()
	basketStorageService.store(basket)
	createOrderAfterFinalization(basket)
}
	\end{lstlisting}
	\begin{itemize}
		\setlength\itemsep{-1pt}
		\item Zeile 2: Laden des Baskets aus dem Repository.
		\item Zeile 4-6: Weist die Durchführung zurück, sofern der Basket sich nicht in dem erwarteten Zustand befindet. Dies kann auftreten, wenn die REST-API aufgerufen worden ist, ohne dass ein Zahlungsprozess zuvor gestartet wurde.
		\item Zeile 8-10: Durchführung des Bezahlvorgangs. Die erforderliche Aufrufreihenfolge stellt einen Teil des Domainwissens dar und begründet die Zuteilung der Klasse in die Gruppe der Domainservices.
		\item Zeile 11: Speichern des angepassten Baskets.
		\item Zeile 12: Erstellen eines Bestellvorgangs durch einen separaten Domainservice.
	\end{itemize}
\end{minipage}

\pagebreak

\section{Anbinden externer Systeme und Datenbanken durch sekundäre Adapter}

Das in der Planungsphase erstellte Context-Diagramm \ref{fig:ContextDiagramm} zeigt verschiedenste Systeme mit denen die Anwendung zum Erfüllen ihrer Aufgaben kommunizieren muss. Für diesen Zweck wurden Services, welche das Aufrufen externen API-Schnittstellen simulieren, erstellt. Damit Brücken zwischen der Domain und diesen Services gewährleistet sind, implementieren die sekundären Adapter ein vom Applikationskern definiertes Interface. Analog zu den primären Adaptern, existieren keine direkten Abhängigkeiten des Anwendungskerns zu dem Teil der Software.

Grundsätzlich wird pro Aggregate ein Repository implementiert. Sie verwalten den Zugriff auf die Datenbank und alle Funktionalitäten, welche in dieses Aufgabengebiet fallen, wie elementare Speicher- und Suchfunktionen. Zusätzlich existieren spezielle Komponenten für das Erfragen der aktuellen Preis- bzw. Artikelinformationen. Aufgrund von Performance-Verbesserungen wurden zusätzliche Abwandlungen der Adapter mit Caching-Funktion erstellt. Der normal fungierende Adapter ruft den zugehörigen API-Service auf, wohingegen der Caching-Adapter den Preis bzw. Artikel aus dem Cache lädt. Falls der Eintrag veraltet ist, wird der eigentliche Adapter angesprochen, um die zum jetzigen Zeitpunkt validen Daten zu erfragen und im Cache abzulegen. Das Beispiel \ref{lst:adapter} stellt die Komponente für das Aktualisieren der Preisinformationen dar. 

\vspace{0,5cm}
\begin{minipage}{\linewidth} % No pagebreak inside a minipage
	\begin{lstlisting}[caption={Preisadapter mit Caching-Funktion}, label={lst:adapter}, language=Kotlin]
class CachedPriceAdapter {
	
	variable PriceCachingRepository priceCachingRepository 
	variable PriceAdapter priceAdapter
	
	function fetchPrice(PriceID priceID) returns Price {
		return priceCachingRepository.getAndUpdateIfInvalid(priceID, fallback = {
			priceAdapter.fetchPrice(priceID)
		})
	}
}
	\end{lstlisting}
	\begin{itemize}
		\setlength\itemsep{-1pt}
		\item Zeile 3-4: Der \emph{CachedPriceAdapter} hat eine Abhängigkeit zum normalen Preisadapter und zu einem Repository zum Abrufen des zwischengespeicherten Preises
		\item Zeile 7: Abfragen des Preises aus dem Caching-Repository. Sollte der Preis invalide sein, weil er beispielsweise veraltet ist, wird Zeile 8 ausgeführt.
		\item Zeile 8: Weiterleitung der Anfrage an den zuständigen Adapter, welcher das externe System aufruft. Das Ergebnis wird mit einem aktuellen Zeitstempel im Cache abgelegt.
	\end{itemize}
\end{minipage}

  
  
%% FAZIT %% 

\chapter{Fazit und Empfehlungen}

Der Aufbau einer Hexagonalen Architektur unterstützt bei der Entkopplung des Applikationskerns und ermöglicht das instinktiv Einhalten der SOLID-Prinzipien. In dem Proof-of-Concept hat eine Einteilung in Adapter, Ports und Businesslogik den Entwicklungsprozess erleichtert. Zusätzlich lässt sich diese Softwarestruktur mit einem Domain-Driven Ansatz kombinieren. Die Hexagonale Architektur bildet ein stabiles und erweiterbares Fundament für die Checkout-Applikation.

Ein effektiver Aggregationsschnitt fördert die Skalierbarkeit und Performance der Anwendung. Damit ein Umbau der aktuellen Produktivsoftware zu empfehlen ist, sollten dementsprechend diese Qualitätsmerkmale positiv betroffen sein. 

Mithilfe der Performance-Analyse kann hierbei eine schnellere Bearbeitungszeit nicht verzeichnet werden. Dies lässt sich auf die vielzähligen Datenbankoperationen bei einem aufgeteilten Aggregationsschnitt zurückführen, welche aufgrund der starken Kopplung einzelner Entitäten untereinander entstehen. In vielen Softwareprojekten existieren kaum Invarianten, die sich über das ganze Domain-Modell spannen. Oftmals ist deshalb eine Trennung der Klassen voneinander problemlos möglich und ein verbessertes Aggregationsdesign kann dadurch erreicht werden. Der Warenkorb ist allerdings ein enger Verbund aus Businessrichtlinien und seine transaktionale Konsistenz muss wegen fiskalischen Anforderungen stets vorliegen. Ein allumfassender Aggregationsschnitt ist aus Businesssicht somit ebenfalls sinnvoll.

Zudem ist die mögliche parallele Bearbeitung eines Warenkorbs zum jetzigen Zeitpunkt nicht notwendig. Weshalb aus diesem Aspekt keine negativen Einflüsse durch den großen Aggregationsschnitt entstehen. Sofern zukünftig zeitgleiche Modifikationen an einem einzelnen Basket durch verschiedene Nutzer einen gängigen Anwendungsfall darstellen, müssen die Artikel als eigenständige Aggregates designt werden. Eine verringerte Performance und erhöhte Softwarekomplexität gehen im Gegenzug mit einem solchen Domain-Modell einher.

Das Abspalten der Preiskalkulation vom Warenkorb ist denkbar, um Berechnungszeiten einzusparen. Die Komplexität des Sourcecodes steigt in diesem Fall leicht an. Sollten die Touchpoints bei der Mehrheit der API-Anfragen jedoch auch zugleich die neu kalkulierten Werte erwarten, verfallen die Performance-Verbesserungen. Weitere Designvariationen sind, anhand der, in den jeweiligen Unterkapiteln besprochenen Auswirkungen zur Verbesserung der Qualitätsmerkmalen ungeeignet.

\textbf{Schlussendlich fallen die Argumente für einen Umgestaltung der Aggregationsaufteilung in Zusammenhang mit den aktuellen Businessanforderungen zu schwach aus, sodass ein Neudesign der Applikation nicht empfehlenswert ist.}

Allgemein kann sich in Softwareprojekten ein idealer Aggregationsschnitt mithilfe einer detaillierten Analyse der Anwendungsfälle und Integritätsgrenzen herauskristallisieren. Die ermittelten Invarianten zwischen Objekte bestimmen maßgeblich umsetzbare Designansätze. Ferner können verwendete Technologien durchaus einen Einfluss auf die Architektur der Software haben, allerdings sollte dies mit dem Bewusstsein geschehen, dass sich diese zeitnahe ändern können und weiterhin bei deren Einbindung in den Entscheidungsprozess ein Risiko entsteht. Anforderungen an die Applikation und in der Entwicklungsphase entstehende Kompromisse besitzen oftmals eine größere Priorität im Vergleich zu den theoretischen Prinzipien des Softwaredesigns. Diese bieten zwar Richtlinien für eine langlebige Anwendung, allerdings sind sie nicht immer die optimale Lösung für ein konkretes Problem. Der zugrundeliegende Antrieb für ihre Einhaltung sollte hinterfragt und bewusst gemacht werden. Nichtsdestotrotz ist in den meisten Fällen ein kleinerer Aggregationsschnitt zu bevorzugen, da viele Applikationen kaum starke Invarianten besitzen. Eine zukunftssicheres Software gelingt somit unter Einhaltung der gängigen Richtlinien eines Domain-Driven Designs in Kombination mit etablierten Designprinzipien und intuitiven Lösungsansätzen von erfahrenen Entwickler:innen.  

\comment{Zu wenig. Mehr ausformulieren. Mehr Zukunftsausblicke etc. Abruptes Ende...}
\comment{Service Mesh?}
  
  \backmatter

	\def\insideAnhang{0}

\newenvironment{anhang}{%
	\ifnum\insideAnhang=1%
	\errhelp={Let other blocks end at the beginning of the next block.}
	\errmessage{Nested Alpha section not allowed}
	\fi%
	\def\insideAnhang{1}
	
	\renewcommand\figurename{%
		\ifnum\insideAnhang=1% 
		Anhang\else%
		Abbildung\fi%
	}%
	
	\ifnum\insideAnhang=1%
	\def\table{\def\figurename{Anhang}\figure}
	\let\endtable\endfigure
	\fi%
	
	\renewcommand\thefigure{%
		\ifnum\insideAnhang=1% 
		\Alph{figure}\else%
		\arabic{figure}\fi%
	}%
	
}{%
	\def\insideAnhang{0}
}%

%%%%%%%%%%%%%%%%%%%%%%%%%%%%%%%%%%%%%%%%%%%%%%%%
%%%%%%%%%%%%%%%%%%%%%%%%%%%%%%%%%%%%%%%%%%%%%%%%
%%%%%%%%%%%%%%%%%%%%%%%%%%%%%%%%%%%%%%%%%%%%%%%%
%%%%%%%%%%%%%%%%%%%%%%%%%%%%%%%%%%%%%%%%%%%%%%%%

\begin{anhang}

\chapter{Anhang}

\section{Sourcecode des Proof-of-Concepts}
\label{label:sourcecode}

Die Binärdateien des Projekts wurden zur Versionierung in ein Git-Repository hinterlegt. Diese umfassen die Definition der Lasttests, API-Beschreibung, und den Quelltext, inklusive der analysierten Aggregationsschnitte. Das Repository kann unter dem Link '\url{https://github.com/Thalmaier/bachelor-thesis-checkout-poc}' aufgerufen werden. Alternativ ist in Anhang \ref{fig:Github} ein QR-Code abgebildet.

\begin{figure}[h]
	\centering
	\vspace{0.8cm}
	\includesvg[inkscapelatex=false, width=100px, height=100px]{svg/github qr code.svg}
	\caption{QR-Code des GitHub Repositories}
	\small URL: \hspace{0.3mm} \url{https://github.com/Thalmaier/bachelor-thesis-checkout-poc}
	\label{fig:Github}
\end{figure}

\newpage

\section{Aktivitätsdiagramme der Anwendungsfälle}

\begin{figure}[h!]
	\centering
	\includesvg[inkscapelatex=false, width=\textwidth]{svg/AD Basketcreation.svg}
	\caption{Aktivitätsdiagramm für die Erstellung eines Baskets}
	\label{fig:SL-Basketcreation}
\end{figure}

\begin{figure}[h!]
	\centering
	\includesvg[inkscapelatex=false, width=\textwidth]{svg/AD Basketstornierung.svg}
	\caption{Aktivitätsdiagramm für die Stornierung eines Baskets}
	\label{fig:SL-Basketstornierung}
\end{figure}

\begin{figure}[h!]
	\centering
	\includesvg[inkscapelatex=false, width=\textwidth]{svg/AD Checkoutdata.svg}
	\caption{Aktivitätsdiagramm für das Setzen der Checkout Daten}
	\label{fig:SL-Checkoutdata}
\end{figure}

\begin{figure}[h!]
	\centering
	\includesvg[inkscapelatex=false, width=\textwidth]{svg/AD PutBezahlmethode.svg}
	\caption{Aktivitätsdiagramm für das Hinzufügen einer Bezahlmethode}
	\label{fig:SL-PutBezahlmethode}
\end{figure}

\phantom{}
\newpage

\section{API-Endpunkte}

Die OpenAPI Definition der API ist in Anhang \ref{fig:Github} hinterlegt. Zudem bietet bietet Anhang \ref{fig:REST-API} eine grafische Übersicht der Endpunkte für Variante A des Aggregate-Designs.

\begin{figure}[h!]
	\centering
	\includesvg[inkscapelatex=false, width=\textwidth, height=0.85\textheight]{svg/REST API.svg}
	\caption{REST-API der Checkout-Software für diesen Proof-of-Concept}
	\label{fig:REST-API}
\end{figure}

\newpage
\section{Vollständiges Datenmodell des Proof-of-Concepts} \label{label:Daten-Modell}

\groupedDomainModell{Basket}{
	\item \textbf{BasketId: } {Eindeutige Identifikation des Baskets zur Referenzierung durch die Touchpoints.}
	\item \textbf{OutletId: } {Eine Referenz zugehörig zu dem Markt oder Onlineshop, durch welchen der Basket angelegt wurde. Unerlässlich für die Bestimmung von unter anderem Lagerbeständen, Lieferzeiten, Fulfillment-Optionen und Versandkosten.}
	\item \textbf{BasketStatus: } {Repräsentiert den aktuellen Zustands des Baskets. Mögliche Werte sind 'OPEN', 'FROZEN', 'FINALIZED' und 'CANCELED'.} 
	\item \textbf{Customer: } {Speichert Kundendaten (IdentifiedCustomer) oder Session-Informationen (SessionCustomer).} 
	\item \textbf{FulfillmentType: } {Lieferart, wie 'PICKUP' oder 'DELIVERY'.} 
	\item \textbf{BillingAddress: } {Adresse für die Rechnungserstellung.} 
	\item \textbf{ShippingAddress: } {Adresse für die Warenlieferung.} 
	\item \textbf{BasketItems: } {Liste aller enthaltenen Produkten und ihren zugehörigen Informationen.}
	\item \textbf{BasketCalculationResult: } {Beinhaltet die berechneten Werte des Basket, wie Nettobetrag, Bruttobetrag und Mehrwertsteuer. Die Speicherung dieser Werte wäre technisch nicht notwendig, spart aber an Rechenzeit ein, da nicht bei jeder Abfrage des Basket dieser Wert neu berechnet werden muss.}
	\item \textbf{PaymentProcess: } {Bindet alle Informationen zur erfolgreichen Abwicklung des Zahlungsprozesses.}
	\item \textbf{Order: } {Speichert eine Referenz auf die Bestellung für einen Basket. Wird erst nach Zahlungsabschluss befüllt.}  
}

\groupedDomainModell{SessionCustomer}{
	\item \textbf{SessionID: } {Eindeutige ID zur Zuweisung einer Session im Onlineshop zum zugehörigen Basket. Notwendig, um eine Einkaufmöglichkeit für anonyme Kunden zu bieten.}
}

\groupedDomainModell{IdentifiedCustomer}{
	\item \textbf{Name: } {Enthält den Vor- und Nachnamen als eigenes Datenkonstrukt.}
	\subDomainModell{
		\item \textit{FirstName: } {Vorname des Kunden.}
		\item \textit{LastName: } {Nachname des Kunden.}
	}
	\item \textbf{E-Mail: } {E-Mail des Kunden.}
	\item \textbf{CustomerTaxId: } {Die Steuernummer des Kunden. Relevant aus Sicht der Rechnungsabwicklung und für den Ausdruck der Rechnung.} 
	\item \textbf{BusinessType: } {Bestimmt ob Kunde als Business-to-Customer (B2C) oder Business-to-Business (B2B) gilt.}
	\item \textbf{CompanyName: } {Firmenname des Kunden. Kann optional angegeben werden oder ist verpflichtend für einen B2B-Kunden.}
	\item \textbf{CompanyTaxId: } {Steuernummer der Firma eines B2B-Kunden.}
}

\groupedDomainModell{BasketItem}{
	\item \textbf{Id: } {Eindeutige Referenz auf den Warenkorbeintrag.}
	\item \textbf{Product: } {Beinhaltet alle Produktinformationen, welche durch die Touchpoints benötigt werden.}
	\item \textbf{Price: } {Aktueller Preis des zugehörigen Products. Kann sich zeitlich ändern, muss daher durch eine Businessfunktion aktualisiert werden. }
	\item \textbf{ShippingCost: } {Betrag der Lieferkosten des Items.}
	\item \textbf{BasketItemCalculationResult: } {Speichert die Bruttokosten des Produktes, die errechneten Nettokosten, Lieferkosten und den Gesamtpreis.}
}

\groupedDomainModell{Product}{
	\item \textbf{Id: } {Eindeutige Referenz des Products im externen System.}
	\item \textbf{Name: } {Textuelle Produktbezeichnung des Products.}
	\item \textbf{Vat: } {Mehrwertsteuerinformationen des Products.}
	\item \textbf{UpdatedAt: } {Zeitstempel notwendig für die Aktualisierungsfunktion der Artikelinformationen.}
}

\groupedDomainModell{Vat}{
	\item \textbf{Sign: } {Identifizierung des Steuertyps, abhängig von jeweiligen Prozentsatz und Land.}
	\item \textbf{Rate: } {Prozentualer Wert der Mehrwertsteuer, wie beispielsweise '19\%'.}
}

\groupedDomainModell{Price}{
	\item \textbf{PriceId: } {Setzt sich zusammen aus der ProductId und der OutletId.}
	\item \textbf{GrossAmount: } {Bruttobetrag mit Währung.}
	\item \textbf{UpdatedAt: } {Zeitstempel notwendig für die Aktualisierungsfunktion des Preises.}
}

\groupedDomainModell{BasketItemCalculationResult}{
	\item \textbf{ItemCost: } {Beinhaltet Netto, Brutto und VAT Informationen in Form eines CalculationResults.}
	\item \textbf{ShippingCost: } {Betrag der Lieferkosten.}
	\item \textbf{TotalCost: } {Zusammengerechnete Werte der einzelnen Preise im Form eines CalculationResults.}
}

\groupedDomainModell{CalculationResult}{
	\item \textbf{GrossAmount: } {Bruttobetrag mit Währung.}
	\item \textbf{NetAmount: } {Nettobetrag mit Währung.}
	\item \textbf{VatAmounts: } {Eine zusammengebautes Set aus VatAmounts der Preise der BasketItems. Benötigt, da Vats mit unterschiedlichen Prozentbeträgen rechtlich nicht kombiniert werden dürfen.}
}

\groupedDomainModell{VatAmount}{
	\item \textbf{Sign: } {Identifizierung des Steuertyps, abhängig von genauen Prozentsatz und zugehörigen Land.}
	\item \textbf{Rate: } {Prozentualer Wert der Mehrwertsteuer.}
	\item \textbf{Amount: } {Berechneter Betrag der Mehrwertsteuer zugehörig zu einem Bruttobetrag.}
}

\groupedDomainModell{BasketCalculationResult}{
	\item \textbf{GrandTotal: } {Betrag der finalen Gesamtkosten des ganzen Baskets.}
	\item \textbf{NetTotal: } {Fasst alle Nettobeträge zusammen in einem einzelnen Betrag.}
	\item \textbf{ShippingTotal: } {Fasst alle Lieferkosten zusammen in einem einzelnen Betrag.}
	\item \textbf{VatAmount: } {Rechnet alle Vats zusammen, welche das gleiche Sign besitzen.}
}

\groupedDomainModell{PaymentProcess}{
	\item \textbf{BasketId: } {Id des zugehörigen Baskets.}
	\item \textbf{ExternalPaymentRef: } {Referenz auf den Bezahlvorgangs im externen System. Anfangs leer bis zur Initiierung des Payments.}
	\item \textbf{AmountToPay: } {Betrag der insgesamt bezahlt werden muss. Entspricht dem GrandTotal des Baskets.}
	\item \textbf{AmountPayed: } {Rechnet alle Payments zusammen und bestimmt in welchem Maße der Basket bereits bezahlt ist.}
	\item \textbf{AmountToReturn: } {Falls der bezahlte Betrag größer ist als gefordert, wird dieser Wert berechnet. Repräsentiert den Betrag, welcher durch das System zurückgegeben werden muss.}
	\item \textbf{PaymentProcessStatus: } {Status wieweit der der AmountToPay bezahlt ist. Kann die Werte 'TO\_PAY', 'PARTIALLY\_PAID' und 'PAID' annehmen.}
	\item \textbf{Payment: } {Liste aller Payments zugehörig zu diesem Prozess.}
}

\groupedDomainModell{Payment}{
	\item \textbf{PaymentId: } {Die Id der Zahlung.}
	\item \textbf{PaymentMethod: } {Bezahlungsart, wie Gutschein oder Barzahlung.}
	\item \textbf{PaymentStatus: } {Aktueller Zustand des Payments. Mögliche Werte entsprechen 'SELECTED', 'INITIALIZED', 'EXECUTED', 'CANCELED'. Ein Payment ist bei Hinzufügung im Status 'SELECTED'.}
	\item \textbf{AmountSelected: } {Betrag, welcher durch dieses Payment bezahlt werden soll. Falls dieser Wert leer ist, wird der gesamte Warenkorb durch dieses Payment bezahlt.}
	\item \textbf{AmountUsed: } {Betrag wie viel insgesamt durch dieses Payment abgedeckt wurde, falls nur ein Bruchteil des AmountSelectes benötigt wird.}
	\item \textbf{AmountOverpaid: } {Berechnet durch Subtraktion von AmountSelected und AmountUsed.}
}

\groupedDomainModell{Order}{
	\item \textbf{OrderRef: } {Referenz auf die Bestellung des Warenkorbs. Wird bei Abschluss des Zahlungsprozesses gesetzt.}
}

\clearpage
\section{Klassendiagramme des Datenmodells}

\vspace{1cm}
\begin{figure}[htbp]
	\centering
	\includesvg[inkscapelatex=false, width=0.95\textwidth]{svg/ValueObjectBasketDiagram.svg}
	\caption{Klassendiagramm eines Baskets}
	\label{fig:VO-Basket}
\end{figure}

\begin{figure}[htbp]
	\centering
	\includesvg[inkscapelatex=false, width=0.95\textwidth]{svg/ValueObjectCustomerDiagram.svg}
	\caption{Klassendiagramm des Customer Value Objects}
	\label{fig:VO-Customer}
\end{figure}

\begin{figure}[htbp]
	\centering
	\includesvg[inkscapelatex=false, width=0.95\textwidth]{svg/ValueObjectPaymentDiagram.svg}
	\caption{Klassendiagramm des Payment Process}
	\label{fig:VO-Payment}
\end{figure}

\section{Ergebnisse des Lasttests} \label{label:Lasttests}

Der Lasttest wurde mithilfe der Software 'JMeter' durchgeführt. Die Vorlage der jeweiligen Testfälle sind im Repository des Proof-of-Concepts unter Anhang \ref{fig:Github} zu finden. Ein Durchlauf bezieht sich auf einen typischen User Story, welche folgende Aspekte beinhaltet: Erstellen eines Baskets, dreimaliges Hinzufügen von Artikeln, Setzen der Checkout-Daten, zweimaliges Abrufen des Warenkorbs, Hinzufügen eines Payments und das Initiieren inklusive Durchführen des Bezahlvorgangs. 

Folgende Abkürzungen wurde zur Übersichtlichkeit genutzt:
\begin{itemize}[noitemsep,nolistsep]
	\item \textbf{A}: 'Variante A' des Aggregationsschnittes
	\item \textbf{D}: 'Variante D' des Aggregationsschnittes
	\item \textbf{M}: Verwendung des Datenbankmanagementsystems MongoDB
	\item \textbf{P}: Verwendung des Datenbankmanagementsystems PostgreSQL
	\item \textbf{F}: Kurz für 'Flag'. Die Kalkulation des Gesamtpreises geschieht erst bei expliziter Abfrage
	\item \textbf{C}: Die Kalkulation des Gesamtpreises findet umgehend bei Anpassungen von relevanten Werten statt.
	\item \textbf{AZ}: Kurz für 'Ablaufzeit'. Angabe in Millisekunden. Die Zeit für einen einzelnen Ablauf des Anwendungsfalles. 
\end{itemize}


\begin{landscape}
\begin{table}[h!]
	\centering
	\small
	\vspace{1cm}
	\begin{tabular}{ |c|c|c|c|c|c|c|c|} 
		\hline
		Name & Anzahl Durchläufe & min. AZ & max. AZ & durchschn. AZ & Median der AZ & Durchläufe pro Sekunde & Testdauer in Millisekunden \\ 
		\hline
		Variante A-M & 100 & 16 & 53 & 28,66 & 29 & 92,08 & 1086,00 \\
		\hline
		Variante A-M & 100 & 17 & 51 & 24,70 & 25 & 88,42 & 1131,00 \\
		\hline
		Variante A-M & 100 & 16 & 53 & 28,19 & 25 & 91,07 & 1098,00 \\
		\hline
		Variante A-M & 1000 & 16 & 83 & 43,56 & 42 & 197,36 & 5067,00 \\
		\hline
		Variante A-M & 1000 & 15 & 92 & 41,56 & 40 & 207,13 & 4828,00 \\
		\hline
		Variante A-M & 1000 & 16 & 86 & 41,53 & 41 & 205,00 & 4878,05 \\
		\hline
		Variante A-M & 10000 & 15 & 78 & 37,67 & 36 & 259,24 & 38574,00 \\
		\hline
		Variante A-M & 10000 & 15 & 76 & 36,86 & 35 & 265,32 & 37690,00 \\
		\hline
		Variante A-M & 10000 & 15 & 78 & 36,78 & 35 & 266,16 & 37571,00 \\
		\hline
		Variante D-FM & 100 & 21 & 55 & 29,10 & 29 & 88,18 & 1134,00 \\
		\hline
		Variante D-FM & 100 & 22 & 58 & 31,03 & 30 & 86,43 & 1157,00 \\
		\hline
		Variante D-FM & 100 & 22 & 60 & 33,25 & 31 & 86,73 & 1153,00 \\
		\hline
		Variante D-FM & 1000 & 21 & 75 & 42,46 & 41 & 199,60 & 5010,02 \\
		\hline
		Variante D-FM & 1000 & 22 & 83 & 44,97 & 44 & 189,47 & 5278,00 \\
		\hline
		Variante D-FM & 1000 & 21 & 80 & 43,41 & 42 & 195,69 & 5110,00 \\
		\hline
		Variante D-FM & 10000 & 21 & 73 & 37,00 & 35 & 264,63 & 37788,00 \\
		\hline
		Variante D-FM & 10000 & 20 & 97 & 36,82 & 35 & 266,36 & 37543,01 \\
		\hline
		Variante D-FM & 10000 & 20 & 72 & 36,83 & 35 & 265,82 & 37619,99 \\
		\hline
		Variante D-CM & 100 & 19 & 54 & 25,22 & 24 & 90,33 & 1107,00 \\
		\hline
		Variante D-CM & 100 & 18 & 60 & 28,40 & 27 & 90,42 & 1106,00 \\
		\hline
		Variante D-CM & 100 & 18 & 63 & 27,24 & 28 & 91,32 & 1095,00 \\
		\hline
		Variante D-CM & 1000 & 19 & 72 & 39,67 & 38 & 211,77 & 4722,00 \\
		\hline
		Variante D-CM & 1000 & 18 & 69 & 40,13 & 39 & 211,46 & 4729,00 \\
		\hline
		Variante D-CM & 1000 & 17 & 79 & 38,92 & 38 & 215,52 & 4640,00 \\
		\hline
		Variante D-CM & 10000 & 17 & 93 & 32,82 & 31 & 297,78 & 33582,00 \\
		\hline
		Variante D-CM & 10000 & 17 & 89 & 33,17 & 31 & 293,90 & 34025,00 \\
		\hline
		Variante D-CM & 10000 & 17 & 91 & 33,42 & 31 & 292,53 & 34185,00 \\
		\hline
	\end{tabular}
	\caption{Analyseergebnis des Lasttests der verschiedenen Variationen in Kombination mit MongoDB}
	\label{fig:performance-mongo}
\end{table}
\end{landscape}

\begin{landscape}
	\begin{table}[h!]
		\centering
		\small
		\vspace{1cm}
		\begin{tabular}{ |c|c|c|c|c|c|c|c|} 
			\hline
			Name & Anzahl Durchläufe & min. AZ & max. AZ & durchschn. AZ & Median der AZ & Durchläufe pro Sekunde & Testdauer in Millisekunden \\ 
			\hline
			Variante A-P & 100 & 37 & 65 & 48,08 & 48 & 76,75 & 1303,00 \\
			\hline
			Variante A-P & 100 & 34 & 77 & 55,08 & 55 & 73,58 & 1359,00 \\
			\hline
			Variante A-P & 100 & 35 & 80 & 55,04 & 54 & 70,95 & 1409,45 \\
			\hline
			Variante A-P & 1000 & 32 & 149 & 60,15 & 57 & 148,39 & 6739,00 \\
			\hline
			Variante A-P & 1000 & 33 & 133 & 60,02 & 57 & 147,21 & 6793,00 \\
			\hline
			Variante A-P & 1000 & 34 & 118 & 59,83 & 59 & 148,35 & 6741,00 \\
			\hline
			Variante A-P & 10000 & 33 & 93 & 57,28 & 57 & 170,35 & 58703,02 \\
			\hline
			Variante A-P & 10000 & 33 & 109 & 55,62 & 55 & 175,18 & 57083,00 \\
			\hline
			Variante A-P & 10000 & 33 & 138 & 55,61 & 55 & 174,23 & 57395,99 \\
			\hline
			Variante D-FP & 100 & 42 & 93 & 63,38 & 63 & 69,93 & 1430,00 \\
			\hline
			Variante D-FP & 100 & 42 & 90 & 63,53 & 63 & 70,13 & 1426,00 \\
			\hline
			Variante D-FP & 100 & 39 & 169 & 69,44 & 64 & 69,49 & 1439,00 \\
			\hline
			Variante D-FP & 1000 & 39 & 170 & 62,82 & 60 & 143,66 & 6961,00 \\
			\hline
			Variante D-FP & 1000 & 37 & 148 & 60,63 & 59 & 148,13 & 6751,00 \\
			\hline
			Variante D-FP & 1000 & 38 & 88 & 61,54 & 60 & 143,55 & 6966,00 \\
			\hline
			Variante D-FP & 10000 & 37 & 135 & 56,06 & 55 & 174,40 & 57338,00 \\
			\hline
			Variante D-FP & 10000 & 41 & 113 & 58,71 & 58 & 166,68 & 59994,98 \\
			\hline
			Variante D-FP & 10000 & 38 & 116 & 56,99 & 56 & 172,22 & 58063,98 \\
			\hline
			Variante D-CP & 100 & 31 & 72 & 47,26 & 44 & 78,93 & 1267,00 \\
			\hline
			Variante D-CP & 100 & 31 & 70 & 45,93 & 45 & 77,82 & 1285,00 \\
			\hline
			Variante D-CP & 100 & 30 & 74 & 46,81 & 46 & 80,13 & 1248,00 \\
			\hline
			Variante D-CP & 1000 & 30 & 78 & 48,06 & 47 & 180,86 & 5529,00 \\
			\hline
			Variante D-CP & 1000 & 30 & 105 & 50,88 & 49 & 170,68 & 5859,00 \\
			\hline
			Variante D-CP & 1000 & 30 & 75 & 50,10 & 50 & 173,97 & 5748,00 \\
			\hline
			Variante D-CP & 10000 & 34 & 94 & 48,87 & 49 & 199,60 & 50099,00 \\
			\hline
			Variante D-CP & 10000 & 34 & 111 & 49,95 & 50 & 195,57 & 51132,01 \\
			\hline
			Variante D-CP & 10000 & 34 & 93 & 48,28 & 48 & 200,92 & 49771,99 \\
			\hline
		\end{tabular}
		\caption{Analyseergebnis des Lasttests der verschiedenen Variationen in Kombination mit Postgres}
		\label{fig:performance-postgres}
	\end{table}
\end{landscape}

\begin{landscape}
	\begin{table}[h!]
		\centering
		\small
		\vspace{1cm}
		\begin{tabular}{ |c|c|c|c|c|c|c|c|} 
			\hline
			Name & Anzahl Durchläufe & min. AZ & max. AZ & durchschn. AZ & Median der AZ & Durchläufe pro Sekunde & Testdauer in Minuten\\ 
			\hline
			Variante A-M & 10000 & 563 & 1526 & 612,04 & 612 & 16,30 & 10,22 \\
			\hline
			Variante A-M & 10000 & 565 & 725 & 617,67 & 617 & 16,06 & 10,38 \\
			\hline
			Variante A-M & 10000 & 576 & 743 & 619,05 & 617 & 16,04 & 10,39 \\
			\hline
			Variante D-FM & 10000 & 1405 & 2904 & 1625,89 & 1612 & 6,13 & 27,19 \\
			\hline
			Variante D-FM & 10000 & 1510 & 1720 & 1608,00 & 1608 & 6,19 & 26,91 \\
			\hline
			Variante D-FM & 10000 & 1512 & 1970 & 1654,91 & 1646 & 6,01 & 27,72 \\
			\hline
			Variante D-CM & 10000 & 1210 & 3307 & 1380,43 & 1364 & 7,23 & 23,06 \\
			\hline
			Variante D-CM & 10000 & 1207 & 1675 & 1384,97 & 1368 & 7,17 & 23,26 \\
			\hline
			Variante D-CM & 10000 & 1219 & 1637 & 1374,96 & 1365 & 7,21 & 23,11 \\
			\hline
			Variante A-P & 10000 & 3354 & 4237 & 3848,23 & 3836 & 2,49 & 66,96 \\
			\hline
			Variante A-P & 10000 & 3402 & 4305 & 3769,70 & 3765 & 2,59 & 64,46 \\
			\hline
			Variante A-P & 10000 & 3364 & 4911 & 3833,21 & 3796 & 2,36 & 70,67 \\
			\hline
			Variante D-FP & 10000 & 5386 & 6919 & 5839,03 & 5804 & 1,66 & 100,49 \\
			\hline
			Variante D-FP & 10000 & 5168 & 6883 & 5778,68 & 5725 & 1,52 & 109,65 \\
			\hline
			Variante D-FP & 10000 & 5235 & 6174 & 5727,97 & 5720 & 1,67 & 99,97 \\
			\hline
			Variante D-CP & 10000 & 4189 & 5799 & 4571,81 & 4549 & 2,11 & 79,14 \\
			\hline
			Variante D-CP & 10000 & 3919 & 5553 & 4553,76 & 4554 & 2,12 & 78,61 \\
			\hline
			Variante D-CP & 10000 & 4000 & 5702 & 4544,34 & 4527 & 2,02 & 82,69 \\
			\hline
		\end{tabular}
		\caption{Lasttest-Ergebnisse mit Datenbanken von einem externen Cloud-Anbieter}
		\label{fig:performance-database}
	\end{table}
\end{landscape}



\begin{landscape}
	\begin{table}[h!]
		\centering
		\small
		\vspace{1cm}
		\begin{tabular}{ |c|c|c|c|c|c|c|c|} 
			\hline
			Name & Anzahl Durchläufe & min. AZ & max. AZ & durchschn. AZ & Median der AZ & Durchläufe pro Sekunde & Testdauer in Minuten\\ 
			\hline
			Variante A-M & 10000 & 488 & 605 & 504,39 & 503 & 19,68 & 8,47 \\
			\hline
			Variante A-M & 10000 & 489 & 585 & 502,43 & 501 & 19,71 & 8,45 \\
			\hline
			Variante A-M & 10000 & 490 & 604 & 502,69 & 501 & 19,73 & 8,45 \\
			\hline
			Variante D-FM & 10000 & 494 & 604 & 511,01 & 511 & 19,39 & 8,59 \\
			\hline
			Variante D-FM & 10000 & 494 & 600 & 509,20 & 508 & 19,38 & 8,60 \\
			\hline
			Variante D-FM & 10000 & 492 & 601 & 508,31 & 507 & 19,50 & 8,55 \\
			\hline
			Variante D-CM & 10000 & 490 & 575 & 503,62 & 502 & 19,69 & 8,46 \\
			\hline
			Variante D-CM & 10000 & 491 & 627 & 503,50 & 502 & 19,68 & 8,47 \\
			\hline
			Variante D-CM & 10000 & 491 & 571 & 505,90 & 504 & 19,61 & 8,50 \\
			\hline
			Variante A-P & 10000 & 507 & 589 & 522,61 & 522 & 18,97 & 8,78 \\
			\hline
			Variante A-P & 10000 & 507 & 630 & 519,33 & 518 & 19,08 & 8,73 \\
			\hline
			Variante A-P & 10000 & 507 & 602 & 521,18 & 520 & 19,03 & 8,76 \\
			\hline
			Variante D-FP & 10000 & 511 & 594 & 526,00 & 525 & 18,86 & 8,84 \\
			\hline
			Variante D-FP & 10000 & 511 & 600 & 526,29 & 525 & 18,82 & 8,85 \\
			\hline
			Variante D-FP & 10000 & 512 & 626 & 526,81 & 525 & 18,80 & 8,87 \\
			\hline
			Variante D-CP & 10000 & 504 & 580 & 517,00 & 516 & 19,14 & 8,71 \\
			\hline
			Variante D-CP & 10000 & 502 & 611 & 518,79 & 518 & 19,11 & 8,72 \\
			\hline
			Variante D-CP & 10000 & 505 & 594 & 518,93 & 518 & 19,09 & 8,73 \\
			\hline
		\end{tabular}
	\caption{Lasttest-Ergebnisse mit Simulation der API-Aufruf durch künstliche Verzögerung}
	\label{fig:performance-delay}
	\end{table}
\end{landscape}

\end{anhang}

  
  %Refs
  %TODO: Add alot of references
  %TODO: Add Effective Aggregate Design
  \addcontentsline{toc}{chapter}{Literatur}
  \printbibliography %Citavi 5
  
\end{document}